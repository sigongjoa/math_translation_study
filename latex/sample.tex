\documentclass[10.5pt, a4paper, twoside, openany]{book}

% ─── 여백 ───
\usepackage[
  top=25mm,
  bottom=20mm,
  inner=30mm,
  outer=25mm,
  bindingoffset=5mm,
  headheight=14pt,
  headsep=12pt
]{geometry}

% ─── 한글 ───
\usepackage{kotex}
\usepackage{fontspec}

% ─── 폰트 ───
\setmainfont{Noto Serif CJK KR}[
  UprightFont={Noto Serif CJK KR},
  BoldFont={Noto Serif CJK KR Bold},
  Ligatures=TeX,
]
\setsansfont{Noto Sans CJK KR}[
  UprightFont={Noto Sans CJK KR},
  BoldFont={Noto Sans CJK KR Bold},
  Ligatures=TeX,
]
\setmonofont{Noto Sans Mono CJK KR}[Scale=0.85]

% Latin fallback for headers
\newfontfamily\headerfont{Noto Sans CJK KR}[Ligatures=TeX]

% ─── 수식 ───
\usepackage{amsmath, amssymb, amsthm}

% ─── 행간 ───
\linespread{1.52}

% ─── 단락 ───
\setlength{\parindent}{1em}
\setlength{\parskip}{0pt}

% ─── 헤더/푸터 ───
\usepackage{fancyhdr}
\pagestyle{fancy}
\fancyhf{}
\fancyhead[LE]{{\small\sffamily I.4\quad 수학 연구의 일반적 목표 \hfill \thepage}}
\fancyhead[RO]{{\small\sffamily \thepage \hfill 2.2\quad 동치, 비동치, 그리고 불변량}}
\renewcommand{\headrulewidth}{0.4pt}
\renewcommand{\footrulewidth}{0pt}

% ─── 제목 스타일 ───
\usepackage{titlesec}

\titleformat{\chapter}[hang]
  {\sffamily\bfseries\LARGE}
  {\thechapter}{12pt}{}
  [\vspace{2pt}{\titlerule[0.8pt]}]
\titlespacing*{\chapter}{0pt}{-10pt}{24pt}

\titleformat{\section}[hang]
  {\sffamily\bfseries\Large}
  {\thesection}{8pt}{}
\titlespacing*{\section}{0pt}{20pt}{8pt}

\titleformat{\subsection}[hang]
  {\sffamily\bfseries\normalsize}
  {\thesubsection}{6pt}{}
\titlespacing*{\subsection}{0pt}{14pt}{6pt}

% ─── 정리/정의 박스 ───
\usepackage[most]{tcolorbox}

\newtcolorbox{definitionbox}[1][]{
  colback=white, colframe=black,
  fonttitle=\sffamily\bfseries,
  title=#1,
  breakable, sharp corners,
  boxrule=0.5pt,
  left=8pt, right=8pt, top=6pt, bottom=6pt
}

% ─── 교차참조 ───
\usepackage{hyperref}
\hypersetup{
  colorlinks=false,
  pdfborder={0 0 0},
}

% ─── 본문 시작 ───
\begin{document}

\setcounter{chapter}{3}
\setcounter{page}{76}

% ═══ Chapter 시작 ═══
\chapter{수학 연구의 일반적 목표}

수학에서 좋은 문제의 가장 중요한 특징은 아마도 \textbf{일반성}일 것이다. 좋은 문제의 풀이는 대개 그 문제 자체를 넘어서는 파급력을 가져야 한다. 이 바람직한 성질을 더 정확하게 표현하자면 ``일반화 가능성''이라 할 수 있다. 왜냐하면 일부 훌륭한 문제들은 겉보기에 매우 구체적으로 보일 수 있기 때문이다.

예를 들어, $\sqrt{2}$가 무리수라는 명제는 단 하나의 수에 관한 것처럼 보인다. 그러나 일단 이를 증명하는 방법을 알고 나면, $\sqrt{3}$이 무리수임을 증명하는 데에도 아무런 어려움이 없으며, 사실 이 증명은 훨씬 더 넓은 부류의 수들로 일반화할 수 있다 (대수적 수~\mbox{[IV.1\S14]} 참조).

좋은 문제는 그것에 대해 생각하기 전까지는 흥미롭지 않아 보이는 경우가 꽤 흔하다. 그러다 문득 그 문제가 어떤 이유에서 제기된 것인지를 깨닫게 된다. 그것은 더 일반적인 문제의 ``첫 번째 어려운 사례''일 수도 있고, 아니면 동일한 난관에 부딪히는 여러 문제들 가운데 잘 선택된 하나의 예일 수도 있다.

% ═══ Section ═══
\section{동치, 비동치, 그리고 불변량}

수학의 여러 상황에서, 일단 특정한 일반적 성질들이 유용하다고 확립되면, 다음과 같은 형태의 광범위한 수학적 질문들이 생겨난다: 주어진 수학적 구조와 그것이 가질 수 있는 흥미로운 성질들의 목록이 있을 때, 어떤 성질들의 조합이 다른 어떤 성질들을 함의하는가?

이러한 질문들이 모두 흥미로운 것은 아니다---많은 것들이 꽤 쉽게 풀리고, 다른 것들은 너무 인위적이다---그러나 그 중 일부는 매우 자연스러우면서도 처음 풀어보려 할 때 놀라울 정도로 난해하다. 이것은 대개 수학자들이 ``깊은'' 질문이라 부를 만한 것을 발견했다는 신호이다.

\begin{definitionbox}[정의: 유한 생성 군]
군 $G$가 \textbf{유한 생성}(finitely generated)이라 함은, $G$의 원소들로 이루어진 유한 집합 $\{x_1, x_2, \ldots, x_k\}$가 존재하여 $G$의 나머지 모든 원소를 이 집합의 원소들의 곱으로 나타낼 수 있는 것을 말한다.
\end{definitionbox}

\vspace{4pt}

예를 들어, 군 $\mathrm{SL}_2(\mathbb{Z})$는 $a, b, c, d$가 정수이고 $ad - bc = 1$인 모든 $2 \times 2$ 행렬 $\bigl(\begin{smallmatrix} a & b \\ c & d \end{smallmatrix}\bigr)$로 이루어진다. 이 군은 유한 생성이다: 모든 그러한 행렬이 네 개의 행렬
\[
\begin{pmatrix} 1 & 1 \\ 0 & 1 \end{pmatrix}, \quad
\begin{pmatrix} 1 & -1 \\ 0 & 1 \end{pmatrix}, \quad
\begin{pmatrix} 1 & 0 \\ 1 & 1 \end{pmatrix}, \quad
\begin{pmatrix} -1 & 0 \\ 1 & 1 \end{pmatrix}
\]
의 행렬 곱으로 만들어질 수 있음을 보이는 것은 좋은 연습 문제이다 (행렬에 대한 논의는~\mbox{[I.3\S3.2]}를 참조하라).

\subsection{위수와 아벨 군}

이제 두 번째 성질을 살펴보자. 군 $G$의 원소 $x$가 \textbf{유한 위수}(finite order)를 가진다 함은, $x$의 어떤 거듭제곱이 항등원과 같아지는 것을 말한다. 그러한 거듭제곱 중 가장 작은 것을 $x$의 \textbf{위수}라 부른다.

예를 들어, 법~$7$에 대한 0이 아닌 정수들의 곱셈군에서 항등원은~$1$이고, 원소~$4$의 위수는~$3$이다:
\[
4^1 = 4, \qquad 4^2 = 16 \equiv 2, \qquad 4^3 = 64 \equiv 1 \pmod{7}.
\]
한편, $3$의 처음 여섯 거듭제곱은 $3, 2, 6, 4, 5, 1$이므로 $3$의 위수는~$6$이다. 이제 어떤 군들은 매우 특별한 성질을 갖는데, 어떤 정수 $n$이 존재하여 모든 $x$에 대해 $x^n$이 항등원과 같아지는---동치적으로, 모든 $x$의 위수가 $n$의 약수인---경우이다.

먼저 모든 원소가 위수 $2$를 갖는 경우를 살펴보자. 항등원을 $e$로 쓰면, 모든 원소 $a$에 대해 $a^2 = e$라고 가정하는 것이다. 이 등식의 양변에 역원 $a^{-1}$을 곱하면 $a = a^{-1}$을 얻는다. 역방향의 추론도 마찬가지로 쉬우므로, 이러한 군은 모든 원소가 자기 자신의 역원인 군이다.

이제 $G$의 두 원소 $a$와 $b$를 생각해 보자. 임의의 군에서 임의의 두 원소 $a, b$에 대해
\[
(ab)^{-1} = b^{-1} a^{-1}
\]
이 성립한다 ($abb^{-1}a^{-1} = aa^{-1} = e$이기 때문이다). 모든 원소가 자기 자신의 역원인 우리의 특별한 군에서는, 이로부터 $ab = ba$를 이끌어낼 수 있다. 즉, $G$는 자동으로 \textbf{아벨 군}이 된다.

\end{document}
