\documentclass[10.5pt, a4paper, twoside, openany]{book}

% ─── 여백 ───
\usepackage[
  top=25mm,
  bottom=20mm,
  inner=30mm,
  outer=25mm,
  bindingoffset=5mm,
  headheight=14pt,
  headsep=12pt
]{geometry}

% ─── 한글 ───
\usepackage{kotex}
\usepackage{fontspec}

% ─── 폰트 ───
\setmainfont{Noto Serif CJK KR}[
  UprightFont={Noto Serif CJK KR},
  BoldFont={Noto Serif CJK KR Bold},
  Ligatures=TeX,
]
\setsansfont{Noto Sans CJK KR}[
  UprightFont={Noto Sans CJK KR},
  BoldFont={Noto Sans CJK KR Bold},
  Ligatures=TeX,
]
\setmonofont{Noto Sans Mono CJK KR}[Scale=0.85]

% ─── 수식 ───
\usepackage{amsmath, amssymb, amsthm}

% ─── 그래픽 ───
\usepackage{graphicx}
\usepackage{tikz}
\usetikzlibrary{arrows.meta, positioning, shapes, calc, decorations.pathreplacing}

% ─── 한글 줄바꿈 및 여백 최적화 ───
\XeTeXlinebreaklocale "ko"
\XeTeXlinebreakskip 0pt plus 3pt
\emergencystretch 5em
\tolerance=2000
\hyphenpenalty=50
\exhyphenpenalty=50
\doublehyphendemerits=10000
\finalhyphendemerits=5000
\setlength{\hfuzz}{2pt}

% ─── 행간 ───
\linespread{1.52}

% ─── 단락 ───
\setlength{\parindent}{1em}
\setlength{\parskip}{0pt}

% ─── 헤더/푸터 ───
\usepackage{fancyhdr}
\pagestyle{fancy}
\fancyhf{}
\fancyhead[LE]{{\small\sffamily\leftmark \hfill \thepage}}
\fancyhead[RO]{{\small\sffamily\thepage \hfill \rightmark}}
\renewcommand{\headrulewidth}{0.4pt}
\renewcommand{\footrulewidth}{0pt}

\fancypagestyle{plain}{
  \fancyhf{}
  \fancyfoot[C]{\small\thepage}
  \renewcommand{\headrulewidth}{0pt}
}

% ─── 제목 스타일 ───
\usepackage{titlesec}

\titleformat{\chapter}[hang]
  {\sffamily\bfseries\LARGE}
  {\thechapter}{12pt}{}
  [\vspace{2pt}{\titlerule[0.8pt]}]
\titlespacing*{\chapter}{0pt}{-10pt}{24pt}

\titleformat{\section}[hang]
  {\sffamily\bfseries\Large}
  {\thesection}{8pt}{}
\titlespacing*{\section}{0pt}{20pt}{8pt}

\titleformat{\subsection}[hang]
  {\sffamily\bfseries\normalsize}
  {\thesubsection}{6pt}{}
\titlespacing*{\subsection}{0pt}{14pt}{6pt}

% ─── 보충 자료 박스 ───
\usepackage[most]{tcolorbox}

% 핵심 요약 박스
\newtcolorbox{summarybox}{
  colback=white, colframe=black,
  fonttitle=\sffamily\bfseries,
  title={\small 핵심 요약},
  breakable, sharp corners,
  boxrule=0.5pt,
  left=8pt, right=8pt, top=6pt, bottom=6pt
}

% 예시 박스
\newtcolorbox{examplebox}[1][]{
  colback=white, colframe=black,
  fonttitle=\sffamily\bfseries,
  title={\small #1},
  breakable, sharp corners,
  boxrule=0.4pt,
  left=8pt, right=8pt, top=6pt, bottom=6pt
}

% 연습 문제 박스
\newtcolorbox{exercisebox}{
  colback=white, colframe=black,
  fonttitle=\sffamily\bfseries,
  title={\small 연습 문제},
  breakable, sharp corners,
  boxrule=0.5pt,
  left=8pt, right=8pt, top=6pt, bottom=6pt
}

% 용어 정리 박스
\newtcolorbox{glossarybox}{
  colback=white, colframe=black,
  fonttitle=\sffamily\bfseries,
  title={\small 용어 정리},
  breakable, sharp corners,
  boxrule=0.4pt,
  left=8pt, right=8pt, top=4pt, bottom=4pt
}

% 정의/정리 박스
\newtcolorbox{definitionbox}[1][]{
  colback=white, colframe=black,
  fonttitle=\sffamily\bfseries,
  title=#1,
  breakable, sharp corners,
  boxrule=0.5pt,
  left=8pt, right=8pt, top=6pt, bottom=6pt
}

% 검증 리포트 박스
\newtcolorbox{verificationbox}{
  colback=white, colframe=black!60,
  fonttitle=\sffamily\bfseries,
  title={\small 번역 검증},
  breakable, sharp corners,
  boxrule=0.3pt,
  left=8pt, right=8pt, top=4pt, bottom=4pt
}

% 딥리서치 교육 콘텐츠 박스
\newtcolorbox{researchbox}[1][]{
  colback=blue!3!white, colframe=blue!40!black,
  fonttitle=\sffamily\bfseries,
  title={#1},
  breakable, sharp corners,
  boxrule=0.5pt,
  left=8pt, right=8pt, top=6pt, bottom=6pt
}

% ─── 교차참조 ───
\usepackage{hyperref}
\hypersetup{
  colorlinks=false,
  pdfborder={0 0 0},
  bookmarksnumbered=true,
}

% ─── 목차 설정 ───
\setcounter{tocdepth}{2}

% ─── 열거 ───
\usepackage{enumitem}
\begin{document}


\begin{titlepage}
\centering
\vspace*{3cm}
{\sffamily\bfseries\Huge 프린스턴 수학 안내서\par}
\vspace{1cm}
{\sffamily\Large The Princeton Companion to Mathematics\par}
\vspace{2cm}
{\large 한국어 번역본\par}
\vspace{1cm}
{\normalsize Timothy Gowers 편저\par}
\vfill
{\small 번역: AI 보조 번역 시스템\par}
\end{titlepage}

\tableofcontents
\newpage

\part{제 III 부: 소개}


% ═══ Section 3.5: Higher Dimensions and Several Variables ═══
\section{고차원과 여러 변수}
\label{sec:3-5}

58 
1. 머리말

$a + b\sqrt{-5}$ (단 $a$, $b$ 는 정수이다.) 형태의 복소수 중 6 은 다음과 같이 표현된다: 2 × 3 또는 $(1+\sqrt{-5})\times(1-\sqrt{-5})$. 여기서 숫자들은 2, 3, $1+\sqrt{-5}$, 그리고 $1-\sqrt{-5}$ 가 더 이상 분해되지 않으므로 해당 환에서는 6 이 두 가지 다른 소인수분해를 지닌다고 할 수 있다. 그러나 "숫자" 개념을 확장하여 [III.81 §2] 에서 언급한 방식처럼 자연스러운 방법으로 접근할 수 있으며, 위에 제시된 방식에서 산술 기본정리를 증명하는 데 도움이 될 것이다. 우선 각각의 수 γ 에 대응되는 모든 배수들의 집합 δγ 을 생각하겠습니다. 이러한 δ는 환에 속한다고 볼 수 있습니다. 다시 말해서, (γ) 라는 집합은 아래와 같은 닫힘 성질을 만족합니다: α 와 β가 (γ) 에 속하고 δ 와 ε 는 환의 임의의 두 원소라면 δα + ϵβ 또한 (γ) 에 속하게 된다. 환 내부의 부분집합이 앞서 언급된 폐쇄적 특징을 가져야 하는 경우 '아이디얼'이라고 불립니다. 아이디얼이 어떤 수 γ 의 형태(γ) 로 주어진다면 ‘주요 아이디얼’이라고 하며, 다만 주요 아이디얼일 필요 없다는 점도 중요하다. 본원적인 링 요소를 모두 개념화하여 세트로 보아봅니다. 결과적으로 추가 및 곱셈과 같이 자연스러운 개념들이 존재하며, 아이디얼 I 를 "프라이임"으로 정의하는 것은 단지 J × K =I 에서 하나가 “단위”인 경우입니다. 확장된 집합에서는 고유 인수 분해 법칙이 적용되기 때문에 이러한 개념들은 우리에게 기본링에 있어 ‘고유 인수 분해법 실패’정도를 평가할 매우 유익한 방법을 제공한다. 더 자세한 내용은 대칭학 [IV.1 §7] 을 참조하십시오.

3.5
높은 차원과 여러 변수

우리는 이미 한 가지 변수의 방정식뿐 아니라 여러 변수의 시스템 방정식을 살펴보는 것처럼 다항식 방정식 연구가 복잡 해지는 것을 알았습니다. 마찬가지로 편미분방정식 [I.3 §5.4], 여러 변수를 포함하고 있는 미분방정식으로 여겨질 때 일반적인 미분방정식보다 분석하기 어렵다는 점 또한 확인했습니다. 바꿔 말하면, 오직 하나의 변수만 사용하는 미분 방정식에서 발생하는 두 사례 모두 몇 년 동안 수학적으로 가장 중요하게 생각되어 온 문제와 결과들을 만들어낸 과정 - 단일변수부터 여러 변수까지의 일반화라는 주제입니다. 예를 들어 세 실수 변수 x , y , z 를 가진 방정식이 있다면 그림 1 에서 나타난 원형 배열에 대한 고려도 도움이 될 것입니다. (x, y, z) 는 세 개의 수들의 집합이 아니라 그 자체로서 하나의 대상이다고 볼 수 있습니다. 더욱이 이 객체는 자연스러운 해석을 가지고 있으며: 삼차원 공간 속 한 점을 나타냅니다. 이러한 기하학적 해석은 매우 중요하며, 변수가 하나에서 여러 개로 확장되는 정리와 정식들에 대한 이해를 돕는데 큰 역할을 합니다. 우리가 연산대수 부분을 하나의 변수에서 여러 변수로 일반화한다면, 우리가 하는 일을 단일 차원 설정에서 고차원 설정으로 일반화하는 것으로 생각해볼 수도 있습니다. 이 아이디어는 알게브라와 기하학 사이 많은 연결들을 만들어내며 두 분야의 기술들이 서로에게 효과적으로 적용될 수 있도록 하여줍니다.

1 반지름인 원형을 평면 안에 가장 치밀하게 채우려 한다면 어떻게 할까요? 이 질문은 ‘포 packing 문제’라고 불리는 유명한 예시입니다. 해답은 잘 알려져 있으며 당연히 다음과 같습니다: 원들의 중심점이 사각격자 형태를 이루도록 배열해야 합니다(그림 1 참조). 세 가지 차원에서는 비슷한 결과가 성립하지만 증명하기에는 매우 어렵고 최근까지 케플러 추측(Kepler conjecture)라는 명성 있는 미해결 문제였으며 여러 수학자가 해설했지만 실질적인 검증되지 않았던 것입니다. Thomas Hales 가 도움받았다고 주장했습니다. 그의 논리적 근거는 복잡하고 오랜 시간 동안 계산되었는데 그중 하나에서 확인된 것도 아직 체크되기 위한 과정으로 진행되었습니다.

구체의 포 Packing 에 대한 질문은 임의의 차원수 에서 제시될 수 있습니다. 다만, 차원이 커짐에 따라 더욱 어려워집니다. 실제로는 구체적으로 설명하는 것이 거의 불가능합니다.

가장 좋은 구조를 찾는 방법은 일종의 "brute force" 검색일 수 있겠죠? 하지만 가장 가능성 있는 복잡한 구조를 찾으려면 불가능하며 심지어 모든 가능성을 유한하게 줄일 수 있다고 가정하더라도 확인할 만큼 너무 많습니다. 
 혹시 이렇게 문제가 해결하기에 지나치게 어렵다면 완전히 포기하지 않아야 합니다. 대신 관련된 하지만 접근하기 용이한 질문들을 제시하는 것이 더 생산적인 반응 방식입니다. 예를 들어 최적화 된 패킹을 발견하려고 하는 대신 단순히 얼마만큼 치밀한 패킹을 찾을 수 있는지를 살펴볼 수 있습니다. 다음과 같은 주장 개요에서 n 차원(n) 에서 상당히 높은 효율성을 제공하는 패킹 스케치입니다: 하나부터 시작하여 원자로 구성되어 있으면서 다른 것들 중 적용되지 않는 한까지 계속해서 추가되는 것을 선택하면 되므로 우리 집합 속 각 공들의 확대 후 $R^n$ 전체를 커버합니다.(여기에 사용될 때는, 'R' 는 실수의 부분집합임을 의미하며, 'Λ', 'M' 등 기호들은 그 문맥 내에서 정의됨.)

2 배 인자가 증명되며 모든 가능성들이 유일하게 나타나는 것은 매우 중요합니다. 위 논리에서는 구체적으로 설명할 필요가 없습니다.






% ═══ Section 5: The Western Revival of Interest ═══
\subsection{서부의 관심 재고양}
\label{sec:5}

2. 기하학 (87 페이지)

그( ) 성립 한다면 평행포스트 ulata 또한 따르는 것을 보였다. 하지만 그의 논증은 잘못되었다. 당시 수학자들이 사용 가능한 기술만으로 이러한 논쟁 중 어떤 것이 잘못되었는지를 파악하는 것은 매우 어렵습니다. 이슬람 세계의 수학자들은 오느른 세기에 서부 후계자가 따라올 만큼 정교한 학문적 지식을 가지고 있었습니다. 그러나 안타깝게도 그들의 저서는 바티칸 도서관에서 발견된 유일한 작품(1594 년 출판)를 제외하고는 오랫동안 서구에게 알려져 있지 않았으며 많은 해 동안에는 알-투스리의 저작으로 오해받았는데, 아들로 추정됩니다.

3. 서양의 수학적 관심 재흥

평행추측에 대한 서구 학자들의 재흥은 16 세기 명신과 마우롤리코가 이끄는 그리스수학 번역의 두 번째 파도와 함께 시작되었습니다. 인쇄술이 발달하면서 전파되기도 했습니다. 여러 개의 오래 된 도서관에서 중요한 책들이 발견되어 결국 유클리드 원론 의 새로운 저작이 만들어졌습니다. 많은 작품들이 평행선 문제에 대해 언급했으며, 해리는 "유클리드에게 남아있는 얼룩"이라고 간결하게 표현했습니다. 예를 들어 요셉 클라비우스 같은 강력한 신부인 그는 《엘레멘츠》 를 1574 년 에 편집하고 수정하며 평행 선을 '거리가 같다' 라고 정의하려고 시도했습니다.

육각형 공간과 유클 리디안 기하학 공간의 직접적인 동일시는 코페르니쿠스 천문학의 받아들임과 고정성 별 구체라는 것의 제거 후 16~17 세기에 점차적으로 나타났습니다. 뉴턴 [VI.14] 은 그의 수학적 사상 에서 중력설을 제시하며 유클리디언 공간 속에 단단히 자리를 찾은 이것을 공식화하여 확립했습니다. 뉴토 니 아 물리법칙이 인식받기 위해 노력해야 하더라도 뉴톤 우주관은 매끄러운 길로 진전되어 18 세기를 지배하는 무반대론이었습니다. 이는 그러한 식별에서 결과가 더욱 중요해졌다는 주장을 할 수 있습니다.왜냐하면 유클리드 원본만으로 얻어진 예측되지 않거나 반직관적인 결론은 현재, 가능하게 바쁜공간에 대한 반직관적인 사실인지 때문입니다.

3. 평행 추측 재검토 (초창)

1663 년 영국의 수학자 존 월리스 는 전례를 가질 만큼 미묘한 관점으로 평행추측 을 취했다 . 그는 라틴 문명 도서관 에 있는 알-타우 시 작품 의 비밀스럽게 번역된 내용에 대해 해리가 읽고 있었으며 또한 증명하려는 시도를 보였습니다. 독특하게 월리스 역시 자신의 논쟁의 오류 부분을 파악했는데 그의 주장은 코어와 함께 평행 추측과 같은 형태의 그림들이 일치하지 않는 것임을 나타내었다 고 언급했습니다.

50 년 후에는 가장 강경하고 완벽한 모든 평행 추측 방위자인 이탈리아 신부 제롬 로 스카케리를 따랐다. 곧 죽음 연도로 인하여 '유클 리디 구원'이라는 이름의 책 한 권을 출판하였다. 이는 고전적 추론의 소중한 걸작이며 삼항 분할에서 시작한다. 평행 추측이 명확히 되면, 삼각형 각합은 두 직선보다 적거나 같거나 크기가 될 것이다. 사커리는 어떤 것이 하나의 삼각형 에서 발생하면 다른 곳에도 모두 동일하다 는 것을 보여주었습니다., 따라서 유클리드 원본 과 호환되는 세 가지 기하학적인 가능성이 있는 것으로 보입니다. 첫 번째 경우 (L)에서는 모든 삼각형에 두 개의 직각인 합계보다 작은 각자가 있습니다. 두 번째는(E)모든 삼각형 에 두 개의 직각 의 합등합니다.(G)세번째 , 모든 삼각형 은 두 개의 직각 이상의 각자를 가집니다 . E 사례는 물론이고 유 클리 디안 기하학이다. 그는 그게 가장 단순하게만 존재하는 케이스라고 주장했고 이것 때문에 L 과 G 와 같은 나머지 둘 다 스스로 파괴된다는 증명하기 위해 노력했습니다.

그는 G 라는 상황을 성공적으로 해결하고 다음에는 '평행축제' 자체의 진실을 방해 하는 것은 바라보았던 L 상태에서 집중했다 고 말한 것입니다

case L 상태가 참이면 서로 만나지 않는 두 선분이 공통 수직선 한 줄밖에 없으며, 양쪽 끝단 에서 발산한다 라고 설명할 수 있다 는 사항들을 확립하여 최종적으로 그의 어려움을 헤쳐내기 위한 방법으로 무극점 행동에 대한 부적절함 문구에 의존하면서도 시도 실패하였다

자크 리 사케리는 점차 오류에 가까워졌으나, 그의 연구는 결국 무산되었다. 그러나 스위츠랜드 출신 수학자 요한 라 mber 트가 그의 업적을 이어받았으며, 사케리를 따르면서도 평행추측을 증명하는 데 집중하기보다는 다른 방향으로 나섰다. 라 mbert 는 특정 케이스에서의 삼각형의 넓이의 관계 등 여러 중요한 결과들을 명확히 제시했다. 예컨대, 'L' 케이스에서 삼각형의 넓이는 내각들의 차와 삼각형의 전체적인 외곽선 길이의 비율로 나타난다고 주장했지만, 그 근거를 잘못 설정하여 논증은 부족하였다. 그는 또한 'L' 케이스에서 서로 유사한 삼각형들은 일치한다는 것을 발견했습니다.






% ═══ Section 6: The Shift of Focus around 1800 ═══
\subsection{약 1800년경의 초점 이동

Let me know if you have more mathematical texts to translate! I'm ready for your next challenge. 😊}
\label{sec:6}

삼각함수 표의 천문 분야 적용은 실제로 효과적이지 못했으며 각 삼각형 크기에 맞춰 새로운 표가 필요하게 되었습니다. 특히 모든 예측 값이 작으면서 $\theta >60^\circ$ 인 경우는 주어진 각으로 정사각형 형태를 만들 수 있었습니다. 철학자들이 "절대적인" 길이의 치(길이는)를 가지도록 불렀던 개념처럼 ($\theta = 30^\circ$인 등변삼각형 변의 길이와 같음), Leibniz[VI.15] 추종자인 Wolff 는 불가능하다라고 말했습니다. 실제로 직관적으로 어려운 일이며 일반적으로 길이는 상황에 따라 비례하여 정의된다는 것을 알고 있습니다 (파리를 기준으로 한 미터 스틱이나 지구둘레 또는 그것과 같은 것들의 일정한 부분). 하지만 Lambert 에 따르면, “그러한 논쟁들은 사랑과 증오에서 나왔으며, 수학자가 할일 없을 것입니다.”

환경 전개 주위에서 약간 다른 관점에서 보겠습니다. 유클리드 원론 현대판 출판 이후 서양은 평행 가설에 대한 열망이 줄기 시작했는데 그 사업을 다시 돌아보며 감소하기 때문입니다. 프랑스 대혁명 후 Legendre [VI.24] 는 École Polytechnique 입학하려는 학생들을 위해 교과서 쓰기를 시작하며 기본적 지오메트리를 유클리드 원론처럼 엄격하게 되돌리는 것이 목표였습니다. 하지만 직관적인 무작정 추측하는 책들도 바꾸어야 할 필요성만큼이나 실질적으로 요구되는 정확함까지 제공할 수 없습니다. 레장드르 역시 결국 자신의 시도가 실패했다는 사실을 인지하고 특히 다른 사람들이 해온것처럼 그는 평행 가설에 충분하고 적절한 방어선을 제시하지 못했습니다. 레장드르의 Éléments de Géométrie 는 여러 판자와 함께 발전하면서 때때로 새로운 도전 과제가 등장하곤 되었습니다. 일부 시도들은 좋게 말해서 보기 어렵고 가장 좋은 것조차 매우 설득력있다고 볼 수 있습니다.

레장드르의 작업은 전통적인 영향으로 이끌려 그의 생각에는 여전히 평행 가설이 반드시 참일 것이라는 확신이 있었는데, 약 1800 년경부터 모든 학자가 동해 의견을 가지기 시작했던 점과 달랐음

위 내용 설명에서는 위험하면서도 명백한 변화를 보여주며 게우스 [VI.26] 에 편지를 보낸 마르부르크 대학교 법학 교수인 F. K. Schweikart 라는 개구리가 간단한 메모 속에 담아냈습니다.Schweikart 는 "별 지오메트리"라고 불렀으며 삼각형 각합이 두 직각보다 작았고 정사각형 형태가 특정하게 되면서 직각등변삼각형 높이는 스웨이크 하트가 “상수”라 부른 값으로 제한되었다는 주요 결과들을 한 페이지 안에서 기술했습니다. 스와익하트는 새로운 기호미법이 우주의 진짜 도덕적 원리를 나타낼지도 모릅니다라고까지 말하며 게우즈에게 답했다. 그는 결과를 받아들였고 상수값 하나만 알면 초등학교 수준의 전체적인 지오매터기를 할 수 있다고 말했습니다. 어느 누군가들은 스워윅 Hart 가 단순히 Lambert 's 사후 저작을 읽은 것 이상의 일을 해내었다고 비난할 수 있지만 (직선 등 변화하는 이론 외에는 새롭게 나온 것은 없다) 가장 중요한 점은 그의 생각 방식입니다: 새로운 지오 매 트릭이 참일 가능성, 그저 순전히 수학적 호기심과 같은 것이 아닌 것입니다 유클리드'원칙에 의해 더 이상 속박받지 않았습니다

가우스 자신이 어떤 생각을 가지고 있었는지는 알 수 없습니다. 하지만 그의 놀라운 수학적 본능을 보았던 학자들이, 분석 결과 그가 처음 유클리드 기하학에서 이탈하여 비율릭 lidean 지오매터리를 발견했다고 주장하기 시작했죠. 다만, 증빙 자료 부족 때문에 분명하게 결론 내리는 것은 매우 어렵습니다. 가우스 초기 연구에는 평행선 정의 등 유클리드 원칙 관련 내용이 포함되어 있습니다. 후반기에는 이미 알고 있는 것들을 바탕으로 편지를 쓰며 자신의 관점을 드러내기도 했습니다. 현재 남아있는 서류들은 게우스가 얼마나 많은지도 모르겠으며 진짜 지식이나 비유 클 리디안 지 오메트리가 사용될 가능성을 제시하는 것이 아닙니다. 대신, 1810 년대에 모든 전통적인 시도에도 불구하고 유클리드 기하학 심부에서부터 평행 가설을 추출하지 못하며 앞으로 나올 가능성은 없다고 여겼다는 것을 볼 수 있습니다. 그는 다른 공간 형태를 더 신중히 고려해보기 시작한 것입니다.

게우스에게 위상은 논리적 산술과 같은 의미였지만 동시에 역동적으로 실험 과학인 메커니즘과 연결되었습니다. 가장 단순하면서 명확한 설명 방식으로 말씀드릴 때면 두 차원 이상까지 계속해서 확립되었던 우주공간입니다: 비율릭 lidean 지오매터리를 사용할 수 있다는 것은 의심의 여지가 없습니다; 물론 하나만 존재합니다 : 상단 언급된 L 사례와 같습니다. 친구 베셀과 올버즈와 함께 천문학자로서 증거들을 통해 자신의 결정을 강화했습니다.






% ═══ Section 6.2: Estimates ═══
\section{추정치}
\label{sec:6-2}

반복 관계와 생성함수

1 서론

반복 관계를 처리하는 효과적인 방식 중 하나가 바로 생성함수입니다.[IV.18 §§2.4, 3]. 집합론적 및 대수적 조합학에서도 비슷하게 설명됩니다.[IV.18 §3]. 위의 논점을 이해하기 위해 R 과 L 를 각각 [ ] 로 치환하면 연결성이 더 명확합니다 (즉 합법적인 구획이 항상 음이 아닌 산책 경로에 해당한다는 의미). 이러한 접근방식은 W(n) 값을 정확하고 효율적으로 계산하는 데 도움을 줄 수 있습니다. 수학에는 다양한 다른 정교한 방법들이 있지만 ' brute force' 기법 없이 기하학적으로 문제를 분석할 수 있는 또 다른 예시입니다 ([IV.18] 서문 참조 - 어떤 상황에서 문제들을 완전히 해결했다고 간주해야 하는지 고려해 보세요.).

(i) 평면을 잘라내는 n 개 선분으로 만들어지는 영역 r(n) 의 수 입니다. 두 직선이 평행하며 세 직선 이상이 공통 점 없는 경우라고 가정하십시오. 첫 네 값인 r(n) 은 2, 4, 7 그리고 11 입니다. 다음 공식을 증명하기가 어렵습니다:

r(n)=r(n−1)+n , 이는 다음과 같은 공식으로 귀납적 추론될 수 있습니다 :
r(n)=1/2($n^2$+n+2 ). 본 명제와 그의 증명은 더 높은 차원까지 일반화 가능합니다.

(ii) 주어진 양의 정수 n 에 대해 사각수들의 합으로 나타낼 수 있는 방식 s(n) 의 수입니다. 우리는 양의 음수 및 허용되는 다르게 표현된 것을 사용하지만 (예를 들어 1 + 2 + 3² + 2², 3² + 4² + 1² + 2², 1² + (−3)² + 4² + 2² 와 같이), 여기서 각 하나가 서로 다른 방법으로 30 를 사각수의 합으로 표현하는 것입니다).

s(n) 는 n 에 대한 모든 약수 중에서 4 의 배수가 아닌 것들에 대해 8 배라는 것이 보일 수 있었습니다. 예를 들면, 12 의 약수들은 1, 2, 3, 4, 6 그리고 12 이며 그중 1, 2, 3 과 6 은 4 의 배수가 아니므로 
s(12) = 8 * (1+2+3+6)=96 입니다. 또한 재정렬하고 부호 변환하여 얻은 식도 가능합니다.

(iii) 공간 내 선분들이 L₁, L₂, L₃ , and L₄ 네 개의 직선과 만나는 경우의 수는 일반적인 위치[IV.18 §5] 에서 이루어집니다.(즉 두개 이상 평행하거나 교차하지 않는 특별한 성질을 가지고 있는 바람직합니다.) 어떤 세 개의 직선에는 R³ 속 일부인 초구표면이 포함되는데 이러한 초구표면은 고유하며 하나만 존재한다. 우리는 L₁ ,L₂ 와 L₃ 을 위한 표면 S 를 취해봅시다 .

S 는 문제 해결할 때 유용하게 사용될 수 있는 여러 독특한 성질들을 가진다는 것을 알아야 합니다. 가장 중요한 점은 각 실수 t 에 대해 연속적으로 변하는 한 집합으로 구성된 모든 라인들 M(t), 그들의 합체로서 전체를 형성함, 그리고 모두 L₁, L₂ 및 L₃ 의 위에 있다는 것입니다. 하지만 또 다른 연속적이고 겹치지 않으며 각 L(t)와 정확히 한 지점에서 만나도록 하는 마음가짐도 있습니다:

M(s). 따라서 매번 M(s) 은 L₁, L₂, L₃ 과 같은 방식으로 접촉하고 있으며, 결국 L₁, L₂, L₃ 에서 연결되어 있으면 반드시 M(s) 중 하나여야 한다는 사실입니다.

(iv) 양의 정수 n 을 양의 정수들의 합으로 나타내는 방법 p(n) 개수 입니다. 여기서 n = 6 이라면 이 값이 11 개이며 (예제 : 
6=1+1+1+1+1+1=2+1+1+1+1=2+2+1+1=2+2+2=3+1+1+1=3+2+1=3+3=4+1+1=4+2=5+1=6 ) 함수 p(n) 는 분할 함수라고 불립니다. hardy [VI.73] 와 ramanujan [VI.82] 에 의해 주어진 놀랍고 정밀한 공식 α(n) 가 존재하며 p(n) 을 근사하여 모든 경우에 있어 p(n)은 α(n)로 가장 가까운 정수임

6.2 추정치

위에서 보여준 예시 (ii)를 살펴보니, 일반화 가능할까라는 의문이 자연스럽게 생깁니다. $t(n)$ 이 나타내는 수열에 대한 공식은 존재할까요? 예를 들어 '10 개의 여섯번째 거듭제곱으로' 합하여 정수로 표현할 방법인 경우일지도 모르겠습니다. 하지만 이 질문에 대한 명확한 해답은 없다고 생각되며 그런 종류의 공식은 발견되지 않았습니다.

하지만 포장 문제와 마찬가지로 정확한 해결책이 명백하지 않더라도 매우 매력적인 것은 계산하기 용이하도록 함수 $f(n)$ 를 정의해 보고 $f(n)$ 가 항상 약간씩 같으면 하는 것입니다. 즉,$t(n)$ 와 비슷한 형태를 가지도록 합니다. 만약 그것조차 어렵다면 L 및 U 라고 불리는 두 개 간단하게 계산되는 함수들을 찾아서 모든 n 에 대해 $L(n) \le t(n) \le U(n)$ 을 만들어 볼 수 있습니다. 우리가 성공한다면 L 는 t 로부터 아래 경계이고, U 는 위경계라고 부릅니다. 누구나 정확히 세기에는 알려진 값을 사용하고 있는데 적분값이나 최소/최대치를 이용하는 것이 유용합니다






% ═══ Section 6.3: Averages ═══
\section{평균

Let me know if you have more mathematical texts to translate! I'm ready for your next challenge. 😊}
\label{sec:6-3}

10 번째 수학 연구 일반 목표

\[ \text{Goal} \]

$x_{i}$ = $a^{2}_{j}$ + $\sum\limits_{k=1}^{m}{b_{ik}}$

\[\frac{\partial f}{\partial x}\] + $\int_{-\infty}^{\infty} g(t) dt$

$= \lim_{\epsilon \to 0}\left(\frac{f(x+\epsilon)-f(x)}{\epsilon}\right)$

π(n) 는 n 미만의 소수 개수를 나타내며, 이 문제에 대한 근사적 해법 찾기가 중요한 과제입니다. 작은 n 에 대해서는 π(n) 을 정확하게 구할 수 있지만 (예시 : 20 ≤ π(20)=8), 큰 n 에서는 이렇게 하지 못합니다. brute-force 알고리즘으로 모든 수들을 검토하는 방법도 생각해 볼 만하지만 시간 복잡성 때문에 효율적으로 사용하기 어렵습니다. 따라서 분석적 수론 분야에서 더욱 유용하고 통찰력 있는 접근 방식을 필요로 합니다. Hadamard 와 de la Vallée Poussin 은 19 세기에 증명된 명성있는 소정리를 통해 π(n) 가 log n / n 로 근사한다고 보여주었습니다. 여기서 π(n) 와 log n/n 의 비율은 무한대로 갈 때 1 에 수렴함을 의미하며, 이것은 매우 매혹적인 결과 중 하나입니다.

소수들의 "포화" 현상과 관련하여 임의 선택된 n 인근의 정수 중 하나가 소수일 확률은 약 1 /log n 입니다. 그러므로 π(n) 는 다음과 같이 로그 적분 공식으로 계산될 것으로 예상됩니다:
∫\_0\^{}n (dt)/log t .

Riemann 가설 [V.26] 에 따르면 li(n) 과 π(n) 사이의 오차는 √n log n 만큼밖에 차이를 가지지 않으며, 두 함수 간의 관계는 매우 중요합니다. S√n log n 값은 π(n)보다 훨씬 작으니, Riemann 가설이 맞다면 우리에게 매우 좋은 근사값임을 알려줍니다.

평면 위에서 길이가 n 인 자기 회피 경로란 점들 ((a₀, b₀), (a₁, b₁), ..., (an, bn)) 을 순열하는 것입니다. aᵢ와 bᵢ 값들은 모두 정수이며 각 i 에 대해서는 (ai, bi) 를 얻기 위해 (ai-1, bi-1) 에서 수직 또는 수평 방향의 단위길이 한 번 이동합니다. 다시 말해서 ai = ai - 1 이고 bi = bi - 1 ± 1 또는 ai = ai - 1 ± 1 그리고 bi = bi - 1 입니다. 두 개 이상의 점들이 같지 않습니다.

첫 번째 두 조건으로부터 시퀀스 형태에서 평면상의 양자적 거리 'length' 가 있는 것처럼 나타나고 세번째 조건은 해당 도보 절대 동일한 위치를 재방문하지 않는다고 명시되어 "self-avoiding" 라는 용어가 사용됩니다. S(n) 을 시작점이 (0, 0) 인 길이가 n 의 self-avoiding walk 들의 총개수라고 하겠습니다. 공식을 알려주지는 않으며 그러한 공식 존재 가능성도 매우 적지만 함수 S(n) 이 어떻게 성장하는지를 이해하기 위한 많은 정보들을 확실히 알고 있습니다. 예를 들어, S(n)/n 은 일정 값 c 로 발산한다는 것을 증명할 만큼 간단하게 할 수 있다. C 값은 아직 정확하게 계량되지 않았으나 컴퓨터 분석 결과로 확인된 바 있으며 2.62 와 2.68 사이임을 보여줍니다.

C(t) 를 원형 반경 t 에 대해 기원점 주변에 포함되는 직교좌표계 점들의 개수라 한다. 다시 말해, a² + b² ≤ t² 와 같은 조건을 만족하는 모든 정수쌍 (a,b) 가 C(t) 입니다.

반지름 t 에서 있는 원의 영역은 πt², 그리고 그 안으로 단위 사각격들을 채워넣으면 각 사각격들은 유리 수 중심값을 가지고 있다. 따라서 큰 t 에서는 πt² 라는 것이 C(t) 의 근사치라는 것은 명백하며 두 값은 비슷합니다. 하지만 이러한 추론이 얼마나 정밀해야 하는지는 불분명하다.

좀 더 구체적으로 질문하기 위해 우리가 |C(t)-πt²| 로 설정하여 ε(t) 을 사용하겠습니다. 여기에서 ε(t) 는 C(t) 값과 πt² 간의 오차입니다. Hardy and Landau 가 1915 년에 증명했듯이 c√t 이상일 수밖에 없으며 일정 상수 c >0 이고 그것은 예상대로 적절하게 작동한다. 최근 Huxley 가 2003 년에 입증된 가장 좋은 위계 제약인 A * $t^{131/208}$(log t)\^{}2.26 (어떤 상수 A 에 대해 ) 입니다.
평균값

전까지 우리의 추정 및 근사 논의는 주어진 종류의 수학적 개체를 세기 위한 목표로 한 문제에 국한되었습니다. 하지만, 단순히 특정 집합 크기를 알고 싶을 때만 해당하는 것이 아닙니다. 특정 집합이 주어졌을 때 사람들은 또한 집합 내 객체들이 어떻게 분포되어 있는지 파악하고자 합니다. 많은 질문은 각 객체와 관련된 일련의 숫자 매개변수들의 평균 값을 물어보는 형태로 나타납니다. 다음 두 가지 예시를 살펴봅시다.

(i) 자율 회피 경로 길이 n 의 시작점과 끝점 사이의 평균 거리는 얼마입니까? 본 경우에는 (0 , 0 ) 에서 출발하여 길이 n 인 자율 회피 경로가 대상이며 기호화되는 매개 변수는 끝-끝 간격입니다. 놀랍도록 이것은 매우 어려운 문제이고 거의 알려져 있지 않습니다. n 는 명백하게 ...






% ═══ Section 6.4: Extremal Problems ═══
\section{극값 문제}
\label{sec:6-4}

수학 교과서 번역 수정본 (부분)

64. 서론

S(n)의 상한선을 찾으려 하지만 일반적인 자기 회피 경로는 시작점에서 n 보다 훨씬 작은 거리를 이동하는 등 많은 구불구불함과 곡률을 가지므로 예상됩니다. 그러나 S(n)에 대해 n 보다 현저히 좋은 상한선이 아직 알려져 있지 않습니다. 반대로, 일반적인 자기 회피 경로의 끝대단거리가 평범한 경로보다 더 크기를 가질 것으로 기대합니다. 이것은 S(n)이 √n 보다 상당히 큰 값임을 시사하지만, 그것이 커다는 사실조차 증명되지 않았습니다. 여전히 논해야 할 부분들이 있으며 세 번째 절에서는 추가적으로 논의할 것입니다.

(ii) 대수적 함수들을 생각해보겠습니다. 어떤 매우 큰 무작위 정수 'm' 을 선택하고 ω(m) = p1 p2 ... * pk 라면 k 는 m 의 고유 소인자 개수입니다. 주어진 조건하에서는 위 식이 성립합니다: loglog m ~ ln√k . 따라서 우리에게 다음 정보가 제공되어 있습니다 :

모든 가능성 있는 경우 중 가장 높게 나타나는 것은 약간 이상하게 느껴집니다.

문제는 C ∈ C 에 속하는 어떤 곡선이 가장 작은 평균 높이를 가지는지 입니다. 이 문제는 변분법이라는 기법을 사용하여 해결할 수 있습니다. 매우 간단히 말하면 집합 C 와 함수 h 를 가정하며, h 는 각 곡선 C ∈ C 에 그 평균 높이를 매핑합니다. 우리의 목표는 h 를 최소화하고, 그것을 달성하기 위한 자연스러운 방법은 일종의 도함수를 정의하고 이 도함수가 0 인 곡선 C 에서 찾아보는 것입니다. "도함수"라는 단어가 여기서 의미하는 바는 곡선 위로 이동하며 높이의 변화율이 아닙니다. 오히려 전체 곡선의 평균 높이가 곡선의 미세한 변동에 따라 어떻게 변하는지를 나타내는 ( 선형 ) 방식입니다. R 에서 정의된 함수의 고정점을 찾는 것보다 이러한 종류의 도함수를 이용해서 최솟값을 찾는 것은 더 복잡합니다. 이유는 C 가 무한차원 세트이고 따라서 R 보다 훨씬 복잡하다는 점 때문입니다. 하지만 접근방식은 작용하게 만들고 가장 적은 평균 높이를 가지도록 하는 곡선(catenary 라 불리는데 사슬을 뜻하는 라틴어에서 유래) 은 알려져 있습니다. 그렇습니다! 또 다른 정확하게 해결된 최적화 문제 입니다.

변분법의 일반적인 문제에서는 특정 양이 최소 또는 최대 되도록 한 개 이상의 곡선이나 표면, 아니거나 더 일반적인 형태의 함수를 찾으려 합니다. 만약 최저치나 최댓값이 존재한다면(무한 차원 집합과 작업할 때 항상 자동으로 발생하지 않기 때문에 매우 중요하고 관심 있는 질문일 수 있음), 그것을 달성하는 대상물은 [I.3 §5.4] 에 Euler–Lagrange 방정식 으로 알려진 편미분 방정식 시스템을 충족시킵니다. 변형 방법에 대한 자세히 보기 위해서는 III.94 (또한 optimization and lagrange multipliers [III.64]) 를 참조하십시오.

(iii) 1 부터 n 까지 선택 가능한 숫자는 세개가 등차수열로 이루어지지 않는 경우?
n = 9 일때 답은 5 입니다. 첫째 , 1, 2, 4, 8, 9 는 중복되지 않습니다 . 지금 우리는 여섯 가지 작동하는 숫자가 될까요 ? 만약 우리가 하나의 숫자를 5 라고 하면 그러므로 다음 두 개인 4 또는 6 을 제외해야 하고 그렇게 되면 4, 5, 6 의 진행 과제가 생깁니다. 마찬가지로, 우리는 3 과 7 중 한쪽이나 2 와 8 중 한쪽이거나 1 과 9 중 한쪽도 배제해야 합니다. 그리고 네 개의 수들을 모두 남겨두었기 때문에 불가능합니다. 따라서 우리는 5 를 포함하지 않아야 한다고 결론을 내릴 수 있습니다. 우리는 각각 1, 2, 3 와 7, 8, 9 에서 한 번씩 삭제 해야 합니다. 만약 5 를 없애려한다면 4 와 6 을 추가할 필요가 있지만 나머지는 둘 다 사용될 수 없습니다. 또한 최소한 1, 4, 7 에서 하나 이상은 제거하도록 강요되며 적어도 네 개의 수를 제거하게 된다는 것을 알았습니다 .

iv) 위 논리를 통해 n = 9 일 때 삼항 등차수열 존재하는 경우를 확인했지만, n 이 매우 큰 경우에는 너무 많은 조합들을 고려해야 하기 어렵습니다. 따라서 대신 n 의 크기에 대해 상하 경계를 찾는 방법을 활용하며, 가장 큰 정수 집합인 1 부터 n 까지로 구성된 세개 길이의 등차수열이 없는 것은 분명하지 않습니다. 그래서 작은 세트를 만들거나 특정 크기의 임의적인 세트 내에 반드시 포함되는 삼항등차수열 증명과 같은 접근법을 사용합니다. 현재 발견된 최선의 경계치들은 상당히 다릅니다. Behrend 는 log n 의 루트곱 (e\^{}√log n )크기의 set 을 생성했고 Jean Bourgain 은 C⋅$n^{-log(log n}$/log n) 에서 각각 중복되지 않는 arithmetic progression 를 가지는 set 에 해당합니다.( 만약 두 값 사이 차이가 모호하다면 예를 들어 n=10 \^{}100 일때 e √log n 는 약 4 백만이고 C⋅$n^{-log(log n}$/log n) 는 약 6.5 라 생각하면 이해하기 용이할 것입니다.)

iv) 계량학적 연구에서 많은 문제들이 최소화하는 형태이며, 이러한 작업들을 빠르게 수행해야 하는 경우에는 더욱 중요해집니다. 다음은 간단한 예제입니다: 자릿수가 n 인 두 개의 정수를 곱하려면 얼마나 걸립니까? 심지어 " 단계"라는 것이 명확하지 않더라도 전통적인 장렬 연산법으로 적어도 n²개 이상의 단계가 필요합니다. 첫 번째 숫자의 모든 위치와 두번째 숫자의 모든 위치에 대해 제품을 하나씩 만들기 때문입니다. 그것은 필연적으로 요구되는 것처럼 보일 수 있지만 실상에서는 매우 지능적인 방법들로 시간 감축 효과를 얻을 수 있습니다. 가장 빠른 알고리즘 중 하나는 [III.26] 에 나오는 FFT (Fast Fourier Transform ) 를 사용해서 단계 수를 n² 에서 C ⋅ log(n) ⋅ log(log n) 로 줄이는데 성공했습니다. 로그 값이 원래값보다 작으므로 C⋅형태라고 생각할 수 있으며 , 이러한 경계치들은 선형적이고 문제에서 최선의 결과이며, 각각의 쌍만 읽는데에도 약간 더 많은 계단이 필요하기 때문에

또 다른 유사한 질문은 행렬 곱셈에 대한 빠르게 실행 가능한 알고리즘들이 있는지 여부 입니다. 일반적으로 크기를 가진 두 개의 nxn 행렬을 멀티플라이하려면 적어도 O(N³) 의 연산 비용이 발생합니다...






% ═══ Section 7: Bolyai and Lobachevskii ═══
\subsection{볼야이와 로바체프스키}
\label{sec:7}

II.2. 기하학

、。。 Disquisitiones Generales circa Superficies Curvas (1827) 。、。、[III.78]、、 1840 。、L 。、。。 L 、 L 。、。。

비유클리드 기하학 발견자: 볼야이 및 로바셰브스키

비유클리드 기하학의 발견은 각각 헝가리 출신의 볼야이 [VI.34] 와 러시아 출신의 로바쉐브스키 [VI.31]에게 돌아갑니다. 두 사람 모두 독립적으로 매우 유사한 설명을 제시했습니다. 특히, 이차원과 삼차원에서 정의된 도형 체계를 서술하며, 오직 유클리드 기하와 다르게 공간의 기하도 가능하게 하는 동등한 주장력을 가지고 있었습니다. 로바쉐브스키는 처음으로 1829 년에 불분명한 러시아 학회지에 게재했으며, 다시 프랑스어로 1837 년, 독일어로 1840 년, 그리고 한 번 더 프랑스어로 1855 년에 재출판했습니다. 볼야이는 아버지 작품인 두 권짜리 지오메트릭 논문 부록에 자신의 저서를 실었는데 시기는 1831 년입니다.

두 명의 과정:
새로운 직선 개념:

주어진 점 P 와 선 m 이 있다면, P 를 통하고 m 과 만나는 선들과 만나지 않는 선들이 존재한다고 보았습니다. 두 집합 세트를 분할하는 것은 하나는 P 의 오른쪽 다른 하나는 P 의 왼쪽까지 거리가 임의적으로 가까워질 수 있는 P 를 통과하는 두 줄 것입니다. 이 상황은 그림 3 에서 볼 수 있습니다: 질문되는 두 선 n' 와 n'' 는 각각 P 를 통해 m 과 마찰하지 않고 매우 근접하게 접근하며 그 사이에는 P 를 포함한 모든 선들은 m 에 대해 나뉘게 된다.

P

n '
m
n ''
병렬사이의 각 (angle of parallelism) : 로바쉐브스키와 볼야이는 새로운 기하학에서는 평행 한 직선들을 위한 병렬사이의 각을 정의했습니다. 그림 위의 선들이 구부러 보이는 것은 평면적인 유클리드 페이지에 그려내기 위해서 비틀어 표현해야 하기 때문이며, 만약 기하학 자체가 유클리드적이라면 한 줄로 만들 수 있으며 양쪽으로 무한히 이끌릴 수 있다는 것을 알아야 한다. 새롭게 사용된 말투에도 불구하고 P 에서 m 까지 법선을 내리는 것처럼 이야기하는 의미는 여전히 타당합니다. m 을 통하여 P 를 거쳐 있는 두 개의 직각인평행은 법선과 같은 각도를 가지며, 이것을 병렬사이의 각라고 부릅니다.만약 각이 직각이라면 해당 기하학은 유클리드적입니다.그러나 직각보다 작으면 새로운 기하학 가능성이 생긴다는 것입니다. 결과적으로 각 크기는 P 에서 m 까지의 법선 길이와 관련되어 있습니다. Bolyai 나 Lobachevskii 는 병렬사이의 각이 직각보다 작다는 가정을 받고 있음에도 불구하고 모순 없는지 증명하려는 노력에는 전혀 힘쓰지 않았습니다. 대신 단지 간단하게 가정을 하고 법선의 길이에 따라 각을 결정하기 위한 많은 노력을 들였습니다.
곡선: 그들은 모두 주어진 선과 평행 한 집합의 선들 중 하나 위의 점으로부터 시작해서 모든 줄들을 수직으로 교차하는 곡선이 존재한다는 것을 보여주었습니다.(Fig4).
유클리드 기하학에서는 이렇게 정의된 곡선은 특정한 지점을 통과하며 여러 개의 평행한 선들의 오른쪽 각도를 가지며 일치되는 직선인데 (Figure5),

그림 5 에서와 같이 유클리드 기하학에서는 한 점 Q 를 통과하는 모든 직선의 집합을 고려하고 다른 점 P 를 선택하면 P 를 지나는 수직인 선에 대한 커브가 존재합니다. 그림 6 처럼 중심점 Q 이고 P 를 지나도록 하는 원입니다.

볼라이와 로바체프스키가 정의한 커브는 이러한 두 가지 유클리드 구성물 모두의 특성을 가집니다:평행사변형에 수직이며 구부러져 있으며 일정하지 않습니다. 볼라이는 그런 커브를 L -커브라고 불렀고, 로바쉐브스키는 더 도움되는 이름으로 호로시클(horocycle) 라고 명명했으며, 이 이름은 현재까지 사용되고 있습니다.

복잡한 논쟁들은 양쪽 사람들을 삼차원 기하 세계로 이끌었다. 여기에서 로바체프스키의 주장이 볼라이보다 분명하게 드러났으며, 두 사람 모두 게우스를 크게 앞섰다고 할 수 있습니다. 호로 시클 형태를 한 평행선 주위로 회전하면 3 차원 공간에서 동일한 방향을 가집니다. 그리고 이렇게 생성된 받침대 모양 표면 (볼라이에게 F-표면, 로바체프스키에게 호로 스피어) 입니다. 둘 다 놀라운 사실을 보여주었습니다.

호로 스피어를 통과하는 평면은 원이나 호로 시클로 그것을 잘랐고, 호로 시클 변으로 구성된 호로 스피어 위에 삼각형을 그려 각도 합이 두 개의 직각임을 알 수 있었습니다. 다시 말해, 호로 스피어가 포함하고 있는 공간이 L 케이스 세 가지 치수 버전이며 확연히 유클리드적이지 않더라도 호로 스피어에 집중하면 제약되는 지오메트리가 이차원적인 유클리드 기하학입니다!

볼라이와 로바체프스키 또한 자신의 세 차원공간 내에 구를 그릴 수 있다는 것을 알았습니다(이는 본질적으로 새로운 것이 아니었지만). 하지만 구기 역설 정리는 병렬 가정 없이 독립한다는 것을 증명했습니다. 로바체프스키는 그의 평행선으로 관용성있는 건축물을 사용하여 구상에서 한삼각형이 단순히 평평함에도 불구하고 같은 방식으로 결합되며 즉, 호로 스피어 상의 삼각형까지 동일하며 모든 공식들이 적용될 가능성은 분명합니다.

세부 사항들을 확인했고 로바체프스키 및 볼라이 모두 (비슷한 방법으로) 호로 시클 위의 삼각형들은 하이퍼보릭 트리고노미터 리 공식들로 표현된다는 것을 보여주었습니다.

spherical geometry 의 공식들은 문제가 되는 구의 반경과 관련되어 있습니다. 마찬가지로 하이퍼보릭 트리고노미티의 공식도 특정 실수 매개변수에 따라 달라집니다. 다만 이 매개 변수에는 유사하게 명확한 기하학적 해석이 없습니다. 예외를 제외하면 공식들은 여러 가지 확신할 수 있는 성질을 지니고 있습니다. 특히 삼각형의 크기가 매우 작으면 일반적인 공간 형태와 가까운 관계입니다.

길이의 그리고 영역의 공식은 새로운 환경에서 개발되었으며; 각 도 조화 합량 차이는 두 직각보다 부족함 만큼 삼각형의 영역 비례한다는 것을 나타냅니다. 즉, 호로 스피어 상의 삼각형들이 단순히 평면 내삼각형들의 일부분일 가능성이 높습니다. 로바체프스키에게 있어서 그러한 종류의 정돈되고 설득력있는 공식 자체가 새롭게 만들어진 기하학을 받아들임으로써 충분했다고 생각했습니다. 그의 관점에서는 모든 기하학은 치밀하고 연결된 계산과 관련되어 있었는데, 그는 그것들을 표현하는 방식인 연속적으로 사용되는 공식을 통해 얻었습니다.

볼라이와 로바체프스키는 독창적인 세차원 기하학에 대한 설명을 제공하여 유클리드 기하학 또는 매개변수 값(그것도 결정할 수 있는) 에 따라 적용될 수 있다는 질문을 제시합니다.






% ═══ Section 8: Acceptance of Non-Euclidean Geometry ═══
\subsection{비유클리드 기하학의 수용}
\label{sec:8}

Ⅱ. 2. 기하학 91

보일라이는 이 문제를 해결하지 못했지만 로바체프스키는 별 운동 parallax 측정이 질문을 해결할 수 있다고 주장했습니다. 그는 실패했다; 그런 종류의 실험은 매우 민감한 작업이며, 결과적으로 그의 아이디어들은 무관심과 적대성 속에서 살펴졌으며 두 사람 모두 자신의 발견들이 가져올 성공을 모른 채 세상을 떠났습니다. 보일라이와 아버지는 가우스에게 그 연구물을 제출했는데, 가우세는 1832 년 "자신을 비찬하는 것은 자기를 추켜세울 것"임을 주장하며 작품을 “칭송하기 어렵다”라고 평가했습니다. 게다가 간단한 증명도 제공하여 야노쉬 보일라이 초기 결과 중 하나입니다. 또한 오래된 친구의 아들보다 우선순위를 차지하게 된 것을 알게 되면서 기뻐한다고 말했습니다. 야노슈 보일라이는 분노해서 다시 출판하지 않으면서 논문 형태로 공개함으로써 가우스 앞에서 선점권이 있음을 입증하려 하였으나 이렇게 함으로써 가능성을 포기합니다. 이상하게도 가우스가 사실 상 소년 화려인 업적에 대한 구체적인 내용을 미리 알았다는 증거는 없습니다. 더 설득력 있는 해석은 1830 년까지 가우스가 물질공간이 비유클 리드 지오메트리를 통해 설명될 수 있다는 확신에 달했다는 것입니다. 그리고 두차원 비 유 클리디안 지오매트리는 위상 삼각법(그것만큼이나 상세한 계산) 을 사용하면서 처리하는 방법을 잘 알고 있습니다 (그러나 그의 손에서는 자세한 정보가 남아있지는 않습니다).

하지만 세 개의 치수 이론은 처음에는 보일라이와 로바 체프스키에게 알렸는데, 마치 그것들을 읽었던 후야 아닌 것처럼 간주되어서는 안됩니다.

로 바셰 프 스 키 는 보일 라이 만 못해 나쁜 결과를 얻었습니다. 그는 1829 년 초기 출판물이 오스트로그라 드 스 기 등보다 명확히 인정받지 않았습니다. 또한 St 페터부르크에서 활동하고 있었다면 로바쉐브스크이는 주요 도시 카잔으로부터 거리가 있다는 점도 고려해야 합니다. Journal für die reine und angewandte Mathematik 에서 그 이야기 (“ 크렐레 저널” ) 은 러시아 논문에서 가져온 결과들이 증명되었다는 것을 언급했기에 심하게 어렵게 된 사실입니다. 그의 책자는 평점과 비슷하지 않으며 더 이상 지적 장애인에 의해서 한 번 리뷰되었습니다. 게다가 가우스에게 전달하여 우수하다고 판단했습니다. 그리고 로바체프스키 를 Göttingen 과학 학회 회원 자격을 부여하는 데 도움을 줄 수 있습니다. 하지만 가우세의 열광은 이것만큼 끝났는데, 그 뒤에는 추가적인 지원을 받았던 것은 아닙니다. 중요 발견에 대한 매우 불쾌한 반응은 여러 가지 관점에서 분석할 필요성을 제시합니다. 두 사람 모두를 신뢰하였던 공통된 직선 정의가 당연히 미흡했다라는 것이 분명하며 작업이 바로 그것 때문에 비판받지 않았습니다. 오직 무례함 속에서 버리고 간 것처럼 보였다; 자신감 있는 태도와 같은 잘못으로 인해 시간낭비라고 생각될 만큼 명백하고 나쁜 실수라서 그러한 오류를 찾아내려면 너무 많은 노력이 들 것입니다. 따라서 저자가 할 일은 작품 주저들을 조롱하기거나 단순히 언급 없이 잊어버리는 것을 선택해야 합니다. 유클리드 지오매트리가 당시 대부분의 사람들의 마음속에 얼마나 강하게 남겨져 있었는지를 알기 위해서는 충분하다고 말씀 드립니다. 예를 들어 코페르니쿠스주의 또는 갈릴레오의 발견조차 전문가들에게 더 좋은 환영을 받았습니다.

8
비유클리드 기하학의 받아들임

1855 년 갈리스 사망 당시 발견된 미발표 자료 중 하나에 그는 보야이의 증거 및 로바체프스키 작업과 함께 비유클리드 기하학의 타당성을 인정하는 편지를 포함했습니다. 이들이 점차 출판되자 사람들은 보야이와 로바체프스키 작품을 더욱 호 Referanser 적으로 바라보기 시작했습니다. 우연히 가우스는 또한 고팅겐 대학교 학생 리만에게 만났습니다. 두 명 사이 직접적인 접촉량이 적었더라도, 물건을 결단적으로 앞진 시킬 수 있는 재능을 지닌 존재였던 것입니다. [VI.49] 참조로 하십시오. 리만은 1854 년 Habilitation 논문 방위 과제에서 독일 수학자들의 교수 자격증과 같은 의미를 가지며 전통대로 삼 가지 주제 제안 후 가우스(그 검사관)가 가장 예상치 못했던 것을 선택하게 합니다." 위치에 대한 추론" . 그의 저서는 오직 사후인 1867 년에 발표되었습니다. 그것은 단순하게 새로운 형태의 지오메트리가 아니었습니다.

리만은 기하학이 자신이 말하는 다양체[I.3 §§6.9, 6.10] 연구라고 제시했습니다. 이들은 점들로 구성된 "공간"이며 거리는 유클리드 공간처럼 작은 스케일에 보이는데 큰 스케일에서는 상당히 다른 특징들을 나타낼 수 있다고 설명했다. 그는 계산법으로 여러 방법으로 진행될 수 있으며 임계 차원을 가진 모든 종류의 다변량 위에서 가능하며 심지어 무한차원적인 다변량까지 고려할 준비가 된 상태라는 것을 강조했습니다.
구글스와 동맹하여 리만의 기하학적 중요 부분은 내재성 만족되는 성질과 관련되어 있었다. 더 크게 주입된 공간 속에 의존하지 않는다는 의미입니다. 구체적으로 두 개의 점 x 와 y 사이의 거리를 정의되며:

\[d(x,y)\] = 가장 짧은 경로 길이의 합인 \[\text{min} \left\{ L(\gamma) : \gamma\subset M,\gamma (a)=x, \gamma (b)=y \right\}\].

이러한 곡선을 지오디식(geodesics) 라고 합니다.(예를 들어 구에서 지오데시크는 대원 아치이다.)

두차원 다양체 조차도 서로 다른 본질적 휨들을 가지고 있었습니다 - 사실 하나의 두 차원 다양체가 여러 곳마다 다른 휨을 가진 경우도 있다.- 리만의 정의는 각 치수별 무한히 많은 진정으로 고유한 기하학을 만들어냈습니다. 더욱이 이들 기하학들은 그 안에 있는 유클리드 공간에 대한 참조 없이는 최상으로 정의되었기 때문에 유클리드 기하학의 우위는 한 번과 영구적으로 끝났습니다.
 그의 논문 제목 속 "추측"이라는 단어처럼 리만은 유클리드에게 필요했던 종류의 전제에 관심이 적었지 않았습니다. 또 그는 유클리드 기하학과 비-유클리드 기하학 사이의 반대에도 별로 관심이 많지는 않았습니다. 그는 레젠트르 노력에도 불구하고 기하학 심장부에 드러나는 어둠에 대해서 자신의 글 시작 부분에서 간단하게 언급하며 마침내 그는 일련된 삼각형들의 변곡률이 일정인 두차원 다양체 위의 세 가지 다른 기하학적 형태를 생각했다고 말했습니다. 하나가 구면 기하학이고, 다른 하나가 유클리드 기하학이며, 셋째는 다시 달랐으며 각 경우 모든 삼각형의 각도 합을 알 수 있도록 하였다 만약 누군가가 임의의 한 개의 삼각형의 각도합을 아느라 한다면 . 그러나 그는 볼라이와 로바셰프스키에 대한 언급조차 없었다며 공간의 기하학이 정확히 상수 휨으로 이루어진 삼차원 기하학일 때만 해당하는 지오메트리를 결정하려면 거대한 지역 내의 치밀한 조사가 필요하다고 주목하였다.

그 역시 가우스의 굽힘을 자유로운 차원의 공간까지 일반화하여 논했으며 [III.56] (즉, 거리의 정의)로써 고려할 수 있는 메트릭들을 표시했다 - 상수 휘게 되는 공간에서 가능합니다. 그의 글은 매우 일반적인 공식임에도 불구하고 보야이와 로바쉐프스키처럼 특정 실제 매개변수(곧 곡률)를 의존한다. 곡률이 음인 경우에는 자신의 거리 정의가 비- 유클리드 기하학적 설명을 제공하게 된다.

리만이 서거한 것은 1859 년이며, 당시 출판된 그의 박사 논문에는 독립적으로 동일한 개념에 도달한 이탈리아 수학자인 Eugenio Beltrami 의 연구 결과들이 포함되어 있었습니다. Beltrami 는 한 표면 S 를 다른 표면으로 변환하는 것이 가능한지 관심있었습니다. 예를 들어 구체적인 표면 S 에서 직선과 같은 지오데식이 평면처럼 나타나는지를 찾아보았습니다. 그리고 얻은 결론들은 항상 조건부 적용을 요구했습니다; 우주 공간 내에서는 절댓값이 일정해야 합니다. 북극반원에서부터 평면까지 연결되는 방식은 잘 알려진 방법 중 하나입니다. Beltrami 는 간단하게 수정하여 공식에서 현재 명확히 부여되고 있는 것을 사용하였는데 - 음성 값을 가진 공간 위쪽 디스크와 연관짓으며 중요도를 인지했습니다. 그의 매핑은 디스크 안쪽 메트릭스 정의하고 생성되었으며 기하적 형태가 비-유클리드 기하학 법칙을 따랐기 때문에 모순되지 않았습니다.

몇 년 전 미딩(Minding) 은 "거짓 스피어"라고 불리는 특별한 삼각형 변곡률이 일정하게 음수인 표면을 발견했던 독일 수학자로 유명합니다. 이것은 트랙시즈를 축에 대해 회전 시켜 나팔모양으로 만들었습니다. 거기에 있어서는 유클리드 평면 기하학 영역보다 자연적으로 보이고 대안적인 존재로 여겨졌으며 결과물 또한 더 적합하지 않게 느껴질 수 있습니다.

비슷한 과제들은 리우빌 [VI.39] 에 의해 다시 찾아냈지만 코달찌 (Codazzi ) 는 그 출처에서 알았던 것들을 통해 역삼항법으로 설명된다는 사실만 증명할 수 있었고 아무도 비-유클리드 기하학 연결 부분까지 볼 수 없었습니다 - Beltrami 는 자신의 디스크 그림이 무한히 큰 상황의 지오메트리를 나타내며 로바셰프스키가 사용하는 것처럼 정확하다고 생각하며 확인했습니다. 그는 자신 작업 중 하나와 관련하여 공간 내부 구조화되어 있는 방식으로 해석하고 있다: "구"라는 단어나 개념은 정확해야 한다라고 말한다. 논문 작성 후에도 베르트라미에게 계속해서 관심사를 가지도록 하였다 : “기본적”인 형태에 대한 명성과 신뢰감 때문에 불필요하게 유행하기보다는 진정함 자체보다 중요하다면 어떻겠니?

1871 년 클라이네(Klein) 가 주목했는데 이미 영국 수학자 케일리가 투영적인 기하학 [I.3 §6.7] 에 유클리드 미터 적용 방법을 만들었다는 것을 알고 있습니다. Berlin 에서 학업 생활하면서 Klein 은 Cayley 의 아이디어를 일반화하고 Beltrami 's 비-유클리드 기하학 을 프로젝션 기하학 특수 사례로 보여주는 것을 발견했다. 그의 아이디어는 위저스트래 스 (Weierstrass)[VI.44], Berlin 의 선두 과학자가 반대하며 거짓말이 측도가 아닌, 따라서 그것은 측량적 용어들을 생성할 수 없다고 주장했지만 그는 고집스럽게 이끌었습니다






% ═══ Section 8.2: Numerical Evidence ═══
\section{수치적 증거}
\label{sec:8-2}

1.4 
수학 연구의 일반적인 목표

69

그러나 거의 대부분이 이러한 사실 때문에 효율적인 알고리즘은 존재하지 않는다고 믿는다. 다시 말해 "P=NP"가 아니다라고 표현하는 것이다. 따라서 특정 문제에 대해 빠른 알고리즘이 없음을 증명하고 싶으면 이미 NP 완전한 것으로 알려진 어떤 문제만큼이나 적어도 그렇게 어렵다는 것을 보여주면 된다. 이는 철저한 증명은 아니지만 P = NP 라는 가설로 인하여 많은 수학자들이 확신하기 때문이다 (더 자세히 보기 위해 IV.20 계산 복잡성 참조).

몇 가지 추측보다 여러 개의 추측에 의존하는 연구 분야들도 있다. 마치 연구원들은 아름다운 수학적 경관을 발견했으며 이해할 만한 부분이 많지 않아도 그것들을 지도하려 하기 원한다는 느낌과 같다. 그리고 심지어 엄격한 증명을 찾는 관점에서라 할 때, 매우 좋은 연구 전략일 수 있다.

추측에는 단순히 무작위 기대와 같은 것이 아니라 중요하게 여겨질 경우 다양한 종류의 검증 과제를 거쳐왔어야 한다. 예를 들어, 이것이 이미 확인된 결과인가? 하나 또는 다른 사례 중 특정적인 것은 증명될 수 있는가? 진실이라고 해서 다른 문제를 해결하는 데 도움이 되나? 수치적 근거는 존재하나? 잘못되거나 반박되는 것으로 보였으면 어려워졌겠다는 명확하고 정밀한 주장은 하는가? 모든 테스트를 통과시키도록 만들고 성공하면 그것은 고립된 문구만이 아니며 많은 연결 관계로 가득 차 있게 된다. 따라서 증명 가능성이 높아지고 한 개의 추론에 대한 증명이 또 다른 것을 증명하기 위한 길잡이는 될 것이다. 좋은 추측에도 대응할 만한 역설적으로 놀랄 만큼 큰 정보 제공력도 가지는데: 추측이 여러 다른 표현들과 관련되어있을 시 
역설의 효과들이 전체 영역에서 확산됩니다.

추측문항이 많이 담긴 분야에는 대수적 소수 이론 [IV.1] 에 속한다. 구체적으로 로버트 라글랜드스(Robert Langlands) 의 작업, "Langlands 프로그램" 은 수학와 행렬표현 이론 (행렬표현 이론 IV.9 §6 참조 ) 을 연관짓는 일련의 추측들을 포함하며 두께들은 서로 일반화하여 하나하나 조율되고 설명하는데 도움을 준다. 예를 들어 안드류 와일즈[V.10]가 페르마 마지막 정리 증명 중심으로 사용되었던 심우라-타니야마-웨일 추측은 'langlands program' 부분만 작용합니다 .

‘ langlands program’ 는 좋은 추측 기준에 대한 테스트 결과를 최상점까지 통합하고 오랫동안 많은 수학자들의 연구 방향 설정에 활발하게 적극적인 역할을 해왔습니다. 유사한 특징을 가진 또 다른 분야는 거울대칭성 [IV.16] 으로 알려져 있다. 이는 계량기법과 관련된 물건들인 카랄비 - 요 만폴 
이들이 대수적 공간에서 발생한다면 스프링 강력함도 나타내며(알레브릭 지오메트리, IV.4) 그리고 양자장 이론에도 (양자 장 이론 IV.17 §2 참조), 그러한 것처럼 미세계의 다형체와 같은 개념들을 연결시키는데 효과적으로 쓰입니다.

어떤 미등식 문제들은 자신의 함수 형태 변환 분석 시 더욱 용이해질 때 있듯이 일정 문맥 내에서는 어떠한 연산 과제가 불가능하다고 여겨지는 경우에는 반대로 그것을 두 배로 만들거나 "거울" 상황으로 전환하면서 가능하다는 것을 보여주기도 합니다. 현재 정확히 명백하지 않지만 이런 방법은 누구도 예상하기 못했던 복잡하고 놀라운 공식에 도움을 주었으며 중대한 부분만큼이나 철저하게 증명되었습니다.

Maxim Kontsevich 는 거울대칭성의 성공적인 결과를 설명하는 완벽한 추측 제안을 하였습니다 .

8.2 수치적 증거

골드바흐 추측[V.27]에 따르면 4 이상의 모든 짝수는 두 소수의 합으로 표현될 수 있다고 한다. 오늘날 우리가 가지고 있는 수학적인 도구로 이를 증명할 가능성은 매우 작다고 생각된다. 리만 가설과 같은 명제들을 받아들이더라도 그렇게 할 것이다. 하지만 거의 확실히 사실이라고 여겨진다.

골드바흐 추측을 지지하는 근거에는 두 가지 중요한 요인이 있습니다. 첫 번째는 이미 관찰된 경향입니다.: 프임수들은 "랜덤하게 분포"한다면 골드바흐 추측이 참일 것으로 예상됩니다. 그 이유는 큰 짝수 n 을 나타내는 방법(a + b = n)들이 많기 때문이며, 충분한 개수의 프임수 존재 시 a 와 b 중 적어도 하나가 프림수라는 기대감이 높아지는 것입니다.

하지만 위 논리에서 어떤 값의 n (너무 크지 않음) 에 대해 불행했던 상황이 발생하고, a 가 프림수 일 때마다 n - a 가 합성수였을 가능성을 열어둔 채 남았다는 우려가 생깁니다. 바로 여기에 수치적 증거가 들어옵니다. 현재까지 10\^{}14 까지만 해서 모든 짝수를 두 소수의 합으로 표현할 수 있었다고 확인되었습니다. 그리고 더 큰 n 은 단순히 “운 좋게” 반례가 되는 것은 매우 비 probable 합니다.

이는 다소 모호해 보여도 설득력을 강화하는 방식이 있습니다. 만약






% ═══ Section 8.3: “Illegal” Calculations ═══
\section{불법 계산}
\label{sec:8-3}

70

I. 서론

소수 분포 개념을 정밀하게 이해한다면 골드바흐 추측의 강력한 버전을 세우고 더 나아가 모든 짝수를 두 소수 합으로 표현할 가능성 및 그 비율 근사치까지 제시할 수 있습니다. 예를 들어, a 와 n−a 가 모두 소수라면 어느 하나도 3 의 배수(자신이 3 과 같으면 제외)일 수 없습니다. 만약 n 이 3 의 배수라면, 단순히 'a' 는 3 의 배수만 아니라는 것을 의미하지 않으며, 특정 형태인 ‘3m +1’ 을 가지는 경우 ‘a’ 역시 ‘3k+1’형태로 나타날 수 없음 (그렇기에 ‘n − a’ 가 3 의 배수가 되버리는 것입니다). 따라서 일종의 관점에서 볼 때, n 이 두 소수의 합인 경우 3 의 배수 확률은 두배 높습니다. 이러한 사실들을 고려하여 n 를 두 소수의 합으로 표현하는 가능성을 "추론적으로" 계산해볼 수 있으며 결과적으로 모든 짝수 n 에 대해 여러 개의 해당표현들이 존재한다는 것이 드러납니다. 놀랍게도 우리가 내린 추측값들은 실제적인 증거와 매우 잘 맞아줍니다: 다시 말해서 작지만 컴퓨터를 통해 확인할 수 있는 값들의 n 에 대한 예언입니다. 이는 골드바흐 추측 자체뿐 아니라 그 근원적 원리를 신뢰하게 하는 강력한 증거라고 할 수 있습니다.

이는 일반 현상을 보여주는데요, 추측에서 유래하는 정확한 예측일수록 후속된 수치적 증거에 의해 검증될 때 얼마나 인상적이고 중요하다는 것을 알 수 있죠. 물론, 이것이 단순히 수학 문제만 아닌 과학 전반에게 적용되는 사실임을 기억해야 합니다

8.3 “불법” 계산

6.3 절에서 “n 단계 자기 회피 도보의 평균 끝대끝 거리는 대략 알려져 있지 않다” 라고 언급했다. 이는 물리학자들에게 반박받힐 주장이다. 오히려 ‘평범한’ n 단계 자기회피 도보의 끝대끝 거리가 약 $n^{3/4}$정도라는 것을 이야길 것이다. 처음에는 불 일치처럼 보이겠지만 실제로는 수학적으로 완전히 밝혀진 것은 많지는 않았기에 비교적인 차이가 발생한다는 점 때문입니다. 물리학자들은 사용하면 올바른 결과를 제공할 가능성 있는 다소 부드러운 방법들을 모아놓았습니다. 그들의 접근 방식은 어떤 분야에서는 수학자가 증명하기 어렵게 앞서 나간 명제들을 확립해냈습니다. 이는 수학적 추측으로 간주될 때 많은 것이 우수하며 (앞선 설명에 따르면) 심층적이고 미리 예상되지 못하고 광범위하게 진실되어 여겨질 만큼 매력적인 특징을 가지므로, 수학자에게 큰 관심사 입니다. 또 다른 중요한 이유는 철저한 기반 마련 노력이 순수 수학에서 상당한 발전을 가져오기도 하기 때문에 합니다.

물리학자들이 하는 '비 정밀 계산' 개념 이해를 위해 Pierre-Gilles de Gennes 의 유명 논증(혹시 해석라고 표현하는 게 좋지 않나요?) 에 대한 단순화된 설명을 들어보세요. 통계역학에는 Ising 모델 및 Potts 모델처럼 가능성분포와 임계 현상 [IV.25] 을 다룬 것으로 알려진 n 벡터 모델이 있습니다. Z d 의 각 점마다 $R^n$ 에서 하나의 단위벡터를 놓아줍니다. 그러고 다음은 "에너지"가 인접하는 벡터 사이의 각도가 증가함에 따라 증가하도록 연결되는 단위 벡터들의 임의 구성 화로 나타납니다. De Gennes 는 자기 회피 도보 문제를 변형하여 $n = 0$인 경우 $n$-vector model 과 관련된 질문으로 간주될 수 있는 방법을 찾았습니다. 직관적으로, $R^0$ 속에서는 단위 벡터라는 것이 존재하지 않으므로 0-vector 문제는 명확한 의미를 가지지 않습니다. 하지만 De Gennes 는 여전히 $n$-vector model 과 연관된 매개변수들을 사용하며 ‘n’ 이 영으로 수렴하면 self-avoiding walks 와 연관된 매개변수들이 얻어짐을 보여 주었습니다. 그는 다른 $n$-vector model 매개변수들을 선택해서 끝대끝 거리 등 자기회피 도보에 대한 정보를 유추했습니다.

수학자들에게 불편스러울 만한 접근 방식이 있습니다. n-벡터 모델에서 $n=0$일 때 공식은 의미를 가질 수 없지만 한계값으로 생각해야 합니다. 하지만 $n$-vector 모델 에서 $n$ 은 분명히 양의 정수입니다. 그렇기에 어떻게 그것이 0 으로 발산한다고 말할 수 있나요? 일반적인 'n' 에 대해 'n'-vector 모델을 정의하는 방법은 없는가요? 물론, 누구든 찾아냈습니다. 또한 de Gennes 와 같은 학자들의 주장처럼 비슷한 종류의 많은 주장들이 수치적 증거와 놀라운 정확성을 가지며 예측합니다. 우리가 이해하지 못하더라도 이러한 근본 원인이 있어야 할 것입니다.

본 절에서는 무정밀 계량법들이 수학적으로 풍부하게 만들도록 하는 다양한 사례들을 보여줍니다. 이런 기술은 수학자가 알려진 영역보다 더 나간 곳까지 파악하여 관찰되지 않았던 현상에 대한 연구를 위한 새로운 길을 제시합니다. 고리 라는 점에서 명백하고 확실하며 논리가 되지 않는 결과들은 중요하다면 rigor (엄격함) 는 얼마만큼 필요합니까?





\end{document}