\documentclass[11pt, a4paper, openany]{book}

% ═══════════════════════════════════════════════════════════
%  파인만 물리학 강의 — 한국어 번역 샘플 페이지 디자인
%  Design: 대화체 강의 + 교양 과학서 + 주석·해설 + 비주얼
% ═══════════════════════════════════════════════════════════

% ─── 여백: 넉넉한 마진, 사이드노트 공간 확보 ───
\usepackage[
  top=28mm,
  bottom=25mm,
  inner=25mm,
  outer=30mm,
  marginparwidth=0mm,
  headheight=14pt,
  headsep=14pt
]{geometry}

% ─── 한글 ───
\usepackage{kotex}
\usepackage{fontspec}

% ─── 폰트: 본문은 세리프, UI요소는 산세리프 ───
\setmainfont{Noto Serif CJK KR}[
  UprightFont={Noto Serif CJK KR},
  BoldFont={Noto Serif CJK KR Bold},
  Ligatures=TeX,
]
\setsansfont{Noto Sans CJK KR}[
  UprightFont={Noto Sans CJK KR},
  BoldFont={Noto Sans CJK KR Bold},
  Ligatures=TeX,
]
\setmonofont{Noto Sans Mono CJK KR}[Scale=0.85]

% ─── 수식 ───
\usepackage{amsmath, amssymb, amsthm}

% ─── 그래픽 ───
\usepackage{graphicx}
\usepackage{tikz}
\usetikzlibrary{
  arrows.meta, positioning, shapes, calc,
  decorations.pathreplacing, decorations.markings,
  patterns, shadows, backgrounds
}

% ─── 색상 팔레트 ───
\usepackage{xcolor}
\definecolor{feynred}{HTML}{C0392B}
\definecolor{feynblue}{HTML}{2471A3}
\definecolor{feyndark}{HTML}{1B2631}
\definecolor{feynnote}{HTML}{F0E6D3}
\definecolor{feynlight}{HTML}{EBF5FB}
\definecolor{feyngreen}{HTML}{1E8449}
\definecolor{feyngray}{HTML}{5D6D7E}
\definecolor{feynwarm}{HTML}{FDF2E9}
\definecolor{feyndeep}{HTML}{154360}

% ─── 한글 줄바꿈 ───
\XeTeXlinebreaklocale "ko"
\XeTeXlinebreakskip 0pt plus 3pt
\emergencystretch 5em
\tolerance=2000
\hyphenpenalty=50
\setlength{\hfuzz}{2pt}

% ─── 행간: 읽기 편한 넉넉한 간격 ───
\linespread{1.65}

% ─── 단락 ───
\setlength{\parindent}{1.2em}
\setlength{\parskip}{3pt plus 1pt}

% ─── 박스 디자인 (tcolorbox) ───
\usepackage[most]{tcolorbox}

% ◆ 파인만의 말 — 인용/강조 블록
\newtcolorbox{feynmansays}[1][]{
  enhanced,
  colback=feynwarm,
  colframe=feynred!60!black,
  coltitle=white,
  fonttitle=\sffamily\bfseries\small,
  title={\raisebox{-1pt}{\textbf{!}}~파인만이 강조합니다},
  sharp corners,
  boxrule=0pt,
  leftrule=4pt,
  breakable,
  left=10pt, right=10pt, top=8pt, bottom=8pt,
  shadow={1pt}{-1pt}{0pt}{black!15},
  #1
}

% ◆ 역주 (번역자 주석)
\newtcolorbox{translatornote}[1][]{
  enhanced,
  colback=feynnote,
  colframe=feyngray!40,
  fonttitle=\sffamily\bfseries\small,
  title={\raisebox{-0.5pt}{\textsf{*}} 역주},
  sharp corners,
  boxrule=0pt,
  leftrule=3pt,
  breakable,
  left=10pt, right=10pt, top=6pt, bottom=6pt,
  fontupper=\small,
  #1
}

% ◆ 심층 해설 (Deep Research)
\newtcolorbox{deepresearch}[1][]{
  enhanced,
  colback=feynlight,
  colframe=feynblue!70,
  coltitle=white,
  fonttitle=\sffamily\bfseries\small,
  title={#1},
  attach boxed title to top left={yshift=-2mm, xshift=4mm},
  boxed title style={
    colback=feynblue!80,
    sharp corners,
    boxrule=0pt,
  },
  sharp corners,
  boxrule=0.6pt,
  breakable,
  left=10pt, right=10pt, top=10pt, bottom=8pt,
  shadow={1.5pt}{-1.5pt}{0pt}{feynblue!15},
}

% ◆ 핵심 개념 박스
\newtcolorbox{keyconcept}[1][]{
  enhanced,
  colback=white,
  colframe=feyndark,
  coltitle=white,
  fonttitle=\sffamily\bfseries,
  title={#1},
  attach boxed title to top center={yshift=-3mm},
  boxed title style={
    colback=feyndark,
    sharp corners,
    boxrule=0pt,
  },
  sharp corners,
  boxrule=1pt,
  breakable,
  left=12pt, right=12pt, top=12pt, bottom=10pt,
}

% ◆ 수식 하이라이트 박스
\newtcolorbox{mathbox}{
  enhanced,
  colback=feynwarm!50,
  colframe=feynred!40,
  sharp corners,
  boxrule=0.5pt,
  left=14pt, right=14pt, top=10pt, bottom=10pt,
  before skip=12pt,
  after skip=12pt,
}

% ◆ 다이어그램 캡션 박스
\newtcolorbox{diagrambox}[1][]{
  enhanced,
  colback=white,
  colframe=feyngray!50,
  fonttitle=\sffamily\small,
  title={#1},
  attach boxed title to top center={yshift=-2mm},
  boxed title style={
    colback=feyngray!20,
    colframe=feyngray!50,
    boxrule=0.4pt,
    sharp corners,
  },
  sharp corners,
  boxrule=0.4pt,
  breakable,
  left=8pt, right=8pt, top=10pt, bottom=8pt,
}

% ◆ 연습 문제 박스
\newtcolorbox{exercisebox}{
  enhanced,
  colback=feyngreen!5,
  colframe=feyngreen!60,
  fonttitle=\sffamily\bfseries\small,
  title={\textsf{?} 생각해 봅시다},
  sharp corners,
  boxrule=0pt,
  leftrule=4pt,
  breakable,
  left=10pt, right=10pt, top=8pt, bottom=8pt,
}

% ─── 헤더/푸터 ───
\usepackage{fancyhdr}
\pagestyle{fancy}
\fancyhf{}
\fancyhead[LE]{{\small\sffamily\color{feyngray} 파인만 물리학 강의 \hfill \thepage}}
\fancyhead[RO]{{\small\sffamily\color{feyngray} \thepage \hfill \rightmark}}
\renewcommand{\headrulewidth}{0.4pt}
\renewcommand{\headrule}{\color{feyngray!40}\hrule width\headwidth height\headrulewidth}
\renewcommand{\footrulewidth}{0pt}

\fancypagestyle{plain}{
  \fancyhf{}
  \fancyfoot[C]{\small\color{feyngray}\thepage}
  \renewcommand{\headrulewidth}{0pt}
}

% ─── 제목 스타일 ───
\usepackage{titlesec}

% 챕터: 큰 번호 + 제목 + 원제
\titleformat{\chapter}[display]
  {\normalfont}
  {\hfill{\fontsize{72}{72}\selectfont\sffamily\color{feynred!20}\thechapter}}
  {-20pt}
  {\sffamily\bfseries\Huge\color{feyndark}}
  [\vspace{2pt}{\color{feyngray!40}\titlerule[1pt]}]
\titlespacing*{\chapter}{0pt}{-30pt}{30pt}

\titleformat{\section}[hang]
  {\sffamily\bfseries\LARGE\color{feyndeep}}
  {\thesection}{10pt}{}
\titlespacing*{\section}{0pt}{24pt}{10pt}

\titleformat{\subsection}[hang]
  {\sffamily\bfseries\large\color{feyndark}}
  {\thesubsection}{8pt}{}
\titlespacing*{\subsection}{0pt}{16pt}{6pt}

% ─── 하이퍼링크 ───
\usepackage{hyperref}
\hypersetup{
  colorlinks=true,
  linkcolor=feynblue,
  urlcolor=feynblue!70,
  pdfborder={0 0 0},
  bookmarksnumbered=true,
  pdftitle={파인만 물리학 강의 — 한국어 번역},
  pdfauthor={AI 보조 번역 시스템},
}

% ─── 목차 ───
\setcounter{tocdepth}{2}

% ─── 열거 ───
\usepackage{enumitem}
\setlist[itemize]{leftmargin=1.5em, itemsep=2pt}
\setlist[enumerate]{leftmargin=1.5em, itemsep=2pt}

% ─── 드롭캡 (첫 글자 장식) ───
\usepackage{lettrine}

% ─── 에피그라프 ───
\usepackage{epigraph}
\setlength{\epigraphwidth}{0.75\textwidth}
\renewcommand{\epigraphflush}{center}
\renewcommand{\epigraphrule}{0pt}

% ─── 각주 스타일 ───
\usepackage[bottom,hang]{footmisc}
\setlength{\footnotemargin}{0.8em}

% ═══════════════════════════════════════════════════════════
\begin{document}

% ─────────────────────────────────────────
%  표지
% ─────────────────────────────────────────
\begin{titlepage}
\begin{tikzpicture}[remember picture, overlay]
  % 배경 그래디언트 효과
  \fill[feyndark] (current page.south west) rectangle (current page.north east);

  % 장식 원 (파인만 다이어그램 느낌)
  \foreach \x/\y/\r/\o in {3/20/2.5/8, 15/24/1.8/6, 8/5/3/5, 17/8/2/7, 12/15/1.5/4} {
    \draw[feynred!\o 0, line width=0.8pt] (\x, \y) circle (\r);
  }

  % 파인만 다이어그램 모티프
  \begin{scope}[shift={(11,12)}, scale=1.5]
    \draw[white!40, line width=1.2pt, decorate, decoration={snake, amplitude=3pt, segment length=8pt}]
      (-2,0) -- (0,0);
    \draw[white!40, line width=1.2pt]
      (0,0) -- (1.5,1.2);
    \draw[white!40, line width=1.2pt]
      (0,0) -- (1.5,-1.2);
    \filldraw[white!50] (0,0) circle (3pt);
    \draw[white!40, line width=1.2pt, -Stealth]
      (1.5,1.2) -- (3,2);
    \draw[white!40, line width=1.2pt, decorate, decoration={snake, amplitude=2pt, segment length=6pt}]
      (1.5,-1.2) -- (3,-0.5);
  \end{scope}

  % 제목
  \node[anchor=west] at (2.5, 18) {
    \begin{minipage}{14cm}
      {\fontsize{14}{18}\selectfont\sffamily\color{feynred!80}\bfseries
        THE FEYNMAN LECTURES ON PHYSICS}
    \end{minipage}
  };

  \node[anchor=west] at (2.5, 15.5) {
    \begin{minipage}{14cm}
      {\fontsize{36}{42}\selectfont\sffamily\color{white}\bfseries
        파인만 물리학 강의}\\[8pt]
      {\fontsize{16}{20}\selectfont\sffamily\color{white!70}
        제1권: 역학, 복사, 열}
    \end{minipage}
  };

  % 저자
  \node[anchor=west] at (2.5, 8) {
    \begin{minipage}{14cm}
      {\Large\sffamily\color{white!80}
        Richard P. Feynman}\\[4pt]
      {\normalsize\sffamily\color{white!50}
        Robert B. Leighton \quad Matthew Sands}\\[16pt]
      {\small\sffamily\color{feynred!60}
        한국어 번역 · AI 보조 번역 시스템}
    \end{minipage}
  };

  % 하단
  \node[anchor=south] at (current page.south) [yshift=15mm] {
    {\small\sffamily\color{white!30}
      개인 학습용 · 비배포 · Sample Design v1.0}
  };
\end{tikzpicture}
\end{titlepage}

% ─────────────────────────────────────────
%  목차 (간략)
% ─────────────────────────────────────────
\tableofcontents
\clearpage

% ─────────────────────────────────────────
%  Chapter 1: 움직이는 원자
% ─────────────────────────────────────────
\chapter{움직이는 원자}
\label{ch:atoms-in-motion}

\vspace{-8pt}
{\large\sffamily\color{feyngray} Atoms in Motion}

\vspace{12pt}
\epigraph{\itshape ``If, in some cataclysm, all of scientific knowledge were to be
destroyed, and only one sentence passed on to the next generation of creatures,
what statement would contain the most information in the fewest words?''}{--- \textup{Richard P. Feynman}}

\vspace{8pt}

% ─── 1-1. 서론 ───
\section{서론}

\lettrine[lines=2, loversize=0.15, nindent=0.5em]{\color{feynred}\textsf{이}}{} 장에서는 물질의 원자적 가설에 대해 이야기해 보려 합니다. 아, ``가설''이란 표현은 좀 약하군요 --- 사실상 물질이 원자로 이루어져 있다는 것은 과학의 가장 핵심적인 사실이라 해도 과언이 아닙니다.

만약 어떤 대재앙이 일어나서 모든 과학적 지식이 파괴되고, 단 \textbf{한 문장}만 다음 세대의 생명체에게 전달할 수 있다면, 가장 적은 단어로 가장 많은 정보를 담을 수 있는 문장은 무엇일까요?

\begin{feynmansays}
저는 그것이 \textbf{원자 가설}이라고 생각합니다. 모든 물질은 원자로 이루어져 있다 --- 끊임없이 움직이며, 조금 떨어져 있을 때는 서로 끌어당기지만, 너무 가까이 밀어 넣으려 하면 밀어내는, 그런 작은 입자들로 말이죠.
\end{feynmansays}

이 한 문장 안에 세상에 대한 \textit{엄청난} 양의 정보가 담겨 있습니다. 물론 약간의 상상력과 사고력을 동원해야 하지만요.

\begin{translatornote}
파인만은 이 강의를 1961--1963년 Caltech 1, 2학년 학생들을 대상으로 진행했습니다. 그의 강의 스타일은 학생들에게 마치 동료 과학자에게 이야기하듯 설명하는 것이 특징입니다. 이 번역에서는 그 대화체 톤을 최대한 살리려 했습니다.
\end{translatornote}


% ─── 1-2. 물질은 원자로 이루어져 있다 ───
\section{물질은 원자로 이루어져 있다}

자, 그러면 원자가 어떻게 생겼는지 한번 상상해 봅시다. 원자의 반지름은 대략 $1$~$2 \times 10^{-8}$\,cm 정도입니다. $10^{-8}$\,cm를 보통 \textbf{옹스트롬}(\AA)이라고 부르는데, 그러니까 원자의 크기는 1--2\,\AA{} 정도인 거죠.

이것을 실감하기 위해 이런 비유를 들어볼게요.

\begin{keyconcept}[원자의 크기를 실감하기]
사과 하나를 지구만큼 크게 부풀린다고 상상해 보세요. 그러면 사과 안의 원자들은 대략 원래 사과만한 크기가 됩니다.

\begin{center}
{\large 사과 속 원자 : 사과 = 사과 : 지구}
\end{center}

\vspace{4pt}
이 비율이 바로 $10^{-8}$이 의미하는 것입니다.
\end{keyconcept}

\subsection{물의 확대}

이제 물 한 방울을 아주 가까이에서 들여다본다고 상상해 봅시다. 보통 배율로는 그냥 매끈한 물이 보이겠죠. 하지만 배율을 $2{,}000$배 정도 올리면 --- 이 정도면 ``아주 좋은'' 광학 현미경이에요 --- 그래도 여전히 매끈한 물입니다.

배율을 더 올려봅시다.

\begin{diagrambox}[그림 1-1. 물의 확대 ($10^9$배)]
\centering
\begin{tikzpicture}[scale=0.9]
  % 배경
  \fill[feynlight!50, rounded corners=3pt] (-0.5,-0.5) rectangle (11.5,7.5);

  % 물 분자들
  \foreach \x/\y/\rot in {
    1.5/6/10, 3.5/6.2/-15, 5.5/5.8/25, 7.5/6/-5, 9.5/6.3/20,
    0.8/4.2/30, 2.8/4/-10, 4.8/4.5/15, 6.8/4.2/-20, 8.8/4/5,
    1.5/2.2/-25, 3.5/2/35, 5.5/2.3/0, 7.5/2.5/-15, 9.5/2/10,
    2.2/0.5/20, 4.2/0.3/-30, 6.2/0.7/15, 8.2/0.5/-5
  } {
    \begin{scope}[shift={(\x,\y)}, rotate=\rot]
      % 산소 (큰 원)
      \filldraw[feynred!70, draw=feynred!90, line width=0.6pt]
        (0,0) circle (0.35);
      \node[font=\tiny\sffamily\bfseries, white] at (0,0) {O};

      % 수소 2개 (작은 원)
      \filldraw[feynblue!60, draw=feynblue!80, line width=0.5pt]
        (-0.38, 0.28) circle (0.22);
      \node[font=\tiny\sffamily, white] at (-0.38, 0.28) {\textbf{H}};

      \filldraw[feynblue!60, draw=feynblue!80, line width=0.5pt]
        (0.38, 0.28) circle (0.22);
      \node[font=\tiny\sffamily, white] at (0.38, 0.28) {\textbf{H}};
    \end{scope}
  }

  % 범례
  \node[anchor=south west, font=\small\sffamily] at (-0.3, -0.3) {
    \textcolor{feynred!80}{\textbullet}~산소\quad
    \textcolor{feynblue!70}{\textbullet}~수소
  };
\end{tikzpicture}
\end{diagrambox}

약 $10^9$배로 확대하면, 드디어 이런 모습이 보이기 시작합니다. 물의 원자들이 서로 달라붙어서 뭉쳐 다니는 모습이죠. 이 덩어리 하나하나가 바로 \textbf{물 분자}($\text{H}_2\text{O}$)입니다.

각 물 분자는 수소 원자 2개와 산소 원자 1개로 이루어져 있는데, 이 세 원자가 $105.5°$의 각도로 결합하고 있습니다.

\begin{mathbox}
\begin{equation}
  \text{H}_2\text{O} : \quad
  \underbrace{d(\text{O--H}) = 0.957\,\text{\AA}}_{\text{결합 길이}}
  \qquad
  \underbrace{\angle(\text{H--O--H}) = 104.5°}_{\text{결합 각도}}
\end{equation}
\end{mathbox}


\begin{deepresearch}{심층 해설: 물 분자의 구조}
\textbf{물 분자}는 산소 원자 1개와 수소 원자 2개가 공유 결합으로 연결된 구조입니다. 산소의 전기음성도가 수소보다 훨씬 크기 때문에, 전자가 산소 쪽으로 치우치면서 분자 전체가 \textbf{극성}을 띠게 됩니다.

이 극성 때문에 물 분자들은 서로 \textbf{수소 결합}(hydrogen bond)을 형성합니다. 수소 결합은 분자 내 공유 결합보다 약하지만, 물의 여러 특이한 성질 --- 높은 끓는점, 높은 비열, 얼음이 물보다 가벼운 현상 --- 을 설명하는 핵심 원리입니다.

\vspace{4pt}
{\scriptsize\color{feyngray} 출처: Wikipedia --- Water (molecule), Hydrogen bond}
\end{deepresearch}


% ─── 1-3. 원자의 운동 ───
\section{원자의 운동}

자, 이제 이 원자들이 어떻게 움직이는지 이야기해 봅시다. 원자들은 계속 꼬물꼬물 움직이고 있습니다. 사실 ``열''이라는 것이 바로 이 원자들의 무작위 운동에 다름 아닙니다.

\textbf{온도}가 올라가면 이 운동이 더 활발해집니다. 반대로 온도가 낮아지면 움직임이 줄어들겠죠. \textbf{절대영도}($0\,\text{K}$, 즉 $-273.15\,°\text{C}$)에서는 --- 글쎄, 양자역학 때문에 완전히 멈추지는 않지만 --- 최소한의 에너지만 갖게 됩니다.

\begin{translatornote}
파인만은 여기서 ``영점 에너지(zero-point energy)''를 암시하고 있습니다. 양자역학의 하이젠베르크 불확정성 원리에 의하면, 입자는 절대영도에서도 완전히 정지할 수 없습니다. 이 주제는 제3권(양자역학)에서 자세히 다룹니다.
\end{translatornote}

실온에서 공기 분자의 평균 속력이 얼마나 되는지 아세요?

\begin{mathbox}
\begin{equation}
  v_{\text{rms}} = \sqrt{\frac{3k_BT}{m}}
  \approx 500\,\text{m/s} \quad (\text{실온, 질소 분자})
\end{equation}

\vspace{4pt}
\begin{center}
{\small 이것은 음속($343\,\text{m/s}$)보다 빠릅니다!}
\end{center}
\end{mathbox}

놀랍지 않나요? 지금 여러분 주위의 공기 분자들이 총알보다 빠른 속도로 날아다니고 있는 겁니다. 다만 사방팔방으로 부딪치면서 날아다니기 때문에, 한 방향으로 ``net movement''는 거의 없는 것이죠.


% ─── 기체, 액체, 고체 다이어그램 ───
\begin{diagrambox}[그림 1-2. 물질의 세 가지 상태 --- 원자 수준에서 본 모습]
\centering
\begin{tikzpicture}[scale=0.85]
  % ─── 기체 ───
  \begin{scope}[shift={(0,0)}]
    \draw[feyngray!40, rounded corners=4pt] (-0.3,-0.3) rectangle (4.3,4.3);
    \node[font=\sffamily\bfseries, feynred] at (2, 4.7) {기체 (Gas)};

    \foreach \x/\y/\dx/\dy in {
      0.5/3.5/0.3/0.15, 3.2/3.8/0.1/-0.3,
      1.8/2.5/-0.2/0.25, 3.5/1.2/0.15/-0.1,
      0.8/0.8/-0.1/0.3, 2.5/0.5/0.25/0.1,
      3.8/2.8/-0.3/-0.15, 1.2/1.5/0.2/-0.2
    } {
      \filldraw[feynblue!60] (\x,\y) circle (0.15);
      \draw[-Stealth, feynred!70, line width=0.7pt] (\x,\y) -- ++(\dx*2.5,\dy*2.5);
    }
    \node[font=\scriptsize\sffamily, feyngray, text width=3.5cm, align=center] at (2,-0.7) {
      분자 간 거리 멀고\\자유롭게 운동
    };
  \end{scope}

  % ─── 액체 ───
  \begin{scope}[shift={(5.5,0)}]
    \draw[feyngray!40, rounded corners=4pt] (-0.3,-0.3) rectangle (4.3,4.3);
    \node[font=\sffamily\bfseries, feynblue] at (2, 4.7) {액체 (Liquid)};

    \foreach \x/\y in {
      0.6/3.5, 1.4/3.6, 2.2/3.4, 3.0/3.5, 3.6/3.3,
      0.5/2.7, 1.3/2.6, 2.1/2.8, 2.9/2.7, 3.5/2.5,
      0.7/1.9, 1.5/1.8, 2.3/2.0, 3.1/1.9, 3.7/1.7,
      0.6/1.1, 1.4/1.0, 2.2/1.2, 3.0/1.1, 3.6/0.9,
      0.8/0.4, 1.6/0.3, 2.4/0.5, 3.2/0.4
    } {
      \filldraw[feynblue!50] (\x,\y) circle (0.15);
    }
    % 약간의 움직임 표시
    \foreach \x/\y/\a in {1.4/3.6/30, 2.9/2.7/-45, 1.5/1.8/120, 3.0/1.1/-70} {
      \draw[-Stealth, feynred!50, line width=0.5pt] (\x,\y) -- ++(\a:0.4);
    }
    \node[font=\scriptsize\sffamily, feyngray, text width=3.5cm, align=center] at (2,-0.7) {
      가까이 모여 있지만\\위치 교환 가능
    };
  \end{scope}

  % ─── 고체 ───
  \begin{scope}[shift={(11,0)}]
    \draw[feyngray!40, rounded corners=4pt] (-0.3,-0.3) rectangle (4.3,4.3);
    \node[font=\sffamily\bfseries, feyngreen] at (2, 4.7) {고체 (Solid)};

    % 격자 결합선 (가로)
    \foreach \y in {0.5, 1.3, 2.1, 2.9, 3.7} {
      \draw[feyngray!40, line width=0.4pt] (0.5,\y) -- (3.7,\y);
    }
    % 격자 결합선 (세로)
    \foreach \x in {0.5, 1.3, 2.1, 2.9, 3.7} {
      \draw[feyngray!40, line width=0.4pt] (\x,0.5) -- (\x,3.7);
    }
    % 원자들
    \foreach \x in {0.5, 1.3, 2.1, 2.9, 3.7} {
      \foreach \y in {0.5, 1.3, 2.1, 2.9, 3.7} {
        \filldraw[feyngreen!50] (\x,\y) circle (0.15);
      }
    }
    % 진동 표시
    \foreach \x/\y/\a in {1.3/2.9/20, 2.9/2.1/-30, 2.1/1.3/45, 3.7/3.7/-15} {
      \draw[<->, feynred!40, line width=0.4pt] (\x,\y) ++(\a:0.25) -- ++(\a+180:0.5);
    }
    \node[font=\scriptsize\sffamily, feyngray, text width=3.5cm, align=center] at (2,-0.7) {
      고정 위치에서\\진동만 함
    };
  \end{scope}
\end{tikzpicture}
\end{diagrambox}


% ─── 1-4. 기체의 압력 ───
\section{기체의 압력}

기체 분자가 용기 벽에 부딪치면 무슨 일이 일어날까요? 분자가 벽에 튕겨 나올 때마다 벽에 약간의 ``밀침''을 줍니다. 수십억 개의 분자가 매초마다 벽에 부딪치니까, 이 밀침이 모이면 우리가 느끼는 \textbf{압력}이 되는 거죠.

이것을 수식으로 정리하면 바로 우리가 잘 아는 이상기체 법칙이 나옵니다:

\begin{mathbox}
\begin{equation}
  PV = nRT = Nk_BT
\end{equation}

\vspace{6pt}
\begin{center}
\begin{tabular}{ll}
  $P$ & 압력 (pressure) \\
  $V$ & 부피 (volume) \\
  $N$ & 분자의 수 \\
  $k_B$ & 볼츠만 상수 $= 1.38 \times 10^{-23}\,\text{J/K}$ \\
  $T$ & 절대 온도 (kelvin)
\end{tabular}
\end{center}
\end{mathbox}

\begin{exercisebox}
\begin{enumerate}
\item 만약 방 안의 공기 분자가 모두 같은 방향으로 움직인다면 어떤 일이 벌어질까요? (힌트: 바람이 불겠죠!)

\item 풍선을 뜨거운 곳에 두면 커지고, 차가운 곳에 두면 줄어드는 이유를 원자 수준에서 설명해 보세요.

\item 볼츠만 상수 $k_B$의 물리적 의미는 무엇일까요? ``온도 1도당 분자 1개가 가지는 에너지''와 어떤 관계가 있을까요?
\end{enumerate}
\end{exercisebox}


\begin{deepresearch}{심층 해설: 볼츠만 상수와 통계역학의 탄생}
\textbf{루트비히 볼츠만}(Ludwig Boltzmann, 1844--1906)은 오스트리아의 물리학자로, 열역학 법칙을 원자의 통계적 행동으로 설명하는 \textbf{통계역학}의 창시자입니다.

19세기 말, ``원자가 정말 존재하는가''는 뜨거운 논쟁거리였습니다. 에른스트 마흐(Ernst Mach)를 비롯한 실증주의 철학자들은 직접 관측할 수 없는 원자의 존재를 부정했습니다. 볼츠만은 이 논쟁의 한가운데서 원자론을 끝까지 옹호했으나, 극심한 학문적 고립감 속에서 1906년 스스로 목숨을 끊었습니다.

아이러니하게도, 그가 죽은 바로 다음 해에 아인슈타인의 브라운 운동 이론과 페랭(Perrin)의 실험이 원자의 존재를 확정적으로 증명했습니다. 그의 묘비에는 유명한 엔트로피 공식이 새겨져 있습니다:
$$S = k_B \ln W$$

\vspace{4pt}
{\scriptsize\color{feyngray} 출처: Wikipedia --- Ludwig Boltzmann, Statistical mechanics}
\end{deepresearch}


\vspace{16pt}
\begin{center}
\color{feyngray}\rule{0.3\textwidth}{0.4pt}

\vspace{8pt}
{\small\sffamily\color{feyngray}
--- 샘플 페이지 디자인 끝 ---\\[4pt]
이 레이아웃은 파인만 물리학 강의 전체 번역에 적용될 디자인 언어입니다.}
\end{center}

\end{document}
