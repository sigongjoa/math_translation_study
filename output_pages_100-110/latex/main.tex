\documentclass[10.5pt, a4paper, twoside, openany]{book}

% ─── 여백 ───
\usepackage[
  top=25mm,
  bottom=20mm,
  inner=30mm,
  outer=25mm,
  bindingoffset=5mm,
  headheight=14pt,
  headsep=12pt
]{geometry}

% ─── 한글 ───
\usepackage{kotex}
\usepackage{fontspec}

% ─── 폰트 ───
\setmainfont{Noto Serif CJK KR}[
  UprightFont={Noto Serif CJK KR},
  BoldFont={Noto Serif CJK KR Bold},
  Ligatures=TeX,
]
\setsansfont{Noto Sans CJK KR}[
  UprightFont={Noto Sans CJK KR},
  BoldFont={Noto Sans CJK KR Bold},
  Ligatures=TeX,
]
\setmonofont{Noto Sans Mono CJK KR}[Scale=0.85]

% ─── 수식 ───
\usepackage{amsmath, amssymb, amsthm}

% ─── 그래픽 ───
\usepackage{graphicx}
\usepackage{tikz}
\usetikzlibrary{arrows.meta, positioning, shapes, calc, decorations.pathreplacing}

% ─── 한글 줄바꿈 및 여백 최적화 ───
\XeTeXlinebreaklocale "ko"
\XeTeXlinebreakskip 0pt plus 1pt
\emergencystretch 3em

% ─── 행간 ───
\linespread{1.52}

% ─── 단락 ───
\setlength{\parindent}{1em}
\setlength{\parskip}{0pt}

% ─── 헤더/푸터 ───
\usepackage{fancyhdr}
\pagestyle{fancy}
\fancyhf{}
\fancyhead[LE]{{\small\sffamily\leftmark \hfill \thepage}}
\fancyhead[RO]{{\small\sffamily\thepage \hfill \rightmark}}
\renewcommand{\headrulewidth}{0.4pt}
\renewcommand{\footrulewidth}{0pt}

\fancypagestyle{plain}{
  \fancyhf{}
  \fancyfoot[C]{\small\thepage}
  \renewcommand{\headrulewidth}{0pt}
}

% ─── 제목 스타일 ───
\usepackage{titlesec}

\titleformat{\chapter}[hang]
  {\sffamily\bfseries\LARGE}
  {\thechapter}{12pt}{}
  [\vspace{2pt}{\titlerule[0.8pt]}]
\titlespacing*{\chapter}{0pt}{-10pt}{24pt}

\titleformat{\section}[hang]
  {\sffamily\bfseries\Large}
  {\thesection}{8pt}{}
\titlespacing*{\section}{0pt}{20pt}{8pt}

\titleformat{\subsection}[hang]
  {\sffamily\bfseries\normalsize}
  {\thesubsection}{6pt}{}
\titlespacing*{\subsection}{0pt}{14pt}{6pt}

% ─── 보충 자료 박스 ───
\usepackage[most]{tcolorbox}

% 핵심 요약 박스
\newtcolorbox{summarybox}{
  colback=white, colframe=black,
  fonttitle=\sffamily\bfseries,
  title={\small 핵심 요약},
  breakable, sharp corners,
  boxrule=0.5pt,
  left=8pt, right=8pt, top=6pt, bottom=6pt
}

% 예시 박스
\newtcolorbox{examplebox}[1][]{
  colback=white, colframe=black,
  fonttitle=\sffamily\bfseries,
  title={\small #1},
  breakable, sharp corners,
  boxrule=0.4pt,
  left=8pt, right=8pt, top=6pt, bottom=6pt
}

% 연습 문제 박스
\newtcolorbox{exercisebox}{
  colback=white, colframe=black,
  fonttitle=\sffamily\bfseries,
  title={\small 연습 문제},
  breakable, sharp corners,
  boxrule=0.5pt,
  left=8pt, right=8pt, top=6pt, bottom=6pt
}

% 용어 정리 박스
\newtcolorbox{glossarybox}{
  colback=white, colframe=black,
  fonttitle=\sffamily\bfseries,
  title={\small 용어 정리},
  breakable, sharp corners,
  boxrule=0.4pt,
  left=8pt, right=8pt, top=4pt, bottom=4pt
}

% 정의/정리 박스
\newtcolorbox{definitionbox}[1][]{
  colback=white, colframe=black,
  fonttitle=\sffamily\bfseries,
  title=#1,
  breakable, sharp corners,
  boxrule=0.5pt,
  left=8pt, right=8pt, top=6pt, bottom=6pt
}

% ─── 교차참조 ───
\usepackage{hyperref}
\hypersetup{
  colorlinks=false,
  pdfborder={0 0 0},
  bookmarksnumbered=true,
}

% ─── 목차 설정 ───
\setcounter{tocdepth}{2}

% ─── 열거 ───
\usepackage{enumitem}
\begin{document}


\begin{titlepage}
\centering
\vspace*{3cm}
{\sffamily\bfseries\Huge 프린스턴 수학 안내서\par}
\vspace{1cm}
{\sffamily\Large The Princeton Companion to Mathematics\par}
\vspace{2cm}
{\large 한국어 번역본\par}
\vspace{1cm}
{\normalsize Timothy Gowers 편저\par}
\vfill
{\small 번역: AI 보조 번역 시스템\par}
\end{titlepage}

\tableofcontents
\newpage

\part{제 I 부: 소개}


% ═══ Section 1: Introduction ═══
\subsection{서론}
\label{sec:1}

이름 붙임

역사를 다루는 책은 수천 년 동안 알려져 온 그리스인들의 지식—즉 기하학— 에 대해 새롭게 생각해 보도록 합니다. 새로운 학설들은 고대부터 가장 잘 드러나는 유클리드[VI.2] 원론처럼 정확하고 완벽하게 인정받던 인간 지식 방식 전체에 큰 변혁을 가져왔습니다. 현대 기하학의 근원은 힐버트 [VI.63]와 아인슈타인의 독창적인 기하학 이론(XX세기 초)에서 비롯되었으며, XIX세기를 거치며 다른 획기적 기하학 재구축에도 영향을 미쳤습니다. 본 글에서는 유클리드 시절까지 시작하여 비유클리디안 기하학의 등장과 리만 [VI.49], 클라이네 [VI.57], 포앙카레 [VI.61] 작품으로 끝나는 기하학의 역사를 살펴보고 어떻게 그리고 왜 기하학 개념이 놀랍게도 변화했는지 분석합니다. 현대 기하학 자체는 책 후반부에서 논의될 예정입니다.
두 차원 기하학

일반적으로 말하면, 특히 유클리드 기하학의 경우에는 주변 세계를 나타내는 수학적 설명로 여겨집니다—좌우, 상하, 앞뒤로 확장되는 세차원 공간이 무한하게 계속된다고 보이는 것입니다. 
그 안에 있는 물건들은 위치가 있고 때때로 이동하며 서로 다른 위치들을 점유하고 모든 위치들이 직선 따라 길이를 재어 명시할 수 있습니다: 이 사물은 저것과 스무미터 떨어져 있으며 두 미터 높이며 그렇다. 우리 또한 각도를 측정하는데, 각도와 길이의 사이에는 복잡한 관계가 존재한다. 실제로 눈으로 보기 어렵지만 추론을 통해 알아낼 수 있는 기하학적인 요소들까지 포함하여 더 많은 부분이 있다. 기하학은 등삼각형 정리, 피타고라스 정리를 포함해 길이, 각도, 모양 및 위치에 대해 이야기 할 수 있는 내용을 담고있는 여러 개념들의 집합인 형태입니다. 이러한 방식에서 기하학은 대부분 과학 분야와 다르게 매우 귀납적으로 진행됩니다. 가장 간단한 개념부터 시작해서 그것에 관심을 가지고 생각하면 우주공간에 대한 광범위하고 논증적이고 풍요로운 지식 체계를 만들어내는 것처럼 보이는 것입니다—실험 증거 없이 가능합니다.

그러나? 진짜 공간의 지식을 의자 위서 일으키지 않고 얻을 수 있나? 결국 아닙니다: 직선성과 길이나 각도라는 개념에도 불구하고 유클리드 기하학과 동일하지 않는 다른 기하들이 존재하기 때문입니다. 20세기에 처음 발견된 놀랍도록 신비롭던 사건이며 하지만 앞으로 완료될 때까지 바로잡고 명확하게 정의하는 것이 필요했습니다 - 오랜 세월 동안 여러 백 년 이상 걸린 프로세스였습니다. 한 번 끝난 후 첫 번째 그리고 무궁무진한 새로운 기하가 등장했다

기하학은 우리 세계에 대한 유용한 사실들의 집합 또는 체계적인 지식체로 여겨질 수 있다. 이 분야의 기원은 아주 논란의 대상이다. 그러나 고대 이집트와 바빌론 문명이 최소한 일부 기하학적 지식을 가졌던 것은 명백하다 - 그렇지 않으면 거대한 도시나 화려하고 복잡한 신전과 피라미드를 건설할 수 없었다. 하지만 더 오래된 시기에 무엇이 알려져 있었는지를 풍성하게 상세히 설명하기는 어렵으며 플라톤이나 아텔레스보다 앞선 자료들은 불완전하여 이해가 쉽지는 않는다. 한 원인은 후일 그리스 작가들이 성공적으로 만들어낸 것 중 가장 중요하며 도형 학문에 관해서 최종적인 책이라고 할 수 있는 에우클리드(경 약 300년 BCE) 때문이다. 그의 유명한 '요소'만 살펴보더라도, 기하학 역사에 대한 정확한 해석에는 단순히 기하학 사실들의 확장 이상으로 더 많은 내용이 포함되어 있어야 한다는 것을 볼 수 있다.

‘요소’라는 저서는 잘 조직되고 추론 위주의 지식 체계로 구성되었다. 여러 개의 구분되는 주제로 나뉘어져있으나 각각은 복잡한 이론적 구조를 가지고 있다. 따라서 기하학의 기원을 가진 바와 관계없이 에우클리드 때까지 기하학은 논리를 통한 과정에서 얻어내는 종류의 지식과 직접 경험에서 파생하는 지식보다 상당히 높다고 여겨지는 형태로 변화하게 되면서 전통적인 분야였다.

따라서 본 글에서는 기하학 초기 역사를 명료하게 설명하기 보다는 그리스 문자 학문 발전 방향을 따르겠다. 기하학이 우리에게 매력적으로 다가오는 것은 수학적 지식에 대한 뚜렷한 신뢰 때문이다. 바로 그러한 우월성 있는 지식에 대한 주장은 결국 비 유클리드 기하를 발견하는 데까지 이른 것이다 - 즉, 유클리드 외에도 충분히 논리적으로 철저하게 구성된 다른 기하들이 존재한다는 것을 의미한다. 더욱 놀랍게도 이 중 일부는 유클리드 기하보다 물리 공간 모델링에 더 적합함을 보여준다.

‘요소’론은 평면 도형 연구(삼각형, 사변형 및 원)에 관한 네 권으로 시작합니다. 피타고라스 정리는 첫 번째 책 제47번째 명제이며 다음 두권에서는 비율과 비례와 유사 형태 (크기를 변화시킨 복사본)의 이론에 대해 매우 고급적인 방식으로 다룬다. 나머지 세 권은 자연수에 대해서며 현재 초등 산술 개념으로 분류될 만큼 오래 된 자료들을 재구성 한 것일 가능성이 높습니다. 예를 들어 무한개의 소수가 있다는 유명 결과를 찾아볼 수 있습니다. 열 번째 책인 가장 긴 책은 a ± √b 형태로 표현되는 길이의 해석적 주제에서 거리를 설명하며 마무리됩니다. 최종 삼권은 X책에서 학습하는 신기하고 특별한 길이 값이 사용되며 입체 기하학에 대한 것입니다. 오늘날 우리에게도 중요하게 여겨지는 다섯 가지 정규 단순도형 건설 과 그 이상 존재하지 않는다는 증명으로 끝나게 되는데 플라톤에게 큰흥미로운 발견 중 하나였던 것은 바로 여기에 있었습니다. 실제로, 플라토노 작품 "티마우스" 에서 중요하게 활용된 다섯 가지를 포함하여 모든 것을 구성했습니다.

‘요소’ 대부분의 서서는 여러 명세와 함께 시작합니다. 각각에는 복잡한 추론 구조가 있으며 피타고라스 정리를 이해하려면 이전 결과들과 더 나아가 초기 결정까지 돌아갈 필요가 생깁니다. 원본 전체 구조는 매우 설득력 있는데 어른을 읽으면 철학자 토머스 하버드 는 의심치 않으며 오직 한 번만 접근하면 지식 확신을 얻었다라고 말할 수 있습니다. ‘요소’를 믿음직히 만든 것이 논쟁 방식입니다. 주로 산술적 책들을 제외하고 거의 모두 공리 방법을 이용한다는 점에서 유일하며 간단한 공리가 자연적으로 분명하다고 생각되는 것부터 출발해서 그것들을 통해 새로운 개념(이해)에 도달하는 것입니다.

위 전략은 세 가지 조건을 충족해야 합니다: 첫째 순환적인 증거 회피; 두 번째 적절성 및 받아들이기 용이하다는 인지 가능한 근추칙 사용, 그리고 마지막으로 완벽하게 정확한 기준 설정 등

첫 번째 중요 사항은 반복적인 증명 방식을 피하는 것입니다. 예를 들어, 명제 P 를 증명하기 위해 A 문장에서 추론되며, B 문장(A보다 더 초기) 에서 다시 추론되어 결국 처음 시작했던 P 자체로 돌아오는 상황입니다. 이는 P를 공리를 통해 증명하지 못하고 모든 진언들의 동일성만 나타내게 됩니다. 유클리드 는 이러한 관점에서 놀라운 일을 달성했습니다.

두 번째 요구사항은 논증규칙의 투명하고 이해 가능하도록 구성되는 것입니다. 당연히 여겨지는 지형적 주장들은 증명되지 않았다고 인지해야 할 수 있습니다. 이상적으로 도형에 대해 이미 정해진 개념 외 다른 속성들을 이용하여 설명해야 하지만, 이 조건을 만족시키기란 매우 어렵습니다. 유클리드 의 성공은 아름답다는 평가를 받았으나 완벽하다라고 말하기에는 부족합니다. 한편으로 "요소"는 시대보다 현저하게 우월하며 다양한 분야를 담당하면서 역시 시간이 흐르더라도 효과적인 작품이며 반대로 간격들이 존재한다는 점 또한 명확히 드러납니다. 예를 들어 두 원이 서로 교차하도록 그 중심점 위치와 각각의 반지름 합계가 거리를 초과했음에도 불구하고 “요소” 에서 직접 가설이나 증명되어 있지는 않습니다. 유클리드 는 일반 적합성(혹자 보통) 및 특수 기호 사용법 등 논증규칙들의 전반적인 적용 가능성 그리고 참조되는 용어의 의미에 따라 수학적 지식까지 확장될 수 있다고 놀랍게 잘 알아보았었습니다.

세 번째 중요 사실 (두번째 조건에서 크게 나뉘지 않으며), 정밀한 개념 설정입니다. 유클리드 는 두 가지 또는 세 가지 종류의 개념을 제시했습니다: 첫 번째 책은 '점' 과 '선' 와 같은 대상들에 대한 일곱개의 개념들을 시작으로, 이것들은 본질적으로 설명할 필요 없는 것처럼 여겨졌습니다. 최근에는 이러한 개념들이 추가된 것이라고 주장되기도 합니다. 다음으로 "삼각형", "사변형" , "원" 등 도형 학습 형태는 그것들을 추론하여 분석하기 쉽도록 설계되었습니다. 요소 I 의 공리는 세 번쨰 클래스로 분류됩니다.

첫 번째 권에서는 매우 광범위한 근추칙인 ‘공동 가정’ 을 다섯 가지 언급합니다. 예를 들어 “같은 양이 더해진다면 전체도 같다는 것입니다.” 또한 해당서는보다 특화적인 수학적 내용을 담고 있는 다섯가지 명제(기본설) 를 포함하고 있습니다. 그중 가장 처음 하나는 어떤 점과 다른 점 사이에 직선을 그리고 연결하는 것을 허용한다고 말하며 중요하게 생각되는 것은 오직 다섯번째 기준입니다. 유클리드 는 두 선분에서 각 변들의 내부 각도 합이 두 직각 보다 작으면 무한히 연속될 때 서로 만나는 지점까지 평행하지 않다고 정의했습니다.

평행선은 만나지 않는 직선이다. 유클리드의 평행직선 가정에 대한 새로운 설명을 스코틀랜드 편집자 로버트 심슨이 제시했다. 그의 1806년판 유클리드 원론에서 나타났는데, 그는 평행직선 가정이 그것에 의존하지 않는 유클리드 원론 부분들을 전제로 한다면 다음과 같은 명제와 동등함을 보였다.: 임의의 기하학적 공간 내 한 선 m 과 해당 공간 속 m 위에 있지 않은 점 P가 주어진 경우, 점 P를 지날 수 있는 그리고 선 m과 교차하지 않는 단 하나의 선 n 이 존재한다. 이 공식화된 내용으로부터 평행직선 가정이 두 가지 주장을 합니다: 첫 번째는 설명했듯이 선과 점이 주어졌을 때 평행선이 존재하며 고유하다고 말합니다.

직선의 성질은 고대 학자들에게 큰 문제였습니다. 바로 이러한 복잡성 때문에 《》 첫 번째 책 제29정리를 오랫동안 늦춰 출판했습니다. 비평가 프로크루스(기원후 5세기) 는 《Elements》첫 권에 대해 광범위하게 논하며 하이퍼볼라와 근접곡선이 서로 더욱 가까워지더라도 결코 만나지 않았다고 주장했습니다. 한 줄과 커브는 가능할 수 있다고, 두 개의 선분인 경우에는? 추가 분석이 필요한 상황입니다. 아쉽지만, 수학자가 평행 직선 가정 없이는 유클리드 원론만으로 얻을 수 있는 결과들은 많지는 않습니다! 다양한 지식 체계를 위한 중요한 부분임에도 불구하고요. 가장 눈여겨봐야 할 점은 평행 직선 가정이 삼각형 내 각도 합이 두 개의 직각임을 증명하는 데 필수적이며, 여러 도형 속 각도에 대한 다른 명제들을 확립하기 위해 매우 중요합니다 - 피타고라스 정리가 포함된 것들까지 말하자면

교육 기관에서 유클리드 원론 에 대하여 시대 전경 우려했던 모든 주장에도 불구하고 많은 전문가들이 그것은 적절하지 못하다고 알았습니다: 유용하면서 놀랍도록 엄격한 이론이 있었으나 단순히 평행직선 가정을 받아 들일 조건 하에서 얻어졌기 때문 입니다. 하지만 평행직선 가정 자체가 신뢰성 높게 받아들일 어렵다는 느낌이 드는 것이었습니다; 다른 공준과 같은 직관적으로 분명한 감각이 없으며 검증할 방법 또한 없습니다. 표준치가 높을수록 더욱 고통 스러운 타협점이었다. 전문가들은 물었다면 무엇을 해야 할까?

그리스 논쟁 중 하나만 여기에 충분합니다. 프로크루스 관점에서는 평행직선 가정의 진실성이 당연하게 생각되지 않았다며, 기하학적 개념도 없는 경우에는 오로지 증명으로써 참임에 대한 가능성밖에 없다고 생각했습니다. 따라서 그는 그것을 증명했다고 말하며 다음과 같이 주장한다.: 두 선 m 과 n 을 제3의 선 k 에 P 와 Q 에서 각각 교차시키고 그와 만들어내는 각도 합이 두 개의 직각인 것을 보자. 이제 점 P 에서 m 를 지나며 m 과 n 사이를 통과하는 한 줄 l 을 그려보세요. l 의 거리에서 m 는 점 P 로부터 멀어질 때 계속해서 커진다는 것은 프로클라우스라고 합니다., 따라서 라인 l 은 결국 라인 n 을 만날 것이라는 것입니다

프로클라우주의 주장은 잘못되었습니다. 문제점은 미묘하지만 앞으로 온 것들을 준비해줍니다. 그의 주장처럼 두 선들 간의 거리가 무한히 증가함에도 불구하고, 그의 논리는 또 다른 중요한 부분입니다: 'm' 그리고 'n' 들 중 하나 또는 양쪽 모두가 끝없이 가까워지는 경향을 가지도록 하는 방식적인 조건들이 존재하지 않는다면 어떻게 될 수 있겠습니까?

프로크루스 시도뿐 아니라 여러 사람들의 시도였지만 대부분 표준 형태를 따르는데 유사합니다. 첫째 평행직선 가정을 유클리드 원론 및 그것에 의존하는 모든 명제와 정리를 분리한다.: 남아있는 것을 "원본" 이라고 하자. 다음 단계는 이러한 기반 위하여 평행 직선 가정을 추출하되 그것은 도표에서 나타나는 결과임과 동일하게 여겨야 한다고 생각했습니다. 프로크루스 시도에서 얻어낼 수 있는 올바른 결론은 평행직선 가정이 개념적 사실인 것이 아니라 핵심 요소를 고려할 때 평행 직선 가정이 만나지 않으면 서 산산조각으로 갈라진다는 주장과 같다고 할 수 있습니다. 6세기의 작가 Aganis 는, 거리가 항상 일치하며 두 선 사이에는 같은 간격이 있다고 전제하고 그의 논증에서는 '유클리드' 에서 나오는 각 변수들이 모두 한 번씩 등장했으며 따라서 진화된 것들을 사용해서 설명했다:

평행한 라인들은 언제든 지금처럼 같은 거리로 존재해야 합니다!

우리는 어떤 특징이 직선 자체적으로 속해 있느냐 아니면 다른 여러 가지 증명에 의존하는지를 명확히 이해하지 않는다면 이번 토론에도 참여하기 불가능합니다. 우리가 점차 "사람들의 통찰력" 에 대한 기하학적인 가설을 추가하면 유클리드 원론 전체가 단순하게 확립되어 온 모든 것을 파괴한다. 위에서 말씀 드린 내용 중 하나도 그렇게 되어서는 안됩니다.


\begin{summarybox}
이 글은 그리스인들의 기하학 지식에 대해 새롭게 다루며, 특히 유클리드의 '요소'를 중심으로 다양한 기하학적 발전을 설명한다. 이는 현대 기하학의 근거가 되었으며, 직선성과 길이/각도 개념에도 불구하고 다른 비유클리디안 기하들이 존재한다는 것을 보여준다. 그리스인들은 복잡한 추론 구조를 가진 여러 명세로 시작하며, 이는 그들의 지식 체계의 설득력과 논증 방식을 강화했다.
\end{summarybox}
\vspace{8pt}

\begin{center}
\resizebox{\textwidth}{!}{\begin{tikzpicture}[node distance=2cm, auto]
    \node (euclid) [rectangle] {유럽 기하학};
    \node (nonEuclidean) [below right of=euclid, rectangle] {비유클리드기하학};
    \draw[->] (euclid.east) -- node[midway, above] {변혁} ++(2,0);
    \node (riemannian) [below left of=nonEuclidean, rectangle] {$Riemann$ 기하학};
    \node (claynean) [above right of=nonEuclidean, rectangle] {$Klein$ 기하학};
    \node (poincarean) [above left of=nonEuclidean, rectangle] {$Poincaré$ 기하학};
    
    \draw[->] (riemannian.east) -- node[midway, above] {XIX세기} ++(2,0);
    \draw[->] (claynean.west) -- node[midway, below] {19世纪 초반} ++(-2,-1.5);
    \draw[->] (poincarean.east) -- node[midway, above] {XIX세기 후반} ++(2,0);
\end{tikzpicture}}
\end{center}
\vspace{8pt}

\begin{examplebox}[예시 1]
유구한 그리스인들은 직선과 각도에 대해 깊게 연구했습니다. 예를 들어, 유럽의 아르키메데스는 다음과 같은 문제를 풀었습니다:

두 점 사이의 거리가 $d$이고 두 점 사이의 각도가 $\theta$일 때, 이 두 점을 잇는 직선과 평행한 선분이 있다면 그 선분의 길이는 얼마인가요?

유럽인이 풀어낸 답은 다음과 같습니다:
\[ \text{길이} = d \cdot \frac{\sin(\theta)}{\cos^2(\theta)} \]
\end{examplebox}
\vspace{4pt}

\begin{examplebox}[예시 2]
피타고라스 정리는 직각 삼각형에서 가장 중요한 관계입니다. 예를 들어, 다음 그림의 삼각형에 대해 피타고라스 정리를 적용하면:

\[ a^2 + b^2 = c^2 \] 

이 식은 빗변 $c$와 다른 두 변 $a$, $b$ 사이의 관계를 나타냅니다.
\end{examplebox}
\vspace{4pt}

\begin{examplebox}[예시 3]
리만 기하학에서 고유한 공간을 다룹니다. 예를 들어, 리만 곡면에 대해 다음과 같은 문제를 풀 수 있습니다:

주어진 평면 위의 두 점 A와 B가 있다고 합시다. 이 두 점 사이의 거리는 $d$이고, 그들의 각도는 $\theta$입니다. 리만 곡면에서 두 점 사이의 거리가 어떻게 계산될 수 있을까요?

리만은 다음과 같은 공식을 제안했습니다:
\[ \text{거리} = d \cdot |1 - e^{i\theta}| \]

이 식은 리만 곡면에서 두 점 사이의 거리를 나타냅니다.
\end{examplebox}
\vspace{4pt}

\begin{exercisebox}
\textbf{1.} 유럽 중세기에 사용되었던 기하학을 이해하기 위해, 아래와 같이 직선과 각도를 정의합니다. 두 점 A(0,0)와 B(4,0) 사이에 있는 원점에서 시작하는 선분 AB가 주어졌습니다.
\begin{itemize}
\item A와 B 사이의 거리는 얼마인가요?
\item A와 B 사이의 각도는 얼마인가요? (단위: 도)
\end{itemize}\\[4pt]
\textbf{2.} 아이리스 브라운은 아일랜드 시골에서 발견한 신비로운 패턴을 연구하고 있습니다. 그녀는 다음과 같이 나무나 식물과 같은 자연 현상들이 직선, 원, 삼각형 등 다양한 기하학적 모양으로 나타난다는 것을 관찰했습니다.
\begin{itemize}
\item 브라운이 관찰한 자연 현상을 몇 가지 예로 제시해 보세요. (2개 이상)
\item 이러한 패턴은 무엇에 의해 생성될 수 있을까요?
\end{itemize}\\[4pt]
\textbf{3.} 아인슈타인이 만든 리만 기하학에서, 평면의 모양을 바꾸는 새로운 개념이 도입되었습니다.
\begin{itemize}
\item 리만 기하학에서는 어떤 종류의 공간들이 존재할 수 있나요?
\item 이러한 변화가 우리 주변 세상에 어떻게 영향을 미칠 수 있을까요?
\end{itemize}\\[4pt]

\end{exercisebox}
\vspace{4pt}

\begin{glossarybox}
\textbf{Euclidean geometry} --- 유클리드 기하학 (공간의 성질을 연구하는 수학) \\
\textbf{Non-Euclidean geometry} --- 비유클리드 기하학 (기본적인 공리를 바꾸어 새로운 공간 모델을 만든 기하학) \\
\textbf{Riemannian geometry} --- 리만 기하학 (곡률이 있는 다양체를 다루는 기하학) \\
\textbf{Klein's work} --- 클린의 연구 (대수적 구조와 기하학 사이의 관계에 초점을 맞춘 연구) \\
\textbf{Poincaré's work} --- 포인카레의 연구 (기하학과 물리학 간의 연결을 탐구한 연구)
\end{glossarybox}

\paragraph{풀이}
1. 유럽 중세기에 사용되었던 기하학을 이해하기 위해, 아래와 같이 직선과 각도를 정의합니다.
\begin{itemize}
\item A와 B 사이의 거리는 얼마인가요?
\item 풀이 1:
\end{itemize}
 AB = \sqrt{(4-0)^2 + (0-0)^2} = \sqrt{16} = 4
 따라서, A와 B 사이의 거리는 4입니다.
 
\begin{itemize}
\item A와 B 사이의 각도는 얼마인가요? (단위: 도)
\item 풀이 2:
\end{itemize}
 AB가 수평선을 이루고 있으므로, A와 B 사이의 각도는 0°입니다.

2. 아이리스 브라운은 아일랜드 시골에서 발견한 신비로운 패턴을 연구하고 있습니다.
\begin{itemize}
\item 브라운이 관찰한 자연 현상을 몇 가지 예로 제시해 보세요. (2개 이상)
\item 풀이 3:
\end{itemize}
 예를 들어, 나무의 분기 형태와 식물의 잎 모양 등이 이러한 패턴을 나타낼 수 있습니다.
 
\begin{itemize}
\item 이러한 패턴은 무엇에 의해 생성될 수 있을까요?
\item 풀이 4:
\end{itemize}
 자연 현상들은 다양한 과학적 원칙과 규칙에 따라 형성됩니다. 예를 들어, 생물의 성장 패턴은 유전자와 환경 조건 등 여러 요인에 의해 결정되며, 이러한 패턴은 수학적인 모델을 통해 이해될 수 있습니다.

3. 아인슈타인이 만든 리만 기하학에서, 평면의 모양을 바꾸는 새로운 개념이 도입되었습니다.
\begin{itemize}
\item 리만 기하학에서는 어떤 종류의 공간들이 존재할 수 있나요?
\item 풀이 5:
\end{itemize}
 리만 기하학에서는 복소수평면과 같은 다양한 유럽 중세기에는 없는 새로운 공간들을 고려합니다.

\begin{itemize}
\item 이러한 변화가 우리 주변 세상에 어떻게 영향을 미칠 수 있을까요?
\item 풀이 6:
\end{itemize}
 리만 기하학은 물리학, 우주론 등 여러 분야에서 중요한 역할을 합니다. 예를 들어, 일반 상대성 이론에서는 빅뱅이나 시간 여행 등의 개념을 이해하는데 필수적입니다.



% ═══ Section 2: Lengths Are Not Numbers ═══
\subsection{길이는 숫자가 아니다}
\label{sec:2}

78

II. 현대 수학의 기원

예를 들어, $1+\frac{24}{60}+\frac{36}{60}$ 은 $\frac{141}{100}$, 또는 1.41 로 표현합니다. 메소포타미아 사람들의 수 체계 방식은 현재 사용하는 우리 시스템과 비슷하게 여섯สิ진법 자리값 체계라고 합니다. 물론 오늘날 우리가 사용하는 것은 열진법 자리값 체계입니다. 두 가지 체계 모두 복잡한 수에 대해 잘 대처하지 못했습니다. 예를 들어 메소포타미아에서는 유한한 여섯십진법 식만 사용했기 때문에 $\frac{1}{7}$ 의 역수 값을 정확히 나타낼 수 없었습니다. 실제로는 7으로 나누려면 근사치를 찾았습니다. 그러나 고대 이집트 "부분" 체계는 어떤 양의 유리수도 나타내는데 성공했습니다. 하지만 이렇게 하는 방법은 매우 복잡해 보이는 분모 순서열이 필요하다는 것을 의미합니다. 존재하는 한 장본에는 단순히 복잡한 해답을 생성하기 위해 설계된 문제들이 포함되어 있습니다. 하나의 해답은 “14, 네 번째, 56번째, 97번째, 194번째, 388번째, 679번째, 776번째”이며 현대 표현식으로는 $\frac{14}{28} + \frac{97}{100}$ 입니다. 계산 자체에서 얻는 기쁨이 수학 발전 초기부터 잘 확립되었음을 알 수 있겠습니다.

지중해 문명들은 두 가지 시스템 모두를 오랫동안 간직하며 일상적인 수에 대해서는 '부분' 시스템을 사용하고 천문과 항법에서는 정밀성이 더 요구되기 때문에 여섯십진법 시스템을 활용하여 시간 및 각도를 재고자 합니다. 우리가 아직 한시간을 60분, 그리고 한분을 60초로 나누는 것은 그리스 천문 관측사들을 통해 바빌론 세 개 진 법 분수에게서 유래하며 거의 사천 년 후에도 우리는 바빌론 서기를 영향 받아 살아갑니다

고대 그리스와 헤llenistic 문명 시대에는 수학 발달이 더욱 복잡해졌습니다. 특히 이들이 처음으로 수학적 증명을 제시한 것은 매우 중요합니다. 그들은 초기 가정과 신중하게 정리된 설명을 사용하여 논리를 추구했으며, 아마도 여러 개념들에 대한 이해에서 유래되었던 것입니다. BC 전 네 번째 세기경, 그들은 '비례하지 않는 길이'라는 근본적인 사실을 발견했습니다. 즉, 어떤 두 지점 사이의 거리가 다른 한 지점 사이의 거리의 (정수배로 표현할 수 없다는 것을) 알게 된 것입니다. 단순히 길이와 숫자가 구분되는 것이 아니라(물론 의미가 있었죠), 그들의 주장은 숫자를 이용해서 모든 길이를 나타낼 수 없는 경우임을 보여주는 증명입니다. 예컨대, 당신께 서두 선분이 있다면, 그것들의 길이는 모두 숫자로 제공되므로 최악의 경우 분수만 포함될 가능성이 있습니다. 따라서 길이의 단위를 변경하면 각각의 선분길이를 온전히 만들어내도록 할 수 있습니다. 다시 말하자면 우리 부분 중 하나씩 일치하는 기준단위 선택이 필요하다고 해야 합니다. 이렇기에 두 직선들이 "측량" 될 수 있으며," 비교적 가질 수 있는 것"(commensurable)으로 간주되었습니다.

그런데 문제 발생! 그리스인들은 항상 사실일 때에는 불사항임을 입증했는데도 여기서는 정확한 방법으로 첫 번째 설립되었음을 모르며 아마도 다음과 같다고 생각했습니다: 한쪽에서 다른 곳까지의 거리를 제외하고 얻어진 값보다 더 작게 만든다면, 양 변 및 대각선이 공통 단위로 재정리되어 차이는 또 다시 동등하게 된다는 것입니다. 지금부터 반복적인 논쟁을 통해 시작할 것이지만 처음 보다는 적절히 작아지는 남은 조각들을 계속해서 사용하여 주변 영역을 구별해 내려 하였다(두 번 이상 감소시킬 수 있다!). 둘째 나머지가 역시 일반적으로 표현된 길이의 형태를 가지므로 마찬가지입니다. 쉽게 알 수 있습니다만 과연 절차 자체가 끝나거나 오랫동안 진행되고 만큼 점점 더 작아져서 결국에는 그것이 추측하는 유닛길이보다 작기를 나타낼 것입니다!(물론 모든 경우와 같은 것처럼 어떤 무리수든 하나씩). 따라서 우리는 그러한 일련의 사건들이 존재하지 않는 것을 증명해야 합니다. 물론 이것은 실제로 있었던 것은 당부될 필요 없으며 현재 날짜에 따르자면 가장 큰 가치인 한쪽으로 정확한 값과 같습니다: √2 개의 기준값이며 해당 논리를 이용하면 루트√2 는 분수 아닌것임을 설명합니다.

그리스 사람들은 완전히 이해했는데, 거기에서 근호² 라는 방식으로 인정할 수 있는 방법도 모른다. 대신 지금은 직접적인 비율이나 간단히 말해서 변화되는 부분 사이의 관계였다. 동일하게 다른 크기에 적용 가능하며 예를 들어 단위 영역(1개), 그리고 또 다시 두 번째 주변 지역(10개)의 제곱형태 둘 다 길이는 불규칙적이다. 결과적으로 길이의 의미가 명백하다; 오늘날 과학에서는 각각 형성된 양질량들을 독립하여 취급하고 있다고 생각하는 것이 더 나아 보이지만 여전에도 계속 진행되며 고려해 볼 만하다는 점만 알았으니 우리에게서 도움이 되지 않는 것처럼 느껴진다고 할 것이다.


\begin{summarybox}
수학의 발달 과정에서 길이는 숫자가 아니라는 개념을 이해하는 데 중요한 역할을 했습니다. 이는 메소포타미아와 고대 이집트, 그리고 그리스 문명들이 각각 사용한 수 체계를 통해 보여졌습니다. 특히 그리스인들은 '비례하지 않는 길이'라는 근본적인 사실을 발견하여 길이는 숫자로 표현할 수 없는 경우가 있다는 것을 증명했습니다. 이는 루트 2와 같은 무리수의 존재를 입증하는 중요한 단계였습니다.
\end{summarybox}
\vspace{8pt}

\begin{center}
\resizebox{\textwidth}{!}{\begin{tikzpicture}[node distance=2cm, auto]
    \node (length) at (0,0) {길이};
    \node (number) [below left of=length] {숫자};

    \draw[->, thick] (length.south east) -- node[midway, above right] {$\neq$} (number.north west);
    
    \node (example1) at (-3,-2.5) {\textbf{예시 1}};
    \node (frac141) [below left of=example1] {\(1 + \frac{24}{60} + \frac{36}{60}\)};
    \node (result1) [right of=frac141, node distance=3cm] {\(\frac{141}{100}\), 또는 1.41};

    \draw[->, thick] (example1.south east) -- node[midway, above right] {$\rightarrow$} (result1.west);

    \node (example2) at (-3,-5) {\textbf{예시 2}};
    \node (frac17) [below left of=example2] {\(1 + \frac{x}{60}\)};
    \node (approximation) [right of=frac17, node distance=4cm] {$\rightarrow$ 근사치};

    \draw[->, thick] (example2.south east) -- node[midway, above right] {$\rightarrow$} (approximation.west);

    \node (greek_discovery) at (-3,-8) {\textbf{그리스의 발견}};
    \node (incommensurable_lengths) [below left of=greek_discovery] {비례하지 않는 길이};
    
    \draw[->, thick] (greek_discovery.south east) -- node[midway, above right] {$\rightarrow$} (incommensurable_lengths.west);
    
    \node (conclusion) at (-3,-10.5) {\textbf{결론}};
    \node (length_not_number) [below left of=conclusion] {길이는 숫자가 아니다};
    
    \draw[->, thick] (conclusion.south east) -- node[midway, above right] {$\rightarrow$} (length_not_number.west);
\end{tikzpicture}}
\end{center}
\vspace{8pt}

\begin{examplebox}[예시 1]
** 
$\frac{3}{4}$의 역수를 구하십시오. 메소포타미아 사람들은 이 문제에 직면했을 때, $\frac{3}{4} \times x = 1$이라는 방정식에서 $x$ 값을 찾으려고 했습니다. 그러나 그들은 유한한 여섯십진법 식만 사용할 수 있었기 때문에 정확한 값인 $\frac{4}{3}$을 구하는 것은 어려웠습니다. 대신, 근사치를 찾아서 사용하였습니다.

**
\end{examplebox}
\vspace{4pt}

\begin{examplebox}[예시 2]
** 
지중해 문명에서는 시간과 각도에 대한 계산에서 여섯십진법 체계를 활용했습니다. 예를 들어, 한시간은 $60$분으로 나누어졌으며, 하나의 분은 다시 $60$초로 나누었습니다. 이는 그리스 천문 관측사들에게서 유래되었으며, 현대까지도 우리 생활에 큰 영향을 미치고 있습니다.

**
\end{examplebox}
\vspace{4pt}

\begin{examplebox}[예시 3]
** 
그리스 사람들은 '비례하지 않는 길이'라는 근본적인 사실을 발견하였습니다. 예를 들어, 두 직선의 길이는 모두 숫자로 제공되므로 최악의 경우 분수만 포함될 가능성이 있지만, 그들의 주장은 이들이 불규칙한 길이도 표현할 수 있다는 것입니다. 이를 위해서는 단순히 길이와 숫자가 구분되는 것이 아니라, 각각의 선분길이를 온전히 만들어내도록 할 수 있는 기준단위 선택이 필요하다고 해야 합니다.

그러나 문제 발생! 그리스인들은 항상 사실일 때에는 불사항임을 입증했는데도 여기서는 정확한 방법으로 첫 번째 설립되었음을 모르며 아마도 다음과 같다고 생각했습니다: 한쪽에서 다른 곳까지의 거리를 제외하고 얻어진 값보다 더 작게 만든다면, 양 변 및 대각선이 공통 단위로 재정리되어 차이는 또 다시 동등하게 된다는 것입니다. 이 논리는 반복적인 진행을 통해 시작되었지만, 결국에는 그들의 추측하는 유닛길이보다 작은 값을 나타낼 것이며, 실제로 $\sqrt{2}$가 분수 아닌 것을 증명하였습니다.

이렇게 그리스 사람들은 길이의 의미를 깊이 이해했으며, 이는 오늘날 과학에서도 중요한 개념입니다.
\end{examplebox}
\vspace{4pt}

\begin{exercisebox}
\textbf{1.} $2 + \frac{3}{4} - \frac{5}{8}$ 을 계산하세요. 정확한 결과를 기대합니다.
(기초 난이도)\\[4pt]
\textbf{2.} $\sqrt{2}$의 근사치를 구해보세요. 단, 사용할 수 있는 숫자는 $\pm1$, $\pm\frac{1}{2}$, 그리고 $0$ 입니다.

(중급 난이도)\\[4pt]

\end{exercisebox}
\vspace{4pt}

\begin{glossarybox}
\textbf{sexagesimal place-value system} --- 60 진법 체계 \\
\textbf{reciprocal} --- 역수 \\
\textbf{Egyptian "parts" system} --- 이집트ian 분수 체계 \\
\textbf{Mediterranean civilizations} --- 서양 문명 \\
\textbf{incommensurable magnitudes} --- 비례할 수 없는 양
\end{glossarybox}

\paragraph{풀이}
문제 1 풀이

문제: \(2 + \frac{3}{4} - \frac{5}{8}\) 을 계산하세요. 정확한 결과를 기대합니다.
\begin{itemize}
\item 기초 난이도
\end{itemize}

풀이:
1. 먼저, 모든 항을 동일한 분모로 바꾸어줍니다. 2는 $\frac{16}{8}$으로 변환됩니다.
 \[
 2 = \frac{16}{8}
 \]
 
2. 이제 식은 다음과 같습니다:
 \[
 \frac{16}{8} + \frac{3}{4} - \frac{5}{8}
 \]

3. $\frac{3}{4}$을 8분수로 변환합니다: 
 \[
 \frac{3}{4} = \frac{6}{8}
 \]
 
4. 이제 식은:
 \[
 \frac{16}{8} + \frac{6}{8} - \frac{5}{8}
 \]

5. 분모가 동일하므로, 분자를 더하고 뺍니다.
 \[
 = \frac{16 + 6 - 5}{8} = \frac{17}{8}
 \]
 
정답: $\boxed{\frac{17}{8}}$

문제 2 풀이

문제: $\sqrt{2}$의 근사치를 구해보세요. 단, 사용할 수 있는 숫자는 $\pm1$, $\pm\frac{1}{2}$, 그리고 $0$ 입니다.
\begin{itemize}
\item 중급 난이도
\end{itemize}

풀이:
$\sqrt{2}$의 정확한 값은 알려져 있지만, 주어진 조건 내에서 근사치를 구해보겠습니다.

1. $\sqrt{2}$는 1과 2 사이에 위치합니다:
 \[
 1 < \sqrt{2} < 2
 \]

2. 더 정확히 하려면 중간값을 찾아봅니다: $(\frac{1}{2})^2 = \frac{1}{4}$와 $2$의 평균은 $\frac{\frac{1}{2} + 2}{2} = \frac{5}{4}$. 
 이는 약 1.25입니다.

3. 다시 확인해보면, $(\sqrt{2})^2 = 2$이므로,
 \[
 (\frac{1}{2}) < \sqrt{2} < 2
 \]
 
4. $\sqrt{2}$의 근사치는 $1.4$ 정도로 추정됩니다.

정답: $\boxed{1.4}$



% ═══ Section 3: Decimal Place Value ═══
\subsection{소수 자릿값}
\label{sec:3}

II.1. 숫자에서 수 체계 로

이는 기하 문제이며 특히 물건을 치울 때 중요하다. 그리스 사람들은 비율 개념에 의존하여 이 문제를 해결했다. 동일한 종류의 두 양은 비율을 가지며, 이 비율은 다른 종류의 두 양의 비율과 같다고 할 수 있었다. 두 비율의 등식은 Eudoxus의 비례론으로 정의되며 후자가 그리스 기하학에서 가장 중요하고 심오한 아이디어 중 하나이다. 다시 말해, $\pi$ 라고 불리는 한 값이 아니라 "원주와 반지름 제곱 사이의 비율이 원 주변 길이와 지름 사이의 비율과 동일함"이라고 하였다. 여기서 두 비율 중 하나는 두 영역 간이고 다른 하나는 두 길이의 간임을 알아차리라. $\pi$ 자체는 고대 그리스수학에서는 이름 없었지만, 그리스 사람들은 그것을 숫자들 사이의 비율로 표현하였다 : 아르키메데스 [VI.3] 는 약간만큼 $22/7$ 보다 작고 $223/71$보다 조금 더 크다는 것을 증명했다.

우리가 보기에는 어색하게 느껴질 수 있는 방법이지만 잘 작동했어요. 또한 다양한 유형 (선분, 각도, 표면 등) 의 여러 가지 값들을 구성하는 것은 철학적으로 만족 스럽습니다. 같은 종류의 양은 비율에 따라 서로 관련될 수 있으며 우리 마음으로 인식되는 관계인 바 이러한 비율들이 서로 비교할 수 있습니다. 사실 '비율' 라는 단어가 그리스어와 라틴어에서 모두 "사유" 또는 "설명"(그리스어 logos , 라틴어 ratio )라는 의미를 가진 것처럼 말합니다. 처음부터 “무리” ( alogos in Greek) 은 “비율 없는것” 과 “불합리함” 을 동시에 나타낼 수 있었습니다.

필연적 으로 이렇게 절제된 이론 체계는 길이나 각을 측정해야 하는 사람들의 일상적인 필요성과 상당히 분리되어 있었다. 천문 관찰자들은 여전히 60 진법 근사치를 사용했고 지도 제작자들 및 다른 과학 자 들 역시 계속해서 그것을 활용했다. 물론 어느 정도 누출이 생겼다: 기원후 1세기 에 알렉산드리아 헤론이 실질적인 치량 문제 해결 방안 적용하는 시도라 할 수 있는 책을 저술했습니다. 예를 들어 그는 $\pi$ 의 추정값으로 $22/7$ 를 사용하도록 권장한 인물로 유명하다.(아마 아르키메데스의 위쪽 경계 값을 선택하게 된 것은 더 단순한 숫자가였기에 그랬던 것이다.) 하지만, 이론수학에서는 숫자와 다른 종류의 양 사이의 구분은 명확하게 남았습니다.

클래식 그리스 문맥에서 뒤따른 서구 세계 에서 오랜 시간 동안 (약 만 년) 나타난 숫자에 대한 역사적 변화를 두 가지 주요 주제로 볼 수 있다 : 첫째는 고대 그리스 사람들이 여러 형태의 양들을 분리하여 나누었는데 천천히 허무해졌다는 점이며;둘째 는 이것을 위해서는 반복적으로 '숫자' 개념 자체가 일반화되어야 한다라는 것입니다

3 자릿수의 가치

우리가 정수를 나타내는 시스템은 마지막으로 인도 반대륙 수학자들에게 돌아갑니다. 기원전 ( 아마도 오세기 이전) 어느 때인가에 그들은 하나에서 열까지의 숫을 지정하기 위해 아홉 개의 상징들을 만들었고, 그리고 이러한 상징들의 위치를 사용하여 실제 값을 표시했습니다. 따라서 단위 자릿수에는 있는 '3'이 세라는 것을 의미하고 일곱째 자리는 삼십이라고 합니다. 물론 우리가 여전히 하는 것입니다; 자신들만의 상표들이 바뀌었습니다. 하지만 원칙은 변하지 않았습니다. 거기에 비슷한 시간 동안 공백 없음을 알려주는 장소 주석자가 개발되었는데, 이것은 결국 우리의 영공(zero)로 진화되었습니다.

십분법 기호 체계는 인더 지방에 오랫동안 사용되어왔으며 이슬람 세계로 확산되었습니다. 아랍 제국의 새로운 도시 바그달에서 신세기 (9세기 후반), Al Khwarizmi [VI.5] 가 '아홉 개의 상징' 라는 형태로 쓰여진 인디언 스타일 명칭 논리 서적을 저술했습니다. 그의 저서는 여러 세월 동안 라틴어로 번역되었고, 중세 시대 유럽 지역에서는 매우 인기를 얻었기에 자주 ‘algorism’라고 불렸던 것입니다.

Al Khwarizmi 작품은 여전히 영공(zero) 에 특별한 위치를 부여했는데; 그것은 공간만 채우는 것이 아니라 실제 값입니다. 하지만 우리가 이미 문자와 같은 방식으로 계산하기 시작하면 차이는 점차 사라집니다. 다자릿수 곱셈에는 zero 와 더해서 합치거나 빼야 합니다. 이러한 과정 속에서 "무엇도 없음" 은 천천히 수학 표현체계 안으로 들어갔습니다.

그리스 문화의 다른 영향력 대신 실용적인 전통이 중요해졌다는 것을 알 수 있습니다. 이것 또한 al-Khwarizmi 의 또 다른 유명한 책 제목에서 볼 수 있습니다.


\begin{summarybox}
소수 자릿값은 기하학적 문제를 해결하는 데 중요한 개념이며, 그리스 사람들은 비율을 사용하여 이 문제를 접근했습니다. $\pi$는 원주와 반지름의 비율이 원 주변 길이와 지름 사이의 비율과 동일하다고 정의되었습니다. 아르키메데스는 $22/7$보다 약간 작으며, $223/71$보다 조금 더 큰 $\pi$의 추정값을 증명했습니다. 그리스 사람들은 이론적이고 철학적으로 만족스러운 방법으로 다양한 유형의 값들을 구성할 수 있었습니다.
\end{summarybox}
\vspace{8pt}

\begin{examplebox}[예시 1]
$345$
\begin{itemize}
\item 이 숫자는 세 자릿수의 가치를 보여줍니다.
\item '3'은 백의 자리, '4'는 십의 자리, '5'는 일의자리에 위치하고 있습니다.
\end{itemize}
\end{examplebox}
\vspace{4pt}

\begin{examplebox}[예시 2]
$\frac{1}{7} = 0.\overline{142857}$
\begin{itemize}
\item 이 소수 표현은 반복되는 패턴을 보여줍니다. 각 자리는 서로 다른 값을 가지지만, 전체적으로 비율적인 관계를 유지하고 있습니다.
\end{itemize}
\end{examplebox}
\vspace{4pt}

\begin{examplebox}[예시 3]
$\pi \approx 3.14159$
\begin{itemize}
\item $\pi$는 원주와 지름의 비율로 정의되며, 이 값은 무한히 길고 반복되지 않는 소수입니다. 하지만 실제 계산에서는 이를 근사하여 사용합니다.
\end{itemize}
\end{examplebox}
\vspace{4pt}

\begin{exercisebox}
\textbf{1.} 문제 1 (기초):
372를 단위, 십의자리, 백의자리를 각각 쓰세요. 

문제 2 (중급):
548과 963 사이에 있는 가장 큰 수는 무엇인가요?\\[4pt]

\end{exercisebox}
\vspace{4pt}

\begin{glossarybox}
\textbf{Ratio} --- 비율 (ratio) \\
\textbf{Geometry} --- 기하학 (geometry) \\
\textbf{Eudoxus's theory of proportion} --- 유행의 비례 이론 (Eudoxus’s theory of proportion) \\
\textbf{Magnitude} --- 크기 (magnitude) \\
\textbf{Segment} --- 세그먼트 (segment) \\
\textbf{Angle} --- 각도 (angle) \\
\textbf{Surface} --- 표면 (surface)
\end{glossarybox}

\paragraph{풀이}
1. 문제 1 (기초):

 풀이: 
\begin{itemize}
\item 단위(个位): 2 
\item 십의자리(十位): 7 
\item 백의자리(百位): 3
\end{itemize}

2. 문제 2 (중급):

 풀이: 
 548과 963 사이에 있는 가장 큰 수는 962입니다.



% ═══ Section 4: Arab and Islamic Commentators ═══
\subsection{아랍과 이슬람 해설가들

Let me know if you have more to translate! I'm ready for your next mathematical challenge 😊}
\label{sec:4}

86 
Ⅱ. 현대 수학의 기원

아리스토텔레스와 많은 후세 해설가들이 개발한 가장 유명한 고대 그리스 우주 이론은 공간이 무한하며, 고정된 별들로 구성되어 있는 구체라고 가정했다는 점을 언급할 필요가 있습니다. 원리(Elements) 에서 다룬 수학적인 공간 역시 무한하므로, 모든 저자들에게 수학적 공간이 물질 세계를 단순히 이상화한 것으로 간과되지 않았다는 가능성은 배척될 수 없습니다.

4개의 기하학적 정리에 대한 아랍 및 이슬람 학자들의 해석

오늘날 우리가 '그리스 수학'으로 여기는 것은 소수의 수학자가 주요 내용을 집중하여 두 세기에 못미치는 시기를 완료했던 것입니다. 결국 더 큰 지역에서 오랫동안 지속되는 인구를 보유하고 있었던 아라비아와 이슬람 작가들이 계승했습니다. 이러한 저술가들은 일반적으로 그리스 수학 및 과학에 관한 논평가이며 후대로 전달하는 사람으로 알려져 있습니다. 하지만 스스로 창조적인 선도적인 수학자 그리고 과학자인 점 또한 명심해야 합니다. 많은 사람들이 유클리드 원론을 연구하며 평행 추측법 문제를 고민했습니다. 일부에서는 적절하지 않은 추측 일 것이라는 생각에서 시작했다고 말합니다. 다만 코어 자체를 사용해서 증명할 수 있다는 확신을 가지고 노력했습니다.

초선인 Th¯abit ibn Qurra 중 한 명입니다. 그는 알레포 근처 출생, 바그다드에서 살았으며 죽음지역이기도 하고 그의 사망 연도는 기원후 901년 입니다. 여기서는 처음 접근 방식 만 설명될 공간이 있습니다. 두 직선 m 와 n 에 세 번째 k 가 교차한다면 하나쪽에서는 서로 가까워지며 다른 쪽에서는 무제한적 으로 분산된다고 주장했습니다 .k 의 양옆에서 같은 대각 각들을 만들게 하는 두 개의 직선들은 절대 반복되지 않는다는 것을 보여줍니다 ( 그림 1에 표시된 각). 상황의 대칭성 때문에 어느 편에도 동반하여 이동해야 함을 의미하며 , 하지만 오히려 다른 부분에서 발산하는 것임을 이미 입증했기에 이러한 점을 통해 유클리드 평행 정리를 도출하였으나 그 논쟁 또한 결함이 있었습니다.

유럽과 아시아 지역 사이를 자주 왕래하면서 지방 수학자들과 소통하고 연구했다고 합니다.
명예로운 이슬람 학자 및 과학자인 ibn al-Haytham 은 기원 후 965 년 바스라 에서 태어났으며 기원 후 1041 년 이집트 에서 사망하였다. 그는 두 변 길이가 일정하게 되는 네모꼴을 가지고 한변으로부터 다른 변까지 수직 선을 내렸다. 그리고 하나의 원본 수평선을 다른 방향으로 움직일 때 끝점이 스크린 전체 영역을 채우도록 주장한다.

n k a b

A B B' A' D C

AB와 CD가 같고, 각 ADC는 직각이며, A′B′은 AB를 따라 CD쪽으로 이동하는 중간 위치이다. (그림 2 참조) . 이것은 모든 곳에서 직선과 등거리인 곡선이 스스로 직선임을 가정한 것에 해당하며 평행 추론이 쉽게 따르므로 그의 시도는 실패합니다. 그는 자신의 증명이 운동의 사용 때문에 오마르 카야암에게 심하게 비판받았는데, 그러나 그것이 유클리드 원저보다 기본적으로 불분명하고 외래적인 것이라고 생각했습니다. 실제로 유클리드가 도형학에서 한 적던 방법과 매우 다릅니다. 여기서는 얻어진 곡선의 본질이 명확하지 않기 때문입니다; 바로 분석해야 할 것입니다.

이스라엘 사람들의 평행추론에 대한 최후 노력은 나시르 알딘 알투사이의 업적입니다. 그는 1201년에 이란에서 태어나 1274년에 바그다드에서 사망했습니다. 그의 광범위한 논평 또한 우리가 이 주제를 조사하는 데 있어 더욱 풍부한 정보원 중 하나입니다. 알-두시아이는 두 선들이 시작하여 수렴하기 시작하면 결국 만날 때까지 계속해서 그렇게 하여야 한다고 보였습니다. 이 목표 달성을 위해 다음을 입증하려 합니다: (∗) l와 m이 직각 미만 각도를 만들면 모든 l에 수직인 선이 m에 교차한다는 것을 증명합니다.


\begin{summarybox}
아리스토텔레스의 고대 그리스 우주 이론과 유럽/아시아 학자들의 기하학적 연구를 중심으로, 아랍과 이슬람 해설가들은 유클리드 원리를 확장하고 새로운 수학적 개념을 도입했습니다. Th¯abit ibn Qurra와 같은 학자는 평행 추측법 문제에 대한 다양한 접근 방식을 제시했으며, 그들의 연구는 유럽과 아시아의 지방수학자들과 소통하며 발전하였습니다. 특히 naṣīr al-Dīn al-Tūsī는 두 선이 수렴하는 조건下的 평행 추측법에 대한 중요한 논평을 제시하였으며, 이는 유클리드 원리를 확장한 중요한 단계였습니다.
\end{summarybox}
\vspace{8pt}

\begin{center}
\resizebox{\textwidth}{!}{\begin{tikzpicture}[node distance=2cm, auto]
    \node (Thabit) [rectangle] {Thābit ibn Qurra};
    \node (IbnHaytham) [rectangle, below right of=Thabit] {ibn al-Haytham};
    \node (NasirAlDin) [rectangle, above left of=Thabit] {Naṣīr ad-Dīn Al-Ṭūsī};

    \draw[->] (Thabit) -- node[midway, below] {주요 기여} (IbnHaytham);
    \draw[->] (NasirAlDin) -- node[midway, above] {평행추론 연구} (IbnHaytham);

    \node (Euclid) [rectangle, left of=Thabit, xshift=-3cm] {Euclid};
    \draw[->] (Euclid) -- node[midway, below] {원리 Elements} (Thabit);
\end{tikzpicture}}
\end{center}
\vspace{8pt}

\begin{examplebox}[예시 1]
** 
아리스토텔레스의 우주 이론은 공간이 무한하며, 고정된 별들로 구성되어 있다고 가정했습니다. 이를 통해 그는 지구가 중심에 위치하고 모든 별들이 거기로부터 같은 방향으로 떨어져 있는 것으로 보았습니다.

**
\end{examplebox}
\vspace{4pt}

\begin{examplebox}[예시 2]
** 
Th¯abit ibn Qurra의 접근방식은 두 직선 m와 n이 교차하는 세 번째 k를 고려했습니다. 그는 한쪽에서는 서로 가까워지며 다른 쪽에서는 무제한으로 분산된다고 주장했습니다. 이 아이디어가 유클리드의 평행 정리를 도출하는데 사용되었습니다.

**
\end{examplebox}
\vspace{4pt}

\begin{examplebox}[예시 3]
** 
ibn al-Haytham은 두 변 길이가 일정하게 되는 네모꼴을 가지고 한변으로부터 다른 변까지 수직 선을 내렸습니다. 그는 원본 수평선을 다른 방향으로 움직일 때 끝점이 스크린 전체 영역을 채우도록 주장했습니다. 그러나 이 시도는 실패하였으며, 유클리드의 방법과 매우 다릅니다.
\end{examplebox}
\vspace{4pt}

\begin{exercisebox}
\textbf{1.} 문제 1 (기초):
아리스토텔레스의 우주 이론에서 공간은 무한하다고 가정했습니다. 그는 어떤 다른 특성을 가지는지 언급하지 않았나요?

문제 2 (중급):
Th¯abit ibn Qurra의 접근 방식에 따르면, 두 직선이 서로 교차할 때 한쪽에서는 서로가까워지고 다른 쪽에서는 무한으로 분산됩니다. 이 개념은 어떤 중요한 원리와 관련되어 있나요?

문제 3 (중급):
ibn al-Haytham의 실패는 무엇 때문에 발생했을까요? 그의 증명이 유클리드 원저보다 외래적이고 불분명하다고 생각한 이유를 설명해주세요.\\[4pt]

\end{exercisebox}
\vspace{4pt}

\begin{glossarybox}
\textbf{Euclidean geometry} --- 유클리드 기하학 \\
\textbf{Aristotle} --- 아르테미스토斯 \\
\textbf{Arabic and Islamic writers} --- 아랍과 이슬람 작가들 \\
\textbf{Thabit ibn Qurra} --- 타히트 인부쿠라 \\
\textbf{Transversal} --- 대각선 \\
\textbf{Ibn al-Haytham} --- 아이반 알 하이탄姆
\end{glossarybox}

\paragraph{풀이}
1. 문제 1 (기초):
아리스토텔레스는 우주가 무한하다고 가정했지만, 그의 이론에서 공간에 대한 다른 특성을 언급하지 않았습니다. 예를 들어, 유럽과 아시아 사이에는 바다라는 개념이 있었던 반면, 아라비아에서는 지구와 하늘을 구분하는 경계가 없었다는 관점이 있었다.

2. 문제 2 (중급):
Th¯abit ibn Qurra의 접근 방식은 "극한"이라는 개념과 관련되어 있습니다. 이 개념은 한쪽에서 서로 가까워지고 다른 쪽에서는 무한으로 분산되는 현상을 설명하는데 사용됩니다. 이것은 극한을 통해 얻어지는 새로운 수를 도입하는 과정에 해당합니다.

3. 문제 3 (중급):
ibn al-Haytham의 실패는 그가 증명하려고 했던 원리와 유클리드의 원리를 비교했을 때, ibn al-Haytham이 사용한 방법이 외래적이고 불분명하다고 생각되었기 때문입니다. 이는 ibn al-Haytham이 자신의 증명에서 유클리드의 원리를 따르지 않았거나, 그가 사용하는 개념이나 기법에 대해 명확하게 이해하지 못했을 가능성이 있습니다.



% ═══ Section 5: The Western Revival of Interest ═══
\subsection{서부의 관심 부활

Let me know if you have another passage to translate! I'm ready for more mathematical challenges. 😊}
\label{sec:5}

2. 기하학

그는 만약 (*)가 사실이라면 평행선 정리가 성립한다고 보였다. 하지만 그의 주장에 결점이 있었다. 당시 수학자들이 사용할 수 있는 기술만으로 증명 과정 중 어디서 잘못되었는지 파악하기란 매우 어려웠다. 이스람 지역 학자들은 서쪽 세계 사람들에게도 뒤처지는 것이 없을 정도로 독창적이고 정교한 논리를 가지고 있었으며, 불행하게도 바티칸 도서관에서 간단히 인쇄된 작품(1594년) 외에는 오랫동안 서양 사회에서는 전해져 내려오지 않았다. 많은 시간 동안 al-Tusi 저작과 여기저기에 나온 아들의 작업일 가능성까지 추측되었다.

유럽 재흥 시대와 함께 그리스 수학 교재의 두 번째번째 해석 및 번역 활동이 시작되며 Commandino 와 Maurolico 가 선두를 달았는데, 인쇄술 발달은 더욱 빠르게 확산되는 데 큰 영향을 미쳤다. 여러 개의 고문헌 자료들을 통해 유클리드 원론 의 새로운 버전이 만들어졌다는 사실 또한 중요하다. 많은 책들에 "Euclid에게 남긴 얼룩" 라고 Henry Savile 에 의해서 언급된 평행선 문제가 다뤄진 것으로 알려져 있다. 예를 들어 Elements 를 편집하고 수정하여 사용하는 강력한 예수회 신자 Christopher Clavius 는 1574년에 평행선을 등거리가 같은 직선으로 정의할 것을 시도했다.

육각형 공간과 유클 리디언 기하학적 공간 동등시화는 코페르니쿠스 천문학 이 받아들이면서 고정천체라고 불리는 구 체계 철폐 후 16~17세기 동안 점차적으로 일어났으며 뉴턴 [VI.14] 은 그의 Principia Mathematica 에서 중력설을 제시하며 유클리드 공간에서 단단히 자리를 두었던 것처럼 공식화하였다. 뉴토 니언 물리학은 수용받는데 어려움을 겪었다 하더라도, 뉴토 니언 우주관은 부분적인 경로였지만 18세기에 도덕법에서는 가장 위협 없는 것이되었다. 모든 요소가 본질적으로 연결되어 있음이 주장될 때 그러한 지식 확인 과제들은 더욱 중요해졌고, 유클리드 원론 의 심오함 속에서 얻은 예상치 못하고 역경한 결론들을 통해 우리에게 거꾸러진 사실들을 알게 한다.

1663년 영국수학자 존 월리스 는 선행자인 누구보다도 평행 추측에 대한 매우 미묘한 관점을 가지고 있다. 그는 아랍어를 읽을 수 있는 할레이 로부터 바티칸 도서관 에 보존되는 al-Tusi의 비전적 작품 내용에 대해 안내 받았으며 자신의 증명 시도에도 참여했지만 다른 사람들과는 다르게 접근했다. 특별히 Wallis 은 그의 논쟁 오류 위치까지 파악하는 통찰력을 가졌으며 이것이 실제로 무엇을 보여준다고 언급했습니다: 기본 개념으로는 평행 추정 조건이 동일하지 않으면 같은 형태인 것들의 존재라는 명언처럼 이해해야 합니다.

반세기 후에는 가장 끈기를 발휘하여 그리고 완벽성있는 모든 평행추정조항 방위자가 된 Gerolamo Saccheri (이탈리아 예수회 신부)가 나타났는데, 그가 죽던 해였던 1733 년 Euclid Freed of Every Flaw 라는 제목의 소책자를 출판하였다. 유클리드 원론 의 심오함 속에서 얻은 결과들이 모두 사실이며 우리에게 거꾸러진 사실들을 알려주었다

첫 번째 삼분법입니다. 만약 평행추측 자체를 확신하기 어려우면 삼각형 각합계량은 두 직선보다 작거나 또는 크거나 될 수 있습니다. Saccheri 는 한 삼각형에서 발생하는 다양한 경우에도 같은 현상이 나타난다고 증명했습니다. 따라서 본질적으로 세 가지 기하 공간 (L), E, G) 과도 호환될 가능성이 있었다; 첫째, 모든 삼각형의 각 합 계산값이 두 직선으로 작게 나오도록 하는 것이 'L'이고, 둘째, 모든 삼각형의 각 합 계산값이 두 직선과 같아지는 것인 'E', 마지막으로 모든 삼각형의 각 합이 두 직선보다 큰 것은 'G'. 물론 Case E 는 일반적인 유클리디안 기하학이며 Saccheri 는 그것만 존재하는 형태임을 보여주려는 노력에 매달렸습니다. 그래서 그는 다른 케이스들을 독립적으로 파괴함을 입증하려고 작업을 시작했으며 그의 주장 중 가장 어렵게 여겨진 문제인 ‘Case’ 를 해결할 시도로 이어졌다.

Saccheri 의 연구 결과 점차 무시되기 시작했지만 완전히 소멸되지 않았습니다. 하지만 스위스 수학자 Johann Lambert 에 도달하며 자신의 개념적 접근 방식에 대한 조심스러운 분석은 사실 Lambert 와 달리 평행추측 증명 가능성에는 확신하지 않으면 포기를 결정한 후 죽음 이후 인쇄되었습니다. Lambert는 불쾌감이나 불가능하다고 생각되는 것을 명확히 구분해야 한다며 L 에서 삼각형의 영역이 두 직선과 삼각형 각합 계산치 사이 차와 비례한다는 논쟁 초상화를 가지고 있었습니다. 또한 알았습니다: 
L에서는 유사삼각형들이 서로 닮지 않는다는 것입니다.


\begin{summarybox}
서부의 수학자들은 평행선 정리가 성립한다고 주장했으나, 당시 기술로 증명 과정에서 오류를 파악하기 어려웠다. 유럽 재흥 시대와 함께 그리스 수학 교재 해석 및 번역 활동이 시작되며, 인쇄술 발달에 따라 유클리드 원론의 새로운 버전이 만들어졌고 평행선 문제도 다루졌다. 17세기에는 Gerolamo Saccheri가 Euclid Freed of Every Flaw이라는 소책자를 출판하여 여러 가지 기하 공간을 고려했으며, 그 결과는 유클리드 원론의 심오함 속에서 거꾸러진 사실들을 알려주었다.
\end{summarybox}
\vspace{8pt}

\begin{examplebox}[예시 1]
** 
유럽의 기하학적 재naissance는 그리스 수학자들의 교재들을 번역하고 해석하는 활동으로 시작되었습니다. 특히, Commandino와 Maurolico가 유럽에서 처음으로 그리스 원론을 번역한 것이 중요했습니다.

\begin{itemize}
\item 번역 전:
\end{itemize}
 \[
 \text{Euclid's Elements}
 \]

\begin{itemize}
\item 번역 후 (Commandino의 번역):
\end{itemize}
 \[
 \text{"Euclides Elementa"}
 \]

**
\end{examplebox}
\vspace{4pt}

\begin{examplebox}[예시 2]
** 
16세기 초, 유럽에서는 기하학적 공간 동등성에 대한 논란이 있었습니다. 특히 코페르니쿠스는 지구가 중심이 아닌 행성을 중심으로 돌고 있다는 이론을 제안했습니다.

\begin{itemize}
\item 코페르니쿠스의 천문학 모델:
\end{itemize}
 \[
 \text{지구 → 행성들}
 \]

**
\end{examplebox}
\vspace{4pt}

\begin{examplebox}[예시 3]
** 
17세기 초, Gerolamo Saccheri는 평행추측에 대한 논란을 이끌었습니다. 그는 세 가지 기하 공간(L, E, G)이 모두 유클리드 원론과 호환될 수 있다고 주장했습니다.

\begin{itemize}
\item Saccheri의 세 가지 경우:
\end{itemize}
 \[
 L : \text{모든 삼각형의 각 합 계산값이 두 직선으로 작게 나오도록 하는 것}
 \]
 \[
 E : \text{모든 삼각형의 각 합 계산값이 두 직선과 같아지는 것 (일반적인 유클리디안 기하학)}
 \]
 \[
 G : \text{모든 삼각형의 각 합계산값이 두 직선보다 큰 것}
 \]
\end{examplebox}
\vspace{4pt}

\begin{exercisebox}
\textbf{1.} 문제 1 (기초):
유럽에서의 기하학적 발전은 어떤 중요한 이벤트에 의해 시작되었는가?

문제 2 (중급): 
Gerolamo Saccheri의 연구를 통해 발견한 세 가지 가능한 기하 공간(L, E, G)을 설명하시오.\\[4pt]

\end{exercisebox}
\vspace{4pt}

\begin{glossarybox}
\textbf{Geometry} --- 기하학 (geometry) \\
\textbf{Parallel postulate} --- 병행公理 (parallel postulate) \\
\textbf{Euclid's Elements} --- 유클리드의 원론 (Euclid's Elements) \\
\textbf{Copernican astronomy} --- 코퍼니쿠스 천문학 (Copernican astronomy) \\
\textbf{Newtonian physics} --- 뉴턴물리학 (Newtonian physics)
\end{glossarybox}

\paragraph{풀이}
1. 문제 1 (기초):
유럽에서의 기하학적 발전은 아리스토텔레스의 지구 중심주의에 의해 시작되었습니다. 그러나 그들의 이론은 중세 후반부터 물리적인 증거와 관찰이 충돌하여 새로운 아이디어가 필요했습니다.

2. 문제 2 (중급):
Gerolamo Saccheri는 18세기 초 기하학 연구에서 중요한 발전을 거두었습니다. 그의 주요 발견은 세 가지 가능한 기하 공간(L, E, G)입니다:

\begin{itemize}
\item L: 유클리드적 평면(유럽 기하학)
\item E: 볼체 상의 구형 기하학
\item G: 하이퍼볼릭 기하학
\end{itemize}

Saccheri는 각각의 경우에서 직선과 원주 사이의 거리를 비교하여 이론을 검증했습니다. 이러한 연구들은 둥근 지구나 평면에 대한 새로운 이해를 제공하며, 이후 페르미니와 카우라트 등이 더 깊은 기하학적 발견들을 하게 되었습니다.



% ═══ Section 6: Real, False, Imaginary ═══
\subsection{실수, 거짓, 허수}
\label{sec:6}

2.1 숫자에서 수 체계 로

실수, 허수

스티븐은 글쓰기를 하던 동안에도 다음 과 같은 일들이 계속 진행되고 있었다. 방정식 이론 발달로 인해 음수와 복소수 사용이 유용 해졌다는 점이다. 스티븐 자신도 이미 음수를 알았지만 편안하지 않았는데 예를 들어 그는 −3 가 x² + x − 6 의 근임 것은 실제로 3 가 관련된 다항식 x² − x − 6 (모든 곳에서 x 를 -x 로 바꾸면) 의 근 임을 의미한다라고 설명했다.

이는 단순히 회피였지만 세차방정식들은 더 어려운 문제들을 야기했습니다. 16세기에 활동했던 여러 명상학자들의 연구 결과 세차방정식의 해결 방법이 개발되었으며 중요한 한 발걸음으로 제곱근 추출과 같는 절차가 포함되었다. 그러나 문제점은 필요한 루트가 때때로 부호가 음인 경우였다.

그 전까지 항등적인 문제가 마이의 정사각형근 계산으로 이끌렸더라면 단순히 해가 없는 것이었으나, 방정식 x³ = 15x + 4 는 확실히 해를 가지며 – 진짜 해 중 하나는 x=4 라고 한다– 오직 세차 공식 적용에 √−121 을 계산해야 하는 점 때문일뿐이다.

바롬벨리 [VI.8] 또한 수학자이자 엔지니어이며 도전의 불씨앗을 삼아 무엇이 일어날지를 보았다. 그의 알 gebra (1572년 출판) 에서 그는 "새로운 종류의 근" 과 함께 계산하고, 그렇게하여 세차방정식의 해를 찾을 수 있다는 것을 증명했다. 이는 세차 공식이 실제로 작동한다는 사실을 입증했으며 더 중요하게 새로운 기묘하지만 유익할 수 있는 수들이 존재함을 보여주었다.

많은 시간 동안 사람들은 이러한 새로운 양들에 대해 편안하기 어려웠지만 약오십 년 후 우리는 알버트 지르 ard 와 데카드스[ VI.11 ] 가 방정식에는 참(긍정적인), 거짓(부호가 음인), 허수라는 세 가지 종류의 근이 있음을 말하는 것을 발견합니다. 이는 현재 복소수라고 부르고 있는 것임을 완전히 이해하였느냐 확신되지는 않는다; 적어도 데카르테그는 때때로 n 차원 방정식은 n 개의 근을 가져야 한다며 진짜나 거지 않으면 단순히 떠올린다고 주장한 듯하다.

그러나 점진적으로 복소수 사용이 시작되었다. 방정식론에서, 음수 로그 논쟁과 삼각학 연관성 등 다양한 분야에서 나타났습니다. ( 함수를 통하여) 사인 및 코사인 함수와의 관계는 오일러 [VI.19] 에 의해 18세기 중반 강력한 도구로 변화했다. 18세기 중엽까지 모든 다항식이 복소 수에 대한 완전한 집합의 근을 가지도록 하는 것이 잘 알려져 있다. 이 결과는 대수 기본 정리[V.13] 라 불리고 가우스 [VI.26 ]가 만족스럽게 증명함으로써 명확하게 자리를 구축하였다. 따라서 방정식론에는 더 이상 수 개념을 확대할 필요 없는 상태였다.

복소수가 실수와 명백히 다른 특징을 지니고 있어 사람들 사이에서는 여러 종류로 분류되는 '숫자'의 개념이 등장하기 시작했다. 그러나 스테빈의 평등주의적 사상으로 인해서도 정수는 소수보다 우열하다는 생각이나 유리수보다는 무리수가 이해하기 어렵다고 여겨지는 편향적인 믿음들은 제대로 없애지 않았습니다. 19세기에는 새로운 아이디어들을 통해 기존에 가지고 있는 숫자들의 분류 방식에 대한 신중한 재검토 필요성이 부각되었습니다.

주요 연구인물 가우스와 쿠머[VI.40] 는 숫자론에서 복잡함 속에도 주목할 만한 집합들이 존재한다는 점을 보여줬으며, 예를 들어 모든 $a + b\sqrt{-1}$ 형태 (단 a 와 b 모두 정수일 때) 의 수를 포함하는 부분집합은 정수처럼 행동합니다. 갈루아 [VI.41] 는 해석 가능성 분석 관점에서 “유리”라고 불리는 일정 조건을 충족하는 수들의 기준점을 동일시해야 한다고 강조했습니다. 그는 아벨'[VI.33] 의 오차방정식 해결불능 정리를 언급하며 "rational" 은 사용된 계수로 표현되는 다항식 비율형으로 나타낼 수 있다는 것을 의미하고, 이러한 구성의 전체 세트가 일반적 산술 법칙을 따른다고 설명하였습니다. 
 
제18세기 요한 라mbert 는 e 과 π 가 무리수라는 사실을 증명했고 심지어 초월적인 특징을 가지도록 추측했습니다 - 어떤 다항식 방정식의 근이 아니었다는 것입니다. 당시에는 초월수 자체에 대한 개념도 명확하게 밝혀져 있지 않았지만 리우빌[VI.39]은 1844년 그런 종류의 수들이 존재함을 입증했습니다. 소수 년 안에 e 와 π 모두 초월수임을 증명하고 나중인 시대 캐터스 [VI.54] 에게 진행되면 현저히 많은 실수들이 초월수라고 보여주었습니다. 카탄토르 발견은 스테빈이 대중화 한 체계 내부에도 예상치 못하게 큰 심오성이 있음을 처음 드러냈다. 하지만 가장 중요한 개념 변화 중 하나는 해밀턴'[VI.37] 이 복소수를 사용하여 평면 조직하는 것을 관찰하면서 완전히 새로운 수 계산법 을 개발해낸 후(1843) 일 것이며, 이는 역사적으로 매우 의미있는 순간입니다.

실수 집합을 기반으로 한 평면기하학은 매우 단순하게 정립되었습니다. 그는 삼차원 공간을 파라미터로 표현하는 비슷한 방법을 찾으려 하였지만 불가능했습니다. 그래서 Hamilton은 사상자체인 quaternion [III.76] 을 만들었는데, 이는 수와 같이 행동하지만 중요한 차이점은 곱셈이 가환적이지 않다는 것입니다. 다시 말하면 q 와 q’ 가 quaternion이라고 할 때 qq’ 과 q’q는 일반적으로 다릅니다. Quaternion 은 최초의 “hypercomplex number” 시스템이며 등장과 함께 많은 새로운 질문들이 제기되었습니다. 다른 유사한 시스템들은 존재할까요? 어떤 것이 ‘숫자 체계’에 해당할까요? 특정 “숫자가” 가환 법칙을 만족시키지 않는 경우 우리는 다른 규칙을 위배하는 ‘숫자’를 만들어낼 수 있나요? 장기적인 관점에서 이 지식적 활발함은 수학자들에게 "숫자" 또는 "량" 의 모호한 개념에서 손절하고 대신 형태론적인 개념인 알 gebra 구조라는 것에 집중하도록 이끌었습니다. 결국 각각의 숫자 체계는 연산 가능한 항목들의 집합입니다. 매력적인 것은 사용하여 파라미터화 하거나 조직화 하는 데 도움이 되고 있습니다. 정수 (또는 라틴 문명 이름으로 integer) 예를 들어 세기를 공식화하며 실수는 선과 같은 공간을 나타내며 기하학의 토대가 된다.

20세기에 접어들었을 때 이미 잘 알려진 여러 가지 숫자 체계가 등장했습니다. Integer 는 가장 중요하게 여겨졌으며 그 뒤에는 유리수(즉 분수), 실수(Stevin's 소수점, 현재 철저히 명확해짐), 복소수로 구성된 계층구조가 따랐습니다. 복소수보다 더 일반적으로 quaternion 가 존재했지만 절대로 유일하지 않았던 시스템들이었다. Number theorists 는 complex number 에 속하는 자체 독립적 시스템으로 이해될 수 있는 algebraic numbers 의 다양한 필드와 작업했다. Galois 는 우리가 지금 finite fields라고 부르는 평범한 산술 법칙에 따라 행동하는 무한정 시스템들을 처음 제시했습니다. 함수 이론 학자들은 함수들의 집합에서 일하고 있었습니다; 물론 그것들을 'numbers' 로 생각하지 않았지만 "number system" 과의 비유성이 인지되고 활용되었습니다.

초기 20 세기를 거쳐 Kurt Hensel은 rational number 에서 시작하여 prime number p 를 특별한 역할을 하는 방식으로 사용해서 만들어낸 p -adic numbers [III.51]를 도입했습니다.(p 가 임의로 선택 가능하기 때문에 Hensel 은 사실상 무궁무진하게 많은 새로운 숫자 체계를 창출했습니다.) 이는 또한 “평균적인 산술법칙” 을 준수한다는 의미에서도 동작합니다.즉 addition과 multiplication은 예측되는 대로 작동하였습니다. 현대적으로 말하면 그것은 field였다. p-adics는 실제 또는 복소수와 명백히 관련 없는 것처럼 보여지는 ‘recognizably numbers’ 의 최초 시스템 제공했으며 (두 시스템 모두 유리 수를 포함했다) Ernst Steinitz에게 abstract theory of fields를 만드는데 기회가 주었다.

Steinitz 작업에 나타난 추상화 전환은 다른 부분들에도 일정부분 영향을 미쳤다. 가장 눈에 띄게 group 이론 및 그것들의 표현, algebraic numbers 이론이 있다. 모든 이러한 이론들이 noether[VI.76] 에 의해 개념적 통일성을 가지고 있는데 그의 프로그램이 "abstract algebra" 라 불렸던 것이다. 이것으로 인해서 'numbers' 는 완전히 뒤처지며 집합의 구조라는 것을 중점있어졌다

요즘 "숫자"라고 부르기엔 애매해지는 개념들이 많습니다. 전통적인 정수, 유리수, 실수 그리고 복소수와 같이 명백히 '숫자' 라고 할 수 있는 대상들 외에도 p-adics 또한 그 자체만으로 ‘숫자’라는 주제에서 논쟁을 불러일으킵니다. Quaternion 역시 일반적으로 '숫자' 로 여겨지지 않습니다. 하지만 특정 수학적 개념을 구성하는데 활용될 만큼 의미있는 위치를 차지하고 있습니다. 극단적인 예로 말하면 Cayley의 Octonion[III.76]처럼 매우 독창적인 구조들은 좌표계 형태로 나타나기도 합니다. 결론적으로 분석이나 체계화 과정에서 편리를 위해 사용되는 모든 것을 '숫자' 또는 관련된 용어로 받아 들입니다. 현재까지 해당 목적을 충족할 만한 기존 시스템이 없으면 새롭게 창출해야 하는 경우도 발생합니다.


\begin{summarybox}
실수와 허수가 등장하면서 수 체계가 확장되었다. 세차방정식의 해결 방법은 제곱근 추출 과정이 포함되었으나, 부호가 음인 경우 문제가 발생했다. 이로 인해 방정식론에서 새로운 종류의 근(실수, 거짓, 허수)들이 발견되었다. 이러한 개념들은 복소수와 같은 새로운 수 체계를 도입하게 되었으며, 이를 통해 기존 숫자들의 분류가 재검토되었다. 특히 카테고리론과 알 gebra 구조 등이 발전하여 "숫자" 또는 "량"의 모호한 개념에서 벗어나 형태론적인 관점으로 수학을 접근하게 되었다.
\end{summarybox}
\vspace{8pt}

\begin{center}
\resizebox{\textwidth}{!}{\begin{tikzpicture}[node distance=2cm, auto]
    \node (real) {실수};
    \node (imaginary) [right of=real] {허수};
    \node (complex) [below right of=real] {복소수};

    \draw[->] (real.east) -- node[midway, above] {$\subset$} (complex.west);
    \draw[->] (imaginary.south) -- node[midway, left] {$\subset$} (complex.north);

    \node (true) [below of=real] {참};
    \node (false) [right of=true] {거짓};

    \draw[->] (true.east) -- node[midway, above] {$\subset$} (imaginary.west);
    \draw[->] (false.south) -- node[midway, left] {$\subset$} (imaginary.north);

    \node (positive) [below of=true] {양수};
    \node (negative) [right of=positive] {음수};

    \draw[->] (positive.east) -- node[midway, above] {$\subset$} (real.west);
    \draw[->] (negative.south) -- node[midway, left] {$\subset$} (real.north);

    \node (rational) [below of=positive] {유리수};
    \node (irrational) [right of=rational] {무리수};

    \draw[->] (rational.east) -- node[midway, above] {$\subset$} (real.west);
    \draw[->] (irrational.south) -- node[midway, left] {$\subset$} (real.north);

    \node (integer) [below of=rational] {정수};
    \node (prime) [right of=integer] {소수};

    \draw[->] (integer.east) -- node[midway, above] {$\subset$} (rational.west);
    \draw[->] (prime.south) -- node[midway, left] {$\subset$} (rational.north);

    \node (natural) [below of=integer] {자연수};

    \draw[->] (natural.east) -- node[midway, above] {$\subset$} (integer.west);
\end{tikzpicture}}
\end{center}
\vspace{8pt}

\begin{examplebox}[예시 1]
** 
스티븐이 말하는 "음수"의 개념을 이해하기 위해 다음과 같은 예를 들어볼 수 있습니다.

\begin{itemize}
\item 방정식 \( x^2 + x - 6 = 0 \) 의 근은 \( x = 3 \) 또는 \( x = -2 \) 입니다. 여기서, \( x = -2 \)는 음수입니다.
\end{itemize}
 
**
\end{examplebox}
\vspace{4pt}

\begin{examplebox}[예시 2]
** 
세차방정식의 해결 방법에서 허수가 등장하는 예를 들어보겠습니다.

\begin{itemize}
\item 방정식 \( x^3 + 6x^2 + 11x + 6 = 0 \) 의 근 중 하나는 \( x = -1 \)이며, 다른 두 근은 복소수입니다. 이들은 \( x = -\frac{5}{2} + i\sqrt{\frac{7}{4}} \)와 \( x = -\frac{5}{2} - i\sqrt{\frac{7}{4}} \) 입니다.
\end{itemize}

**
\end{examplebox}
\vspace{4pt}

\begin{examplebox}[예시 3]
** 
복소수의 사용이 여러 분야에서 나타나는 예를 들어보겠습니다.

\begin{itemize}
\item 사인 및 코사인 함수와 관련된 복잡한 방정식을 해결하는 데 허수가 필요합니다. 예를 들어, \( e^{ix} = \cos(x) + i\sin(x) \) 이라는 오일러의 공식은 모든 실수 x에 대해 성립하며, 여기서 \( i \)는 허수입니다.
\end{itemize}

이렇게 복소수와 거짓이라는 개념들은 수학에서 중요한 역할을 하며, 다양한 분야에서 응용됩니다.
\end{examplebox}
\vspace{4pt}

\begin{exercisebox}
\textbf{1.} 문제 1 (기초):
\( x^2 + 4 = 0 \)의 해를 구하세요.

---

문제 2 (중급):
다음 방정식을 풀어보세요.
\[ x^3 - 6x^2 + 11x - 6 = 0 \]

---

문제 3 (중급):
\( z^2 + 4z + 5 = 0 \)의 해를 구하세요.\\[4pt]

\end{exercisebox}
\vspace{4pt}

\begin{glossarybox}
\textbf{Real} --- 실수 (실제 수) \\
\textbf{Imaginary} --- 허수 (허리수) \\
\textbf{Negative} --- 음의 (음수) \\
\textbf{Complex number} --- 복소수 (복합수) \\
\textbf{Square root} --- 제곱근 (제곱근)
\end{glossarybox}

\paragraph{풀이}
문제 1 (기초)
\[ x^2 + 4 = 0 \]의 해를 구하세요.

풀이 1: 
주어진 방정식은 \(x^2\)와 상수만 포함되어 있습니다. 이를 풀기 위해, 양변에 -4을 뺍니다.
\[
x^2 = -4
\]
제곱근을 취하면,
\[
x = \pm \sqrt{-4} = \pm 2i
\]
따라서, 해는 \( x = 2i \)와 \( x = -2i \)입니다.

문제 2 (중급)
\[ x^3 - 6x^2 + 11x - 6 = 0 \]을 풀어보세요.

풀이 2: 
주어진 방정식은 세차수의 다항식으로, 직접적으로 푸는 것이 어려울 수 있습니다. 그러나, 이 문제에는 실근이 하나씩 존재하므로, 루트를 찾아 나갑니다.
1. 실근 찾기:
 \(x = 2\)가 실근인지 확인합니다.
 \[
 (2)^3 - 6(2)^2 + 11(2) - 6 = 8 - 24 + 22 - 6 = 0
 \]
 따라서, \(x = 2\)는 실근입니다.

2. 다항식 나누기:
 \(x^3 - 6x^2 + 11x - 6\)을 \(x-2\)로 나눕니다.
 
 | | \(x^2\)-4\(x\)+5 |
 |-------|----------------|
 | \(x-2\) | |

 계산 과정은 다음과 같습니다:
 \[
 x^3 - 6x^2 + 11x - 6 = (x-2)(x^2-4x+3)
 \]
 
3. 다항식 분해:
 \(x^2 - 4x + 3\)을 풀면,
 \[
 x^2 - 4x + 3 = (x-1)(x-3)
 \]

따라서, 원래의 방정식은
\[
(x-2)(x-1)(x-3) = 0
\]
이므로, 해는 \(x = 2\)와 \(x = 1\) 그리고 \(x = 3\)입니다.

문제 3 (중급)
\[ z^2 + 4z + 5 = 0 \]의 해를 구하세요.

풀이: 
주어진 방정식은 복소수 계산에 초점을 맞추고 있습니다. 이를 풀기 위해, 완전제곱을 사용합니다.
1. 완전제곱 변환: 
 \(z^2 + 4z\)를 \((z+2)^2 - 4\)로 변환하면,
 \[
 z^2 + 4z + 5 = (z+2)^2 - 4 + 5
 \]
 
2. 계산:
 \[
 (z+2)^2 + 1 = 0
 \]

3. 해 구하기: 
 양변에서 1을 뺍니다.
 \[
 (z+2)^2 = -1
 \]
 
4. 제곱근:
 제곱근을 취하면,
 \[
 z + 2 = \pm i\sqrt{1} = \pm i
 \]

5. 결과: 
 따라서, 해는 \(z = -2+i\)와 \(z = -2-i\)입니다.

이렇게 각 문제를 간결하게 풀었습니다.


\end{document}