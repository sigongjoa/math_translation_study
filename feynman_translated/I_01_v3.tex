\documentclass[11pt, a4paper, openany]{book}

% ─── 여백 ───
\usepackage[
  top=28mm,
  bottom=25mm,
  inner=25mm,
  outer=30mm,
  marginparwidth=0mm,
  headheight=14pt,
  headsep=14pt
]{geometry}

% ─── 한글 ───
\usepackage{kotex}
\usepackage{fontspec}

% ─── 폰트 ───
\setmainfont{Noto Serif CJK KR}[
  UprightFont={Noto Serif CJK KR},
  BoldFont={Noto Serif CJK KR Bold},
  Ligatures=TeX,
]
\setsansfont{Noto Sans CJK KR}[
  UprightFont={Noto Sans CJK KR},
  BoldFont={Noto Sans CJK KR Bold},
  Ligatures=TeX,
]
\setmonofont{Noto Sans Mono CJK KR}[Scale=0.85]

% ─── 수식 ───
\usepackage{amsmath, amssymb, amsthm}

% ─── 그래픽 ───
\usepackage{graphicx}
\usepackage{tikz}
\usetikzlibrary{arrows.meta, positioning, shapes, calc, decorations.pathreplacing, decorations.markings, patterns, shadows, backgrounds}

% ─── 색상 팔레트 ───
\usepackage{xcolor}
\definecolor{feynred}{HTML}{C0392B}
\definecolor{feynblue}{HTML}{2471A3}
\definecolor{feyndark}{HTML}{1B2631}
\definecolor{feynnote}{HTML}{F0E6D3}
\definecolor{feynlight}{HTML}{EBF5FB}
\definecolor{feyngreen}{HTML}{1E8449}
\definecolor{feyngray}{HTML}{5D6D7E}
\definecolor{feynwarm}{HTML}{FDF2E9}
\definecolor{feyndeep}{HTML}{154360}

% ─── 줄바꿈 ───
\XeTeXlinebreaklocale "ko"
\XeTeXlinebreakskip 0pt plus 3pt
\emergencystretch 5em
\tolerance=2000
\hyphenpenalty=50
\setlength{\hfuzz}{2pt}

% ─── 행간 ───
\linespread{1.65}
\setlength{\parindent}{1.2em}
\setlength{\parskip}{3pt plus 1pt}

% ─── 박스 디자인 (tcolorbox) ───
\usepackage[most]{tcolorbox}

% ◆ 파인만의 말
\newtcolorbox{feynmansays}[1][]{
  enhanced, colback=feynwarm, colframe=feynred!60!black,
  coltitle=white, fonttitle=\sffamily\bfseries\small,
  title={\raisebox{-1pt}{\textbf{!}}~파인만이 강조합니다},
  sharp corners, boxrule=0pt, leftrule=4pt, breakable,
  left=10pt, right=10pt, top=8pt, bottom=8pt,
  shadow={1pt}{-1pt}{0pt}{black!15}, #1
}

% ◆ 역주
\newtcolorbox{translatornote}[1][]{
  enhanced, colback=feynnote, colframe=feyngray!40,
  fonttitle=\sffamily\bfseries\small,
  title={\raisebox{-0.5pt}{\textsf{*}} 역주},
  sharp corners, boxrule=0pt, leftrule=3pt, breakable,
  left=10pt, right=10pt, top=6pt, bottom=6pt,
  fontupper=\small, #1
}

\newtcolorbox{deepresearch}[1][]{
  enhanced, colback=feynlight, colframe=feynblue!70,
  coltitle=white, fonttitle=\sffamily\bfseries\small,
  title={#1}, attach boxed title to top left={yshift=-2mm, xshift=4mm},
  boxed title style={colback=feynblue!80, sharp corners, boxrule=0pt},
  sharp corners, boxrule=0.6pt, breakable,
  left=10pt, right=10pt, top=10pt, bottom=8pt,
  shadow={1.5pt}{-1.5pt}{0pt}{feynblue!15}
}

% ◆ 수식 하이라이트
\newtcolorbox{mathbox}{
  enhanced, colback=feynwarm!50, colframe=feynred!40,
  sharp corners, boxrule=0.5pt,
  left=14pt, right=14pt, top=10pt, bottom=10pt,
  before skip=12pt, after skip=12pt
}

% ◆ 핵심 개념 박스
\newtcolorbox{keyconcept}[1][]{
  enhanced, colback=white, colframe=feyndark, coltitle=white, fonttitle=\sffamily\bfseries,
  title={#1},
  attach boxed title to top center={yshift=-3mm},
  boxed title style={colback=feyndark, sharp corners, boxrule=0pt},
  sharp corners, boxrule=1pt, breakable,
  left=12pt, right=12pt, top=12pt, bottom=10pt
}

\newtcolorbox{diagrambox}[1][]{
  enhanced, colback=white, colframe=feyngray!50,
  fonttitle=\sffamily\small, title={#1},
  attach boxed title to top center={yshift=-2mm},
  boxed title style={colback=feyngray!20, colframe=feyngray!50, boxrule=0.4pt, sharp corners},
  sharp corners, boxrule=0.4pt, breakable,
  left=8pt, right=8pt, top=10pt, bottom=8pt
}

% ◆ 연습 문제 / 생각 해봅시다
\newtcolorbox{exercisebox}{
  enhanced, colback=feyngreen!5, colframe=feyngreen!60,
  fonttitle=\sffamily\bfseries\small,
  title={\textsf{?} 생각해 봅시다},
  sharp corners, boxrule=0pt, leftrule=4pt, breakable,
  left=10pt, right=10pt, top=8pt, bottom=8pt
}

% ─── 헤더/푸터 ───
\usepackage{fancyhdr}
\pagestyle{fancy}
\fancyhf{}
\fancyhead[LE]{{\small\sffamily\color{feyngray} 파인만 물리학 강의 \hfill \thepage}}
\fancyhead[RO]{{\small\sffamily\color{feyngray} \thepage \hfill \rightmark}}
\renewcommand{\headrulewidth}{0.4pt}
\renewcommand{\headrule}{\color{feyngray!40}\hrule width\headwidth height\headrulewidth}
\renewcommand{\footrulewidth}{0pt}
\fancypagestyle{plain}{\fancyhf{}\fancyfoot[C]{\small\color{feyngray}\thepage}\renewcommand{\headrulewidth}{0pt}}

% ─── 제목 스타일 ───
\usepackage{titlesec}
\titleformat{\chapter}[display]
  {\normalfont}{\hfill{\fontsize{72}{72}\selectfont\sffamily\color{feynred!20}\thechapter}}{-20pt}
  {\sffamily\bfseries\Huge\color{feyndark}}[\vspace{2pt}{\color{feyngray!40}\titlerule[1pt]}]
\titlespacing*{\chapter}{0pt}{-30pt}{30pt}

\titleformat{\section}[hang]
  {\sffamily\bfseries\LARGE\color{feyndeep}}
  {\thesection}{10pt}{}
\titlespacing*{\section}{0pt}{24pt}{10pt}
\titleformat{\subsection}[hang]{\sffamily\bfseries\large\color{feyndark}}{\thesubsection}{8pt}{}
\titlespacing*{\subsection}{0pt}{16pt}{6pt}

\usepackage{lettrine}
\usepackage{epigraph}
\setlength{\epigraphwidth}{0.75\textwidth}
\renewcommand{\epigraphflush}{center}
\renewcommand{\epigraphrule}{0pt}

\usepackage{hyperref}
\hypersetup{colorlinks=true, linkcolor=feynblue, urlcolor=feynblue!70, pdfborder={0 0 0}, bookmarksnumbered=true, pdftitle={파인만 물리학 강의 — 한국어 번역}, pdfauthor={AI 보조 번역 시스템}}

\setcounter{tocdepth}{2}
\usepackage{enumitem}
\setlist[itemize]{leftmargin=1.5em, itemsep=2pt}
\setlist[enumerate]{leftmargin=1.5em, itemsep=2pt}

\begin{document}


% ─────────────────────────────────────────
%  표지
% ─────────────────────────────────────────
\begin{titlepage}
\begin{tikzpicture}[remember picture, overlay]
  \fill[feyndark] (current page.south west) rectangle (current page.north east);
  \foreach \x/\y/\r/\o in {3/20/2.5/8, 15/24/1.8/6, 8/5/3/5, 17/8/2/7, 12/15/1.5/4} {
    \draw[feynred!\o 0, line width=0.8pt] (\x, \y) circle (\r);
  }
  \begin{scope}[shift={(11,12)}, scale=1.5]
    \draw[white!40, line width=1.2pt, decorate, decoration={snake, amplitude=3pt, segment length=8pt}]
      (-2,0) -- (0,0);
    \draw[white!40, line width=1.2pt] (0,0) -- (1.5,1.2);
    \draw[white!40, line width=1.2pt] (0,0) -- (1.5,-1.2);
    \filldraw[white!50] (0,0) circle (3pt);
    \draw[white!40, line width=1.2pt, -Stealth] (1.5,1.2) -- (3,2);
    \draw[white!40, line width=1.2pt, decorate, decoration={snake, amplitude=2pt, segment length=6pt}]
      (1.5,-1.2) -- (3,-0.5);
  \end{scope}
  \node[anchor=west] at (2.5, 18) {
    \begin{minipage}{14cm}
      {\fontsize{14}{18}\selectfont\sffamily\color{feynred!80}\bfseries
        THE FEYNMAN LECTURES ON PHYSICS}
    \end{minipage}
  };
  \node[anchor=west] at (2.5, 15.5) {
    \begin{minipage}{14cm}
      {\fontsize{36}{42}\selectfont\sffamily\color{white}\bfseries
        파인만 물리학 강의}\\[8pt]
      {\fontsize{16}{20}\selectfont\sffamily\color{white!70}
        제1권: 역학, 복사, 열}
    \end{minipage}
  };
  \node[anchor=west] at (2.5, 8) {
    \begin{minipage}{14cm}
      {\Large\sffamily\color{white!80} Richard P. Feynman}\\[4pt]
      {\normalsize\sffamily\color{white!50} Robert B. Leighton \quad Matthew Sands}\\[16pt]
      {\small\sffamily\color{feynred!60} 한국어 번역 · AI 보조 번역 시스템}
    \end{minipage}
  };
  \node[anchor=south] at (current page.south) [yshift=15mm] {
    {\small\sffamily\color{white!30} 개인 학습용 · 비배포}
  };
\end{tikzpicture}
\end{titlepage}

\tableofcontents
\clearpage

\chapter{원자들이 움직이는 것~}
\label{ch:Ch1}
\vspace{-8pt}{\large\sffamily\color{feyngray} Atoms in Motion}
\vspace{12pt}
\epigraph{\itshape ``If, in some cataclysm, all of scientific knowledge were to be destroyed,
and only one sentence passed on to the next generations of creatures,
what statement would contain the most information in the fewest words? I
believe it is the atomic hypothesis (or the atomic fact ,
or whatever you wish to call it) that all things are made of
atoms—little particles that move around in perpetual motion,
attracting each other when they are a little distance apart, but
repelling upon being squeezed into one another . In that one sentence,
you will see, there is an enormous amount of information about
the world, if just a little imagination and thinking are applied.''}{--- \textup{Richard P. Feynman}}
\vspace{8pt}
\section{오늘의 수업은 물리학의 기본 개념에 대해 살펴볼 거예요. 준비되었으셨나요? 시작해볼게요.}
\lettrine[lines=2, loversize=0.15, nindent=0.5em]{\color{feynred}\textsf{이}}{ 강의는 물리학을 두 년 동안 공부하는 것입니다. 이 강의는 여러분이 물리학자라는 전제에서 시작됩니다. 물론, 모든 사람이 이렇게 생각하는 것은 아니지만, 모든 과목의 교수들은 그렇게 생각합니다! 만약 여러분이 물리학자가 될 거라면, 많은 것을 배워야 할 거예요: 가장 빠르게 발전하고 있는 지식의 분야 중 하나인 200년간의 지식입니다. 사실, 이렇게 많은 지식을 4 년 동안 다 배울 수는 없어요; 대신 박사 과정에 가셔야 할 거예요!}


놀랍게도, 이 모든 시간 동안 탁월한 노력을 투자했음에도 불구하고 결과의 무대를 대략적으로 요약하는 데는 상당히 많은 부분이 가능해요. 즉, 우리 지식을 요약할 수 있는 법칙들을 찾을 수 있어요. 그럼에도 불구하고, 이러한 법칙들은 이해하기가 너무 어려워서 과학의 일부 분야와 다른 부분 사이의 관계에 대한 맵이나 개요를 제공하지 않고 이 무대를 탐구하는 것은 불공평해요. 따라서 이 전제를 설명한 후 첫 번째 세 장에서는 물리학이 과학의 나머지 부분과 어떻게 관련되어 있는지, 과학들이 서로 어떻게 관련되어 있는지 그리고 과학의 의미가 무엇인지 설명하여 우리에게 주제에 대한 "감각"을 발전시켜줄 것입니다.


물론이죠! 물리학을 가르치는 데에는 단순히 첫 페이지에 기본 법칙을 제시하고 모든 가능한 상황에서 어떻게 작동하는지 보여주는 것만으로는 충분하지 않아요. 우리가 유클리드 기하학에서처럼 공리를 제시한 후 모든 종류의 추론을 해보는 것과 비슷해요. (즉, 4년 동안 물리학을 배우고 싶은데 4분이면 어떻게 배울 수 있을까?) 이런 방법으로 할 수 없는 이유는 두 가지가 있어요. 첫째, 아직 모든 기본 법칙을 모른다고 해요. 우리 앞에는 무한히 확장되고 있는 무지의 경계가 있어요. 둘째, 물리학의 법칙은 매우 익숙하지 않은 아이디어들이 포함되어 있고, 이들을 설명하려면 고급 수학이 필요해요. 따라서 단순히 단어의 의미를 알아야 할 정도로 충분한 준비 훈련을 해야 하죠. 아니요, 그렇게 할 수는 없어요. 우리는 점점씩만 할 수 있어요.


\begin{feynmansays}
물론이죠! 자연의 모든 부분은 항상 완전한 진실에 대한 근사치일 뿐입니다. 사실, 우리가 알고 있는 모든 것이 단지 어떤 종류의 근사치일 뿐이죠. 왜냐하면 아직 모든 법칙을 알지 못하기 때문이죠. 따라서, 무언가를 배우는 것은 다시 배우고 싶어하거나, 더 가능성은 그걸 수정하는 것입니다.
\end{feynmansays}
과학의 원칙, 정의는 거의 다음과 같습니다. 과학적 지식의 테스트는 실험입니다. 실험은 과학적인 "진실"의 유일한 평가자입니다. 그러나 지식의 출처는 어디에 있나요? 검증해야 할 법칙들은 어디서 오나요?

실험 자체도 이러한 법칙들을 만들어 주지만, 그것은 단순히 힌트를 제공하는 것입니다. 또한, 이 힌트에서 큰 일반화를 만들기 위해 상상력이 필요합니다. 모든 것들 아래에 있는 멋진, 간단하지만 매우 이상적인 패턴을 추측하고, 그런 다음 다시 실험을 통해 우리의 추측이 맞았는지 확인해야 합니다. 이 상상 과정은 매우 어렵기 때문에 물리학에서는 분업이 이루어집니다: 이론물리학자는 상상, 유도, 새로운 법칙을 추측하며 실험하지 않습니다. 그리고 실험물리학자는 실험, 상상, 유도, 추측을 합니다.


물론이죠! 자연의 법칙은 근사적인 것이에요. 먼저 우리가 "틀린" 법칙을 찾아서 "맞는" 법칙을 찾게 되죠. 그럼 실험도 어떻게 "틀릴 수 있나요"? 첫째, 간단하게 말하면 실험 장비가 잘못된 거예요. 하지만 이런 것들은 쉽게 수정하고 확인할 수 있어요. 그래서 그런 작은 문제 때문에 실험 결과가 틀리지 않아요? 아니면 정확하지 않아요. 예를 들어 물체의 질량은 절대로 변하지 않는다고 느껴져요: 회전하는 둔근과 멈춰있는 둔근이 무게가 같아요. 그래서 "법칙"을 만들어놓았어요: 질량은 속도에 관계 없이 상수예요. 하지만 이 "법칙"은 이제 발견되지 않아요. 질량은 속도와 증가하며, 중요한 증가는 빛의 속도에 가까워야 해요. 진짜 법칙은: 물체가 100마일/초보다 적은 속도로 움직이면 질량은 백만분의 한 부분까지 상수예요. 이런 근사적인 형태에서는 이는 올바른 법칙이에요. 그래서 실제로 보면 새로운 법칙이 중요한 차이를 끼치지 않을 것 같아요. 그렇지만, 일반 속도에서는 확실히 무시하고 간단한 상수 질량 법칙을 좋은 근사로 사용할 수 있어요. 하지만 높은 속도에서는 틀리고 있고, 속도가 높아질수록 더 틀립니다.


마지막으로, 그리고 가장 흥미로운 것은 철학적으로 우리가 근사 법칙에 완전히 틀렸다는 점입니다. 세상의 전체적인 이미지는 질량이 아주 조금씩 바뀌더라도 변경되어야 합니다. 이는 법칙 뒤에 숨겨진 철학이나 아이디어가 매우 특별한 특징을 가지고 있다는 것입니다. 아주 작은 효과가 때때로 우리의 생각에 깊은 변화를 요구할 수 있습니다.


우선 가르쳐야 할 것이 무엇인가요? 정확하지만 익숙하지 않은 법칙과 그 특이하고 어려운 개념적인 아이디어를 가르치는 게 좋을까요? 예를 들어 상대성이론, 4차원 시공간 등등? 아니면 먼저 간단한 "질량이 변하지 않는" 법칙을 가르치는 게 좋을까요? 이 법칙은 약속된 것일 뿐이고, 그런 어려운 개념들이 포함되어 있지 않습니다. 첫 번째 방법은 흥미롭고 멋지며 즐겁게 보일 것입니다. 그러나 두 번째 방법은 처음에는 더 쉬워서 이해하기 쉽습니다. 그리고 이는 첫 번째 아이디어에 대한 진정한 이해의 첫걸음이 될 것입니다. 이런 점은 물리학 가르침에서 반복됩니다. 서로 다른 시점에서 다르게 해결해야 할 때가 있지만, 각 단계에서는 지금까지 알려진 것들을 배우고, 얼마나 정확한지, 어떻게 모든 것을 포함하며, 더 많이 배울 때 어떻게 변할 수 있는지를 알아보는 것이 중요합니다.


우리가 오늘 이해하는 과학의 전반적인 맵을 그려보도록 하겠습니다. 특히 물리학에 대해 이야기할 것이지만, 다른 주변 과학들도 포함될 것입니다. 이렇게 해서 나중에 특정한 지점을 집중할 때 배경지식이 무엇인지, 왜 그 지점이 흥미롭고, 어떻게 큰 구조에 맞는지를 이해하게 될 것입니다.

그런데, 우리 전체적인 세상의 이미지는 무엇인가요?


\section{물질은 원자로 이루어져 있어요.}
\begin{center}
\begin{diagrambox}{다음은 그림 1-1에요.}
\includegraphics[width=0.8\textwidth]{feynman_json/images/f01-01_tc_big.pdf}
\end{diagrambox}
\end{center}
원자 개념의 힘을 보여드리기 위해, 한쪽 면에 ¼인치 크기의 물 한 قطر을 생각해봅시다. 매우 가깝게 봐도 단지 물뿐입니다—부드럽고 연속된 물입니다. 가장 좋은 광학显미경으로 확대해도—약 2천 배 정도하면—물 قطر은 약 40피트(약 12미터) 크기로 될 것입니다. 큰 방과 비슷한 규모가 되겠죠. 매우 가깝게 봐도 여전히 부드러운 물이지만, 여기와 저기 작은 축구 모양의 것들이 움직이고 있습니다. 매우 흥미롭네요. 이들은 파라메틱입니다. 여기서 멈추시고 파라메틱의 꾸밈없는 복자와 회전하는 몸에 대한 궁금증이 생길 수 있습니다. 더 이상 진행하지 않는 것이 좋지만, 파라메틱을 더욱 확대해보면 안쪽을 볼 수도 있을 것입니다. 이는 당연히 생물학의 주제입니다. 그러나 지금은 물자 자체를 더욱 가깝게 살펴봅시다. 다시 2천 배로 확대하면 이제 물 قطر은 약 15마일(약 24킬로미터) 크기로 될 것입니다. 매우 가깝게 봐도 부드럽지 않은 모습을 보입니다—축구 경기에서 멀리서 본 사람들의 모임처럼 보일 것입니다. 이 흩어진 현상을 이해하려면, 다시 250배로 확대하면 Fig. 1–1과 비슷한 것을 볼 수 있습니다. 이것은 물이 불리언(10억 배)으로 확대된 사진입니다만, 여러 가지 방식으로 이상화되었습니다. 먼저 입자는 단순한 모양으로 그리고 있으며, 간격이 정확하지 않습니다. 두 번째로, 간단함을 위해 거의 스키마틱하게 2차원 배열로 그려졌지만, 사실은 세차원에서 움직이고 있습니다. 여기서는 산소(검정색)와 수소(하얀색)의 원자들을 표현하기 위해 두 가지 종류의 "블롭" 또는 원을 사용하고 있으며, 각 산소에는 두 개의 수소가 묶여 있습니다. 즉, 한 개의 산소와 그 둘째 수소를 가진 작은 그룹이 분자가 됩니다. 사진은 자연에서 실제로 있는 입자를 더 이상적으로 표현했습니다. 물체는 항상 흔들리고 충돌하며 회전하고 휘어집니다. 이는 정적인 사진보다 동적인 이미지로 생각해보세요. 그리고 그림에 표현할 수 없는 중요한 점이 있습니다. 입자가 서로 "끼워져 있다는 것입니다"—즉, 서로에 끌려 있고, 이 하나가 저 하나를 끌고 등등입니다. 전체 그룹은 "틀어놓여 있다고" 말할 수 있습니다. 그러나 입자는 서로 통과하지 않습니다. 두 개의 입자를 너무 가깝게 휘어넣으시면 반대 방향으로推开됩니다.


원자는 반지름이 $1$ 또는 $2\times10^{-8}$ cm입니다. 이제 $10^{-8}$ cm은 아우스트로ーム(다른 이름으로도 불립니다)이라고 하므로, 그들은 반지름이 약 1 또는 2 아우스트로ーム(Å) 정도라고 말할 수 있습니다. 그들의 크기를 기억하는 또 다른 방법은 이죠: 사과를 지구의 크기로 확대하면, 사과 내부의 원자는 원래 사과와 거의 같은 크기일 것입니다.


물론이죠! 이제 이 큰 물滴을 상상해봅시다. 여기에는 모든 입자가 서로 붙어 있고 함께 움직이고 있는 모습이죠. 물은 그 크기를 유지하고, 떨어지지 않아요. 이것은 분자 간의 서로 끌고 있는 힘 때문입니다. 만약 물滴이 경사면에 있다면, 이동할 수 있어요. 하지만 물은 사라지지 않고요—물체는 단순히 분리되지 않는 거예요. 그것은 분자 간의 서로 끌고 있는 힘 때문입니다.

이제 떨어지는 움직임이 열이라는 것을 나타내는 것입니다. 온도를 높일 때, 물체 내부에서의 움직임도 증가해요. 물을 더워지면 떨어짐이 증가하고, 원자 사이의 거리도 증가해요. 하지만 계속 끓여서는 안 돼요. 결국에는 분자가 서로 끌고 있는 힘보다 충분히 멀어져서 서로 분리하게 되죠. 물론 이는 우리가 물에서 기름을 만드는 방법입니다—온도를 높이는 거예요. 입자는 더 큰 움직임 때문에 분리되게 돼요.


\begin{center}
\begin{diagrambox}{그림 1-2}
\fbox{Missing vector conversion: f01-02\_tc\_big.svgz}
\end{diagrambox}
\end{center}
물론이죠! 오늘의 강의 내용은 물과 냄새에 대해 알아볼게요. 먼저, 그림 1-2를 보세요. 이 그림에서 냄새가 어떻게 생겼는지 보여주고 있어요.

그런데, 여기서 한 가지 문제가 있습니다. 일반적인 대기압에서는 이 그림에 물 분자가 많을 수 없어요. 대부분의 정사각형에는 물 분자가 하나도 없겠죠? 하지만 우리가 운이 좋게 두 개半이나 세 개가 그려져 있어요. 이제 냄새의 경우, 물보다는 더 명확하게 물 분자를 볼 수 있어요.

간단히 말하자면, 물 분자는 화학식으로 $H_2O$라고 쓰는데, 여기서 환원 원자와 산소 원자 사이에 $120^\circ$의 각이 있다고 그려져 있어요. 실제로는 $105^\circ3'$의 각이고, 환원 원자와 산소 원자의 중심 사이의 거리는 $0.957 \, \text{Å}$입니다. 이렇게 잘 알고 있는 물 분자를 보여주고 있어요.

이렇게 하면 이해하기 쉬워지겠죠? 물과 냄새에 대해 배우는 것은 재미있고 동시에 중요한 내용이에요. 궁금한 점이 있으면 언제든지 물어봐주세요!


물론이죠! 오늘 우리는 스팀 빛나기 또는 다른 모든 기체의 몇 가지 특성을 살펴볼게요. 분자들은 서로를 떨어져 있고 벽에 부딪히고 있습니다. 상상해봐요, 수백 개의 테니스 공이 영원한 운동을 하면서 방 안에서 뛰어다니는 장면을 생각해보세요. 이 테니스 공들이 벽에 충돌할 때마다 벽을 밀어내게 됩니다. (물론 우리는 벽을 다시 밀어넣어야겠죠.) 이것은 가스가 흔들리는 힘으로, 우리의 굵은 감각이 실제로는 평균적인 푸시로 느끼게 만듭니다. 가스를 고정하려면 압력을 가해야 합니다. 그림 1-3에 표준적인 기체를 보여드리겠습니다. 이 건축물은 모든 교재에서 사용하는 것인데, 실린더와 Piston이 있는 것입니다. 이제 물 분자의 형태가 무엇인지 중요하지 않습니다. 단순화하기 위해 테니스 공이나 작은 점으로 그려보도록 하겠습니다. 이들은 모든 방향에서 영원한 운동을 하고 있습니다. 이렇게 많은 테니스 공들이 항상 상단 Piston에 충돌하고 있으므로, 이 연속적인 충격 때문에 Tank에서 조용히 밀어내지 않으려면 특정한 힘을 가해야 합니다. 이것을 압력이라고 부릅니다(실제로는 압력과 면적의 곱이 힘입니다). 분명히 힘은 면적에 비례합니다. 만약 우리가 면적이 증가시키지만, 센티미터 당 분자 수를 유지한다면, Piston과의 충돌 횟수가 면적을 증가시킨 만큼 동일하게 증가할 것입니다.


\begin{center}
\begin{diagrambox}{그림 1-3}
\includegraphics[width=0.8\textwidth]{feynman_json/images/f01-03_tc_big.pdf}
\end{diagrambox}
\end{center}
이제 이 탱크에 두 배의 분자를 넣어 밀도를 두배로 하고, 그들의 속도는 동일하게, 즉 온도가 동일하게 유지해 보세요. 그러면 충돌의 수는 근사적으로 두배가 되고, 각각이 전에 비해 동일한 "에너지"를 발산하므로 압력은 밀도와 비례한다고 할 수 있습니다. 만약 원자 간의 진짜적인 힘의 본질을 고려한다면, 원자의 서로의 끌임으로 인해 약간의 압력 감소와 그들의 유한한 부피로 인해 약간의 압력 증가를 기대할 수 있습니다. 그럼에도 불구하고, 충분히 낮은 밀도에서 원자가 많지 않다면 압력은 밀도와 비례한다고 근사적으로 볼 수 있습니다.


원자기체의 밀도를 변화시키지 않고 온도를 높이는 경우, 즉 원자의 속도를 증가시킬 때 일어날 일은 무엇인가요? 원자는 더 빠르게 움직이기 때문에 더 강하게 충돌하며, 더 자주 충돌하기도 하므로 압력이 증가합니다. 그런가요? 원자 이론의 개념이 얼마나 간단한지 어떻게 이해하는지 보여드릴 수 있어요.


다른 상황을 생각해봅시다. 페널티스가 내부로 움직일 때, 원자는 점차 더 작은 공간으로 압축됩니다. 원자가 움직이는 페널티스에 부딪힐 때 일어나는 일이 무엇인가요? 당연히 충돌로부터 속도를 빨라집니다. 예를 들어, 전진하는 패들에서 파이핑볼을 튕기는 것으로 실험해보세요. 그러면 그게 충돌할 때보다 더 빠르게 나갈 것입니다. (특수한 예: 만약 원자가 정지 상태이고 페널티스가 그것을 부딪히면, 확실히 움직일 것입니다.) 따라서 원자는 페널티스를 부딪히기 전에 있었던 것보다 더 "뜨겁게" 나갑니다. 그래서 그들이 페널티스에서 벗어날 때 모든 원자가 속도를 빨라집니다. 이는 가스를 점진적으로 압축할 때 가스의 온도가 상승한다는 것을 의미합니다. 따라서, 점진적인 압축에서는 가스의 온도가 증가하고, 점진적인 확장에서는 온도가 감소하게 됩니다.


\begin{center}
\begin{diagrambox}{그림 1-4}
\fbox{Missing vector conversion: f01-04\_tc\_big.svgz}
\end{diagrambox}
\end{center}
물론이죠. 이제 우리는 물滴으로 돌아와서 다른 방향을 보겠습니다.
물의 온도를 낮추면, 물滴 내 원자들의 분자가 점점 더 잘라지고 있다고 가정해봅시다. 우리는 원자들 사이에 서로에 대한 끌임력이 있다는 것을 알고 있으므로, 시간이 지나면서 물滴은 더 잘라지지 못할 것입니다.
매우 낮은 온도에서는 Figure 1–4에서 보여주는 것처럼 분자가 새로운 패턴으로 잠금됩니다. 이 패턴은 얼음입니다. 그러나 이 얼음의 스키마틱 그림은 잘못된 부분이 있는데, 이는 두 차원에 있기 때문입니다. 그러나 정량적으로는 맞습니다. 중요한 점은 물질이 각 원자가 확정된 위치를 가지며, 만약 어떤 방법으로든 물滴의 한쪽 끝에서 특정 배열로 모든 원자를 고정한다면, 구조가 내구적인 경우 다른 끝이 마일리 정도 떨어져도 정확한 위치를 유지한다는 것입니다. 따라서 얼음의 한쪽 끝에 나비를 잡으면, 다른 끝은 우리에게 그림자처럼 밀려들지 않게 됩니다. 그러나 물의 경우 구조가 깨지고 원자가 서로 다른 방식으로 움직이기 때문에 그렇습니다. 고체와 액체 간의 차이는 고체에서 원자는 어떤 종류의 배열, 즉 결정된 배열로 정렬되어 있으며, 이들은 긴 거리에서 무작위한 위치를 가지지 않는다는 것입니다. 결정된 배열 한쪽에 있는 원자의 위치는 결정된 배열 다른 쪽에 있는 수백만 개의 원자와 관련이 있습니다. Figure 1–4는 얼음의 잘못된 배열일 뿐입니다. 그러나 이 배열에는 얼음의 많은 올바른 특징이 포함되어 있습니다. 정확한 특징 중 하나는 여섯각형 대칭이라는 것입니다. 만약 그림을 원통축을 중심으로 $60^\circ$로 돌리면 그림은 다시 돌아옵니다. 따라서 얼음에는 눈결구의 육각형 모양을 설명하는 대칭성이 있습니다. Figure 1–4에서 보여주는 특정 얼음 패턴은 많은 "홀"이 있으며, 진짜 얼음 구조도 마찬가지입니다. 구조가 깨지는 경우 이러한 홀은 분자가 차지할 수 있습니다. 가장 단순한 물질 중에는 물과 유형의 금속을 제외하고는 녹일 때 확장됩니다. 왜냐하면 고체 얼음에서 원자는 밀접하게 배치되어 있고 녹으면 더 많은 공간이 필요하여 움직일 수 있지만, 열린 구조는 접어들고 있습니다.


물론이죠! 얼음은 "고정된" 단양체 형태를 가지지만, 그 온도는 변할 수 있어요—얼음에는 열이 들어가요. 만약 원하는 대로라면, 그 열의 양을 바꿀 수도 있어요. 얼음에서 열이 무엇인가요? 원자는 움직이지 않고 있지는 않아요. 그들은 떨리는 거리와 진동하고 있어요. 따라서 단양체에는 특정한 순서가 있지만—특정한 구조가 있지만—모든 원자는 그 위치에 고정된 상태에서 진동하고 있어요. 온도를 높일수록 그 진동은 점점 더 크게 되고, 결국에는 위치를 잃어버리게 됩니다. 이 것을 녹는 현상이라고 부르죠. 온도를 낮추면 진동도 점점 줄어들고, 절대영하에서도 원자들은 충분한 움직임을 가지고 있어 얼음이 형성되지 않아요. 절대영하에서도 얼음은 녹지 않는 거예요—단 하나의 예외가 있습니다: 휘헬륨. 휘헬륨은 가능한 한 적게 원자의 움직임을 줄이는 것 뿐이고, 절대영하에서도 충분한 움직임이 있어 얼음이 형성되지 않아요. 휘헬륨은 даже 절대영하에서도 녹지 않는 거예요—단 압력이 원자들이 서로 밀어버리게 만드는 정도로 강해질 때까지요. 만약 압력을 높일수록 그들은 고체로 형성될 수 있어요.


\section{원자 과정들~}
\begin{center}
\begin{diagrambox}{그림 1-5}
\fbox{Missing vector conversion: f01-05\_tc\_big.svgz}
\end{diagrambox}
\end{center}
물질의 종류인 고체, 액체, 기체에 대한 원자 관점에서의 설명은 마치고 이제 우리는 원자의 관점을 통해 여러 과정을 살펴볼 것입니다. 첫 번째로 살펴볼 과정은 물 표면과 관련된 것입니다. 물 표면에서 일어나는 일상이 무엇인가요? 이제 우리는 이미 본 것처럼 물 분자들이 액체 물을 이루고 있지만, 이번에는 물 표면도 볼 수 있습니다. 표면 위에는 물 분자가 있는데요, 이는 우리가 보던 스팀과 같습니다. 이것은 물증기로, 항상 물이 있는 곳에서 발견됩니다. (물증기와 물 사이에 균형이 있다는 점을 나중에 설명하겠습니다.) 또한, 다른 분자도 있습니다—여기서는 두 개의 산소 원자가 서로 묶여 있어 산소 분자가 되었고, 마찬가지로 두 개의 질소 원자가 묶여 있어 질소 분자가 되었습니다. 공기는 거의 전부 질소와 산소, 일부 물증기, 그리고 적은 양의 이산화탄소, 아르곤 등으로 구성되어 있습니다. 따라서 물 표면 위에는 공기가 있고, 이 공기는 물증기를 포함한 기체입니다. 이 사진에서 일어나고 있는 일상이 무엇인가요? 물 분자는 항상 떨어지고 있습니다. 때로는 표면에 있는 한 분자가 보다 강하게 충돌하여 물을 떠납니다. 그림에서는 일어나는 일이 볼 수 없지만, 정지된 사진이라서 그렇습니다. 그러나 우리는 이미지를 훨씬 크게 해보면, 표면 근처의 한 분자가 충돌해서 날아가거나, 또 다른 분자가 충돌해서 날아가는 모습을 상상할 수 있습니다. 따라서 물은 하나씩 분자로 분해되어 사라집니다—즉, 냅니다. 그러나 만약 위에 있는 장치를 닫으면, 시간이 지나면서 공기 중에는 많은 수의 물 분자가 있을 것입니다. 때로는 이 물증기가 날아와서 다시 물 표면에 묶입니다. 따라서 우리는 마치 거기에 두고 있던 깨끗한 물병이라는 것 같지만, 실제로는 항상 변하는 동적인 현상이 일어나고 있는 것을 볼 수 있습니다. 우리 눈으로 보는 것은 변하지 않는 것 같지만, 만약 한 억배가 되면 그 자체의 관점에서 보면 항상 변하고 있다는 것을 알 수 있습니다: 분자가 표면을 떠납니다, 분자가 돌아옵니다.


왜 변화를 보이지 않아요? 그것은 분자가 들어오고 나가는 것이 동일하기 때문입니다! 장기적으로 보면 "아무것도 일어나지 않습니다." 만약 우리가 그릇의 상단을 열고 습한 공기를 제거하고 건조한 공기를 넣는다면, 나가는 분자의 수는 이전과 동일할 것입니다. 그러나 돌아오는 분자는 훨씬 적게 될 것입니다. 왜냐하면 물 위에 있는 수많은 물 분자가 있기 때문입니다. 따라서 더 많은 것이 나가고 들어오지 않기 때문에 물이 끓어갑니다. 그래서 만약 물을 끓이려면 선풍기를 켜야 해요!


여기 또 다른 것 있어요: 어떤 분자가 떠나나요? 분자가 떠나는 것은 보통 에너지보다 조금 더 많은 임시적이고 추가적인 에너지 때문에 발생합니다. 이 에너지는 그가 근처의 분자들의 유도력을 벗어날 수 있도록 필요합니다. 따라서, 떠나가는 분자는 평균 에너지보다 더 많기 때문에, 그들이 남은 분자는 전에 보였던 것보다 적은 평균 운동을 가집니다. 그래서 증발이 일어나면 액체는 점차 냉각됩니다. 물론, 공기가 물 위로 올라가는 분자가 있을 때에는 분자가 표면에 다가올수록 갑자기 큰 유도력이 발생합니다. 이는 들어오는 분자를 빠르게 하고 열을 생성하게 됩니다. 따라서 그들이 떠나갈 때 열을 가져가고, 돌아오면 열을 생성하게 됩니다. 물론, 증발의 총합이 없는 경우에는 결과는 없어요—물의 온도가 변하지 않습니다. 만약 공기를 불어넣어 증발하는 분자의 수가 지속적으로 많아지도록 하면 물은 냉각됩니다. 그래서 국수를 불어넣으려고 해서 국수가 냉각됩니다!


물론이죠, 설명한 프로세스들은 우리가 나타낸 것보다 훨씬 복잡해요. 물이 공기에 들어가고, 시간을 지나면서 산소나 질산 분자가 때때로 물 분자 중 하나와 만나서 "실패"하고 물 속으로 들어가게 되죠. 이렇게 공기는 물 속에 녹아들여지고, 산소와 질산 분자가 물 속으로 들어가며 물에는 공기가 포함됩니다. 갑자기 그릇에서 공기를 뺀다면, 공기 분자는 더 빨리 들어온 것보다 빠르게 나오게 되고, 그렇게 하면서 버블이 만들어질 거예요. 이는 수영자가 매우 나빠요, 알고 있으시죠.


\begin{center}
\begin{diagrambox}{그림 1-6}
\includegraphics[width=0.8\textwidth]{feynman_json/images/f01-06_tc_big.pdf}
\end{diagrambox}
\end{center}
\begin{center}
\begin{diagrambox}{그림 1-7}
\fbox{Missing vector conversion: f01-07\_tc\_iPad\_big\_a.svgz}
\end{diagrambox}
\end{center}
다음으로 다른 과정에 대해 설명하겠습니다. Fig. 1–6에서 우리는 원자 관점에서 물 속에 고체가 녹는 것을 볼 수 있습니다. 만약 소금의 단단체를 물에 넣으면 어떻게 될까요? 소금은 고체, 즉 단단체이며 "소금 원자"로 구성된 정리된 배열입니다. Fig. 1–7은 일반적인 소금인 나트륨염의 세차원 구조를 보여주는 그림입니다. 엄격히 말해 단단체는 원자로 이루어져 있지만, 우리가 불리는 이온으로 이루어져 있습니다. 이온은 몇 개의 추가 전자가 있거나 몇 개의 전자를 잃어버린 원자입니다. 소금 단단체에서는 클로라이드 이온(하나의 전자가 추가된 클로라이드 원자)과 나트륨 이온(하나의 전자가 없는 나트륨 원자)을 찾습니다. 이온들은 고체 소금에서 전기적 유동성에 의해 서로 끌려 있지만, 물에 넣으면 음성 산소와 긍정적 수소의 유동성으로 인해 일부 이온이 벗겨져 나갑니다. Fig. 1–6에서는 클로라이드 이온이 벗겨지고 다른 원자가 물에서 이온 형태로 떠나는 모습을 볼 수 있습니다. 이 그림은 신경 쓰여 제작되었습니다. 예를 들어, 물 분자 중의 수소 부분이 클로라이드 이온에 더 가까운 경향이 있고, 나트륨 이온 근처에서는 산소 부분이 더 많이 있을 가능성이 높아지기 때문입니다. 왜냐하면 나트륨은 긍정적이고 물의 산소 부분은 부정적이기 때문에 전기적으로 유동성에 의해 그렇습니다. 이 그림으로 소금이 물에서 녹는 것인지 단단체가 물에서 형성되는 것인지 알 수 있나요? 물론 알 수 없습니다. 일부 원자가 단단체를 떠나지만 다른 원자가 다시 단단체로 돌아갑니다. 이 과정은 증발과 마찬가지로 동적인 과정이며, 물에 있는 소금의 양이 균형을 이루는 양보다 많거나 적으면 달라집니다. 균형 상태는 원자가 떠나는 속도와 돌아오는 속도가 일치하는 상황을 의미합니다. 물에 거의 소금이 없으면 더 많은 원자가 떠나고 돌아오지 않으므로 소금이 녹습니다. 반면에 "소금 원자"의 양이 너무 많다면 더 많은 원자가 돌아와서 나가지 않으므로 소금이 단단체를 형성합니다.


물론이죠~ 물질의 분자라는 개념은 근사적이고, 특정한 종류의 물질에만 존재하는 거예요. 물의 경우, 세 개의 원자가 실제로 서로 묶여 있다는 걸 명확해요. 하지만 고체 상태의 나트륨염의 경우는 좀 더 모호해요. 여기서는 나트륨과 클로리늄 이온이 큐비케 패턴으로 배치되어 있어요. 그들이 "소금의 분자"로 자연스럽게 그룹화되는 건 쉽지 않아요~


물질과 배설에 대해 이야기하고 있었습니다. 만약 소금의 해수를 높여주면, 원자가 제거되는 속도가 증가하게 되고, 원자가 돌아오는 속도도 증가하게 됩니다. 일반적으로 보면, 어떤 방향으로 진행될지 예측하기는 쉽지 않습니다. 대부분의 물질은 온도가 높아질수록 더 많이 배설되지만, 일부 물질은 온도가 높아질수록 덜 많이 배설됩니다.


\section{물화학 반응~}
지금까지 설명한 모든 과정에서 원자와 이온은 파트너를 바꾸지 않았지만, 당연히 원자가 조합을 바꿀 상황들이 있습니다. 이를 통해 새로운 분자를 형성하게 됩니다. 이것은 Fig. 1–8에 그려진 것처럼 보여집니다. 원자의 파트너 재배치가 발생하는 과정을 우리는 화학 반응이라고 부릅니다. 지금까지 설명한 다른 모든 과정들은 물리적 과정으로 불립니다만, 두 가지 사이에는 명확한 경계가 없습니다. (자연은 무엇이라고 부르는지는 신경 쓰지 않고 계속해서 일하고 있습니다.) 이 그림은 탄소가 산소에 태어나는 것을 나타내야 합니다. 산소의 경우, 두 개의 산소 원자가 매우 강하게 서로 결합됩니다. (왜 세 개 또는 даже 네 개가 결합하지 않는 거야? 이것이 같은 원자 과정에서 가장 특이한 특징 중 하나야. 원자는 매우 특별하다: 특정한 파트너를, 특정한 방향을 좋아하는 등등. 물리학의 임무는 각각이 무엇을 원하는 이유를 분석하는 것이다. 적어도 두 개의 산소 원자가 결합하여 포화하고 행복하게 분자를 형성합니다.)


\begin{center}
\begin{diagrambox}{그림 1-8}
\fbox{Missing vector conversion: f01-08\_tc\_big.svgz}
\end{diagrambox}
\end{center}
탄소 원자는 고체 크리스탈에 있어야 합니다. 이 크리스탈은 그래핀이나 다이아몬드일 수 있습니다. 예를 들어, 산소 분자가 탄소로 올라와서 각 원자는 탄소 원자를 잡고 새로운 조합인 "탄소-산소"로 날아갑니다. 이는 가스로 불리는 탄소 다이옥사이드의 분자입니다. 화학적으로 CO라고 부릅니다. 매우 단순합니다. "CO"라는 글자는 거의 그 분자를 나타내고 있습니다. 그러나 탄소가 산소보다 산소가 산소나 탄소가 탄소를 더 끌어들입니다. 따라서 이 과정에서 산소는 적은 에너지만, 산소와 탄소는 충격적으로 결합하고 혼란을 일으킵니다. 그들의 근처에는 모든 것이 에너지를 흡수하게 됩니다. 그래서 많은 운동 에너지, 즉 열에너지가 생성됩니다. 물론 이 것은 불이지만, 산소와 탄소의 조합으로 열을 얻고 있습니다. 열은 보통 열기체 분자의 운동 에너지로 나타납니다만, 특정 상황에서는如此 큰 에너지를 발산하여 빛을 발생시킬 수 있습니다. 이렇게 불꽃이 생깁니다.


탄소 산화물도 거의 만족하지 않습니다. 다른 산소를 더 붙일 수 있으므로, 우리는 탄소와 산소가 결합하는 복잡한 반응을 가질 수 있습니다. 이 때, 동시에 탄소 산화물 분자가 충돌할 수도 있습니다. 한 개의 산소 원자가 CO에 붙어서 결국 탄소와 두 개의 산소로 구성된 분자를 형성하게 될 것입니다. 이 분자는 CO$_2$라고 지칭되며, 탄소 dioxide라고 불립니다. 자동차 엔진 같은 곳에서 매우 빠른 반응으로 탄소를 태우면(예를 들어, 폭발이 너무 빨라서 탄소 dioxide가 만들어질 시간이 없을 때), 상당한 양의 탄소 산화물이 형성될 수 있습니다. 많은 이러한 재배열에서 매우 큰 양의 에너지가 발산되어 폭발, 불꽃 등이 발생할 수 있습니다. 화학가는 이 원자들의 배열을 연구했으며, 모든 물질은 어떤 종류의 원자 배열이라고 발견했습니다.


이 아이디어를 보여드리기 위해 다른 예를 들어보겠습니다. 작은 비올레트 풍경에 들어가면 "그 냄새"는 무엇인지 알고 있습니다. 그것은 특정한 분자나 원자의 배열일 것입니다. 이 분자가 우리 코로 어떻게 들어갔는지 알아볼까요? 이것은 상당히 쉽습니다. 만약 냄새가 공기 중의 어떤 종류의 분자라면, 그 분자는 무작위로 움직이며 충돌을 겪고 있을 것입니다. 따라서 이 분자가 실수로 코로 들어갈 수 있습니다. 물론 그것은 우리 코로 들어가려는 특별한 욕구는 없습니다. 그것은 단순히 무력한 한 부분일 뿐, 충돌하는 분자들의 군에서 하나의 일부분일 뿐입니다. 이 무작위 움직임으로 인해 이 특정 물질이 코로 들어가게 되었습니다.


\begin{center}
\begin{diagrambox}{그림 1-9}
\includegraphics[width=0.8\textwidth]{feynman_json/images/f01-09_tc_big.pdf}
\end{diagrambox}
\end{center}
물리학 강의 노트를 한국어로 번역하는 번역가입니다. 린카드 페인먼의 전신적인, 친절하고 흥미진진한 분위기를 유지하며 번역합니다.

---

지금 화학가는 꽃향기를 가진 특별한 분자를 이용해 그 분자의 구조를 분석하고 원자들이 공간에서 어떻게 배열되어 있는지 알려줄 수 있습니다. 우리는 카르복테이트 분자가 직선이고 대칭적이라는 것을 알고 있습니다: O—C—O. (물리학적 방법으로도 쉽게 확인할 수 있습니다.) 그러나 화학에서 원자의 훨씬 복잡한 배열까지도, 감찰의 긴 과정을 통해 원자들의 배열을 찾아낼 수 있습니다. 그림 1-9는 빅토리아 근처의 공기 사진입니다. 다시 보면 공기는 질소와 산소, 그리고 물분유가 포함되어 있습니다. (물분유가 왜 있는지? 빅토리아가 습기 때문에요. 모든 식물이 수증후로 물을 분비하므로입니다.) 그러나 우리는 또한 카ربون, 헥사곤, 산소 원자로 구성된 "몬스터"를 볼 수 있습니다. 이는 카르복테이트보다 훨씬 복잡한 배열이며, 실제로는 엄청나게 복잡한 배열입니다. 아쉽게도 우리는 그 분자의 정확한 구조를 이미지화할 수 없습니다. 화학적으로 알려진 것처럼 모든 원자가 실제로는 세 개의 차원에서 배치되어 있지만, 우리의 이미지는 두 개의 차원에만 있습니다. 루프를 이루는 여섯 개의 카르복테이트는 평평한 루프가 아니라 "접어있는" 형태의 루프입니다. 모든 각도와 거리가 알려져 있습니다. 따라서 화학식은 이런 분자의 이미지를 나타냅니다. 화학가는 검정판에 이렇게 적을 때, 두 개의 차원에서 대략적으로 그림을 그리는 것을 시도하고 있습니다. 예를 들어, 우리는 여섯 개의 카르복테이트가 구성한 "루프"와 끝에 첨부된 "체인"을 볼 수 있으며, 끝에서 두 번째 위치한 산소 원자와 그 카르복테이트에 묶여 있는 세 개의 헥사곤, 여기서 위쪽으로 오르는 두 개의 카르복테이트와 세 개의 헥사곤 등등을 볼 수 있습니다.


\begin{center}
\begin{diagrambox}{그림 1-10에 나타난 물질은 $\alpha$-아이론입니다~.}
\fbox{Missing vector conversion: f01-10\_tc\_big.svgz}
\end{diagrambox}
\end{center}
원자들의 배열을 어떻게 화학가가 찾는지 알아보세요. 그들은 다양한 재료를 섞어보고, 만약 빨간색이 되면 거기에는 한 휘트로몬과 두 카본이 묶여 있다는 것을 알려줍니다. 반대로 파란색이 되면 그렇지 않다는 것을 의미합니다. 이것은 가장 멋진 사라지다니는 해결 방법 중 하나인 유기 화학입니다. 원자들의 배열을 이런 엄청난 복잡한 구조에서 알아내려고 화학가는 서로 다른 재료를 섞었을 때 일어나는 현상을 관찰합니다. 물리학자는 화학자가 원자들의 배열에 대해 말하는 것을 믿지 못했습니다. 대략 20년 동안, 일부 경우에는 이러한 분자를(이보다 더 복잡하지만 일부는 그의 일부를 포함한) 물리적 방법으로 보고, 각각의 원자를 색깔로 찾아보는 것이 아니라 위치에서 찾을 수 있었습니다. 그리고 놀랍게도! 화학가는 거의 항상 정확합니다.


우리가 알게 된 것은, 빛나는 화려한 꽃의 향에서 실제로 세 가지 약간 다른 분자가 있습니다. 이 분자는 단지 휴리미드 원자의 배열이 다르다는 점에서 서로 다릅니다~


물질을 이름 붙이는 것이 화학의 하나의 문제입니다. 어떤 물질이 무엇인지 알 수 있도록 이름을 찾아야 합니다. 이 모양에 이름을 지어봅시다! 이름은 모양뿐만 아니라 여기에는 산소 원자가 있고, هناك에는 휴산원자 등 각 원자의 종류와 위치를 정확하게 전달해야 합니다. 따라서 화학식의 이름이 완전해지려면 복잡해야 할 것입니다. 당신은 이 물체의 더 완전한 이름인 4-(2, 2, 3, 6-테트라메틸-5-цикл로헤닐)-3-부텐-2-올리가 있다는 것을 보고 있습니다. 그 이름이 이 구조를 알려주며, 이것이 원자들의 배열을 나타냅니다. 화학가는 이러한 어려움을 겪고 있지만, 단어로 분자를 설명하는 데 매우 힘든 문제에 직면하고 있습니다.


원자가 있는지 어떻게 알 수 있나요? 이전에 언급한 하나의 트릭 중 하나로는, 원자를 가정하고 순서대로 실험 결과가 예측한 것처럼 나옵니다. 만약 원자가 항상 움직이고 있다고 가정한다면, 물 속에서 큰 공을 던져 놓으면 그 공은 불규칙하게 움직입니다. 이는 푸시볼 게임과 비슷합니다. 많은 사람들이 큰 공을 푸시하면서 공이 필드를 불규칙하게 움직이는 것처럼요. 그래서, 마찬가지로 "큰 공"은 한쪽에서 다른 쪽으로의 충돌 불균형 때문에 움직입니다. 따라서, 매우 작은 입자(콜라겐)을 물 속에서 탁월한 미세경로를 통해 보면, 입자가 원자의 폭발적인 충돌에 의해 지속적으로 움직이는 모습을 볼 수 있습니다. 이 현상은 브라우니안 운동이라고 불립니다.


우리는 원자의 구조가 결정적인 증거로 보이게 됩니다. X선 분석으로 추론된 구조는 자연에서 나타나는 결정체의 실제 형태와 공간적 "형태"에 일치합니다. 결정체의 여러 "면" 사이의 각도는 원자로 구성된 여러 "층"이라는 가정을 토대로 추론된 각도와 밀리초 단위 내에서 동일합니다.


모든 것이 원자로 이루어져 있다는 것이죠. 이는 핵심 가설이에요. 예를 들어 생물학에서 가장 중요한 가설 중 하나는 모든 동물이 하는 일, 즉 모든 것이 원자가 물리 법칙에 따라 행동한다는 것입니다. 말해보면, 살아있는 것들이 하는 모든 일이 물리법칙에 따라 작동하는 원자로 이루어져 있다는 관점에서 이해할 수 없는 것은 없어요. 이건 처음부터 알지 않았어요: 이것을 가설로 제안하는데 실험과 이론이 필요했죠. 그러나 이제는 받아들여지고 있고, 생물학 분야에서 새로운 아이디어를 만들어내는데 가장 유용한 이론이라고 여겨요.


스테인리스 또는 소금 같은 원자들이 서로 붙어있는 철이나 솔트가 이런 흥미로운 성질을 가질 수 있다는 거야. 물이 이런 작은 루프들로, 지구 위에서 마일마일 동일한 것이 반복되는 것처럼 파동과 밀기를 만들고, 바닥에 닿으면 빠르게 흘러가는 소리를 내고, 이상한 패턴을 만드는 거야. 이런 모든 것, 물의 생명력이 단순히 원자들로 구성된 쌓임이라고 생각하면, 더 많은 가능성은 있을 거야. 만약 원자를 정해진 패턴으로 일관되게 배치하는 대신, 한 곳에서 다른 곳까지 항상 다르게 배치하고, 다양한 종류의 원자가 여러 가지 방식으로 변동하며, 반복되지 않고 지속적으로 변화한다면, 이것이 어떻게 행동할 수 있을지 얼마나 더 신이 뛰거울 거야? 당신 앞에 걸어가고 말하는 "그것"이 단순히 원자들로 구성된 큰 구체일 수도 있어. 이 구체의 극단적인 복잡성이 상상력을 놀라게 만들 수 있다. 우리가 원자들로 쌓임이라고 말할 때, 우리는 단순히 원자들로 쌓임이 아니라, 반복되지 않는 배치로 다양한 종류의 원자가 여러 가지 방식으로 변동하며 지속적으로 변화할 수 있는 가능성이 있다는 걸 의미해요.


\end{document}